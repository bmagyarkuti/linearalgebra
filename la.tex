\documentclass[9pt,showtrims]{memoir}
\let\Aref\relax
\usepackage[x11names]{xcolor}
%%%%%%% pdflatex %%%%%%%%%%
%\usepackage[T1]{fontenc}
%\usepackage[utf8]{inputenc}
%\usepackage[hungarian]{babel}[2015/11/24]
%%%%%%% pdflatex %%%%%%%%%%

%%%%%% lualatex %%%%%%%%%%%
%\usepackage{polyglossia}\setdefaultlanguage{magyar}
\usepackage[hungarian]{babel}[2015/11/24]
\usepackage{fontspec}
\defaultfontfeatures{Ligatures=TeX}
\setmainfont{TeX Gyre Pagella}
\setsansfont{Kurier}[Scale=MatchLowercase]
\setmonofont{inconsolata}[Scale=MatchLowercase]
%%%%%% lualatex %%%%%%%%%%%

\frenchspacing
\usepackage{amsmath,amsthm,amsfonts,amssymb,fixme}
\usepackage{natbib}\bibliographystyle{natdin-es}
\fxsetup{status=draft, theme=color, layout={inline}}
\renewcommand{\fixmelogo}{\textcolor{black}{\colorbox{Firebrick1}{\textsf{\textbf{FIX}}}}}

\usepackage[unicode]{hyperref}\hypersetup{final}\usepackage{memhfixc}


\usepackage[a4paper]{geometry}
%\usepackage[missing={(Oh my!)},dirty={Oh no!},mark]{gitinfo2}
\usepackage[mark,dirty={(Dirty)}]{gitinfo2}
%\edef\gitBranch{\gitBranch}
\renewcommand{\gitMark}{References: \gitReferences\,@\,\gitFirstTagDescribe{}
    \textbullet{}
    Date: \gitAuthorIsoDate
}


%\setsecheadstyle{\Large\sffamily\bfseries\raggedright}
%\setsubsecheadstyle{\large\sffamily\bfseries\raggedright}
%\setsubsubsecheadstyle{\sffamily\bfseries\raggedright}


\nouppercaseheads
%\makeoddhead{myheadings}{\footnotesize\selectfont\sffamily\leftmark}{}{\footnotesize\selectfont\sffamily\thepage}
%\makeevenhead{myheadings}{\footnotesize\selectfont\sffamily\thepage}{}{\footnotesize\selectfont\sffamily\rightmark}
\makeoddhead{myheadings}{\footnotesize\leftmark}{}{\footnotesize\thepage}
\makeevenhead{myheadings}{\footnotesize\thepage}{\footnotesize\myBotmark}{\footnotesize\rightmark}
\makepsmarks{myheadings}{%
    \renewcommand\chaptermark[1]{%
    \markboth{%
%      \ifnum \value{secnumdepth} > 1
%      \if@mainmatter %
            \thechapter.~\chaptername:~%
%      \fi
%      \fi
        ##1}{\thepart.~\partname}}%
    }

\makeatletter
\newcommand\arraybslash{\let\\\@arraycr}
\patchcmd{\@makechapterhead}
    {\printchaptername \chapternamenum \printchapternum}
    {\printchapternum.\@\chapternamenum \printchaptername}
    {}{}
\renewenvironment{proof}[1][\proofname]
    {\par\pushQED{\qed}%
    \normalfont \topsep6\p@\@plus6\p@\relax
    \trivlist
    \item[\hskip\labelsep
        \itshape
    #1\@addpunct{:}]\ignorespaces}
    {\popQED\endtrivlist\@endpefalse}
\makeatother

\newcommand{\addQEDstyle}[2]{\AtBeginEnvironment{#1}{\pushQED{\qed}\renewcommand{\qedsymbol}{#2}}\AtEndEnvironment{#1}{\popQED}}
%% qed trükkök:
%% https://tex.stackexchange.com/questions/16453/denoting-the-end-of-example-remark
%\swapnumbers %% a magyar.ldf megfordítja. A polyglossia nem. De a magyar ldf pontot is tesz a számcimke után

\renewcommand{\qedsymbol}{$\centerdot$}
\newcommand{\myqedsymbol}{$\lrcorner$}
\theoremstyle{plain}

\newtheorem{proposition}{állítás}[section]
\newtheorem{lemma}[proposition]{lemma}
\newtheorem*{SL}{Steinitz-lemma}
%
\theoremstyle{remark}
\newtheorem{note}[proposition]{megjegyzés}

\theoremstyle{definition}
\newtheorem{definition}[proposition]{definíció}
\newtheorem{corollary}[proposition]{következmény}
\newtheorem{defprop}[proposition]{definíció-állítás}
\addQEDstyle{definition}{\myqedsymbol}\addQEDstyle{proposition}{\myqedsymbol}\addQEDstyle{lemma}{\myqedsymbol}\addQEDstyle{note}{\myqedsymbol}\addQEDstyle{corollary}{\myqedsymbol}\addQEDstyle{SL}{\myqedsymbol}\addQEDstyle{defprop}{\myqedsymbol}


%%% https://tex.stackexchange.com/questions/319474/put-current-theorem-like-items-name-number-in-header
% \myBotmark feltöltése a lapon lévő utolsó tétel környezettel
\makeatletter
    \@ifdefinable\@my@claim@mark{\newmarks\@my@claim@mark}
    \newcommand*\myMark[1]{\marks\@my@claim@mark{#1}}
    \newcommand*\myBotmark{\botmarks\@my@claim@mark}
    \patchcmd{\@begintheorem}{% search for:
        \thm@swap\swappedhead\thmhead % more specific than before
    }{% replace with:
        \myMark{#2.\@ifnotempty{#1}{\ #1}\@ifnotempty{#3}{\ (#3)}}%
        \thm@swap\swappedhead\thmhead
    }{
        \typeout{>>> Made patch specific for amsthm.}
    }{
        \typeout{>>> Patch specific for amsthm FAILED!}
    }

 %part
\long\def\@part[#1]#2{%
  \M@gettitle{#1}%
  \def\f@rtoc{#1}%
  \@nameuse{part@f@rtoc@before@write@hook}%
  \phantomsection
  \mempreaddparttotochook
  \ifnum \c@secnumdepth >-2\relax
    \refstepcounter{part}%
    \addcontentsline{toc}{part}%
      {\protect\partnumberline{\thepart}\f@rtoc}%
    \mempartinfo{\thepart}{\f@rtoc}{#2}%
  \else
    \addcontentsline{toc}{part}{\f@rtoc}%
    \mempartinfo{}{\f@rtoc}{#2}%
  \fi
  \mempostaddparttotochook
  \partmark{#1}%
  {\centering
   \interlinepenalty \@M
   \parskip\z@
   \normalfont
   \ifnum \c@secnumdepth >-2\relax
     \printpartnum.\ \printpartname \partnamenum%%MGy
     \midpartskip
   \fi
   \printparttitle{#2}\par}%
  \@endpart}
\makeatother

\renewcommand{\mathbf}{\mathbb}
\DeclareMathOperator{\lin}{lin}
\DeclareMathOperator{\crank}{crank}
\DeclareMathOperator{\rrank}{rrank}
\DeclareMathOperator{\srank}{srank}
\DeclareMathOperator{\rank}{rank}
%\DeclareMathOperator{\arg}{arg}

\def\scwords #1 #2 #3 {\textsc{#1} \textsc{#2} \textsc{#3} }
\citeindextrue
\makeindex
\synctex=1
\begin{document}
\frontmatter*
\section*{Verzió információk}
\begin{center}
\begin{tabular}{l|r}
    \hline
References & \gitReferences\\
Branch & \gitBranch\\
Dirty & \gitDirty\\
Hash&\gitHash\\
Author Iso Date & \gitAuthorIsoDate\\
\hline
First Tag Describe & \gitFirstTagDescribe \\
Reln& \gitReln \\
Roff & \gitRoff \\
Tags & \gitTags \\
Describe & \gitDescribe \\
\hline
\end{tabular}
\end{center}
\chapter*{Előszó}
\scwords%
A legfontosabb forrás \citep{DancsPuskas2001}.

\dots

Igyekszem strukturáltan írni.
Ennek oka, hogy evvel hangsúlyozzam, hogy az olvasónak igyekeznie kell struktúráltan gondolkodni.
A hátulütője, hogy hibásan azt a helytelen képzetet keltheti, 
Nem, nem erről van szó. 
A puzzle-ban minden elem egyenrangú, az egyik elem hiánya éppen annyira fájdalmas mint a másiké.
Itt nem erről van szó, ez egyetlen matematikai diszciplína esetében sem igaz!
Az olvasónak igyekeznie kell, 
hogy meglássa mi a legfontosabb gondolat a sok-sok állításnak, 
mint építménynek egy-egy ,,nyilvánvaló következményében''.
Kicsi, atomszerű építőkövek egymás utáni megértése visz az anyagban előre, 
ezek az egymástól feltűnő módon szeparált állítások és azok érvekkel való alátámasztása,
amit bizonyításnak is szokás mondani.
Hogy e kis lépések egymástól még határozottabban váljanak el azt az írás
typográfiája is erősíti azzal, 
az állítás-szerű környezeteket a \,\myqedsymbol~, 
és a bizonyítás környezetet a \,\qedsymbol~ karakterekkel zárom le.


\bigskip\noindent 
Magyarkuti Gyula
\hfill{Budapest, \ontoday}


\clearpage
\tableofcontents*
\pagestyle{myheadings}
\mainmatter*
\part{F-dúr hegedűverseny No. 3, Op. 8, RV 293, ,,L'autunno''}
\chapter{Előzmények}
\scwords A lineáris algebra tárgyalásához elengedhetetlenül szükséges általános algebrai ismereteket foglaljuk össze.
\section{Algebrai struktúrák}
\begin{definition}[$n$-változós művelet]\index{művelet}\index{algebrai struktúra}
    Legyen $H$ egy halmaz. Egy 
    \[
        \varphi\colon H^n\to H
    \]
    függvényt $n$-változós \emph{műveletnek} nevezünk.
    Egy halmazt és rajta véges sok műveletet együtt \emph{algebrai struktúrának} mondunk.
    Jelölés: 
    $$\left(H,\varphi_1,\dots,\varphi_n  \right),$$ 
    ahol $H$ a halmaz és
    $\varphi_1,\dots,\varphi_n$ a $H$ halmazon értelmezett műveletek.
\end{definition}
\begin{definition}[félcsoport]\index{félcsoport}
    Egy $\left( S,\star \right)$ algebrai struktúrát \emph{félcsoportnak} mondjuk,
    ha $\star$ egy kétváltozós \emph{asszociatív}\index{asszociatív}
    művelete az $S$ halmaznak,
    azaz minden $a,b,c\in S$ mellett
    \[
        a\star\left( b\star c \right)=\left( a\star b\right)\star c.\qedhere
    \]
\end{definition}
Lefordítva ez azt jelenti, hogy 
\begin{enumerate}
    \item minden $a,b\in S$ mellett $a\star b\in S$, és
    \item minden $a,b,c\in S$ esetén $a\star\left( b\star c \right)=a\star\left( b\star c \right)$
\end{enumerate}
\begin{definition}[neutrális elem]\index{neutrális elem}\index{neutrális elemes félcsoport}
    Az $\left( S,\star \right)$ félcsoportban az $s\in S$ elem \emph{balról (jobbról) neutrális},
    ha $s\star t=t$ ($t\star s=t$) minden $t\in S$ mellett.
    Ha $s\in S$ balról is és jobbról is neutrális, akkor $s$-et egy \emph{neutrális elemnek}
    mondjuk.
    A félcsoportot \emph{neutrális elemes félcsoportnak} nevezzük, ha van benne neutrális elem.
\end{definition}
\begin{proposition}
    Ha egy félcsoportban, van egy balról neutrális elem és egy jobbról neutrális elem, 
    akkor ezek megegyeznek. 
    Emiatt egy neutrális elemes félcsoportban neutrális elem csak egy van.
\end{proposition}
\begin{proof}
    Legyen $s_1$ balról-- és $s_2$ jobbról neutrális elem.
    Ekkor
    \(
        s_1=s_1\star s_2=s_2.
    \)
\end{proof}
A félcsoport additív írásmódja esetén természetes a neutrális elemet \emph{zérusnak},
míg multiplikatív írásmód esetén \emph{egységnek} nevezni.
\begin{definition}[csoport]\index{csoport}
    Egy $\left( G,\star \right)$ algebrai struktúrát \emph{csoportnak} nevezünk,
    ha neutrális elemes félcsoport, amelyben minden $g\in G$-hez létezik $g'\in G$, hogy
    \[
        g\star g'=e=g'\star g.\tag{\dag}
    \]
    Itt $e\in G$ jelöli a $G$ csoport neutrális elemét.
\end{definition}
\begin{defprop}[inverz elem]\index{inverz}
    Legyen $\left( G,\star \right)$ egy csoport.
    Ekkor minden $g\in G$-hez, csak egyetlen $g'\in G$ létezik, 
    amelyre a fenti ($\dag$) azonosság fennáll.
    Adott $g$-hez ezt ez egyetlen $g'\in G$ elemet, 
    amelyre ($\dag$) teljesül a $g$ elem \emph{inverzének} mondjuk.
\end{defprop}
\begin{proof}
    Tegyük fel, hogy $g',g''$ inverz elemei $g$-nek.
    Azt mutatjuk meg, hogy ha $g'$ baloldali-- és $g''$ jobboldali inverze $g$-nek,
    akkor a két elem megegyezik:
    \[
        g'=g'\star e=
        g'\star\left( g\star g'' \right)=
        \left(g'\star g\right)\star g'' =
        e\star g''=g''.\qedhere
    \]
\end{proof}
Példaként gondoljuk meg, hogy a $H\to H$ függvény halmaza a kompozíció művelettel
neutrális elemes félcsoport, és a $H\to H$ kölcsönösen egyértelmű függvények halmaza a kompozíció művelettel csoportot alkotnak. 
Ez utóbbi csoportot mondjuk \emph{permutáció csoportnak}\index{permutációk}.
\begin{proposition}[egyszerűsítési szabály]\index{egyszerűsítési szabály}
    Csoportban igaz az egyszerűsítési szabály, azaz
    \[
        a\star c=b\star c\implies a=b.\qedhere
    \]
\end{proposition}
\begin{proof}
    \begin{math}
        a=a\star e
        =
        a\star \left( c\star c'\right)=
        \left( a\star c \right)\star c'=
        \left( b\star c \right)\star c'=
        b\star\left( c\star c' \right)=
        b\star e=
        b.
    \end{math}
\end{proof}
\begin{definition}[Abel--csoport]\index{Abel--csoport}
    Egy $\left( G,\star \right)$ csoportot \emph{Abel--csoportnak} nevezünk,
    ha a művelete \emph{kommutatív}\index{kommutatív} is, 
    azaz minden $s,t\in G$ mellett $s\star t=t\star s$.
\end{definition}
\begin{definition}[gyűrű]\index{gyűrű}
    A kétműveletes $\left( R,+,\cdot \right)$ algebrai struktúrát \emph{gyűrűnek} nevezzük,
    ha
    \begin{enumerate}
        \item $\left( R,+ \right)$ Abel--csoport;
        \item $\left( R,\cdot \right)$ félcsoport;
        \item és a két műveletet összeköti a következő két disztributivitás:\index{disztributív}
            \[
                a\cdot\left( b+c \right)=a\cdot b + a\cdot c\qquad
                \left( a + b \right)\cdot c=a\cdot c+a\cdot b.
            \]
    \end{enumerate}
    Ha $\left( R,\cdot \right)$ neutrális elemes félcsoport, akkor azt mondjuk, hogy $R$ egy
    \emph{egységelemes gyűrű}, és ha $\left( R,\cdot \right)$ kommutatív félcsoport, akkor
    azt mondjuk, hogy $R$ egy \emph{kommutatív gyűrű}.
\end{definition}
\begin{definition}[test]
    Egy $\left( \mathbb{F},+,\cdot \right)$ kétműveletes algebrai struktúrát \emph{testnek}
    nevezünk, 
    ha olyan kommutatív egységelemes gyűrű, 
    amelyben minden nemzérus\footnote{értsd: minden elemnek, amely a $+$ műveletre nézve neutrális elemtől különbözik}
    elemnek van inverze\footnote{értsd: a $\cdot$ szorzás neutrális elemére mint egységelemre nézve}, 
    és $0\neq 1$\footnote{érts: az összeadásra nézve és a szorzásra nézve képzett neutrális elemek nem azonosak.}.
\end{definition}
A test az algebrai struktúra, ahol a az összeadás és szorzás műveletekkel úgy számolhatunk, mint amit a valós számok során megszoktuk.
Példaként néhány tulajdonság.
\begin{proposition}
    Az $\left( R,+,\cdot \right)$ gyűrűben minden $a\in R$ mellett 
    \begin{equation*}
        a\cdot 0=0\text{ és }
        \left( -1 \right)a=-a.\qedhere
    \end{equation*}
\end{proposition}
\begin{proof}
    \begin{math}
        0+a\cdot 0=
        a\cdot 0=
        a\left( 0+0 \right)=
        a\cdot 0+a\cdot 0.
    \end{math}
    A jobboldali $a\cdot 0$-val való egyszerűsítés után kapjuk, 
    hogy $0=a\cdot 0$.
    A második azonosságot az első felhasználásával kapjuk:
    \begin{math}
        0
        =
        0a
        =
        \left( 1+\left( -1 \right) \right)\cdot a
        =
        1\cdot a + \left( -1 \right)\cdot a
        =
        a +\left( -1 \right)\cdot a.
    \end{math}
    Az additív inverz definíciója szerint ez éppen azt jelenti, hogy $-a=\left( -1 \right)\cdot a$.
\end{proof}
Ami nagyon fontos, hogy egy gyűrűben nem feltétlen teljesül, 
hogy elemek szorzata csak úgy lehet zérus, ha legalább az egyik elem zérus.
Számunkra a legfontosabb példa  a mátrixok gyűrűje\footnote{lásd kicsit később},
ahol pont ennek a hiánya jelenti nehézséget.

Egy testben ilyen nem fordulhat elő.
\begin{definition}[nullosztómentes gyűrű]
    Egy gyűrűt \emph{nullosztómentesnek} nevezzük, 
    ha két elem szorzata csak úgy lehet nulla, 
    ha legalább az egyik elem nulla.
\end{definition}
\begin{proposition}
    Egy test egyben nullosztómentes gyűrű, azaz
    ha $\mathbf{F}$ egy test, és $a,b\in\mathbf{F}$.
    Akkor 
    \[
        ab=0\implies a=0\text{ vagy }b=0.\qedhere
    \]
\end{proposition}
\begin{proof}
    Tegyük fel, hogy $ab=0$.
    Ha $b\neq 0$, akkor létezik $b'\in\mathbf{F}$, hogy $bb'=1$.
    Így 
    \[
        0= 0b'=\left( ab \right)b'=a\left( bb' \right)=a1=a.\qedhere
    \]
\end{proof}
\begin{proposition}
    Nullosztómentes gyűrűben nem zérus elemmel való szorzatot egyszerűsíteni lehet
    azaz, ha $a,b,c\in R,b\neq 0$ esetén
    \[
        ab=cb\implies a=c.\qedhere
    \]
\end{proposition}
\begin{proof}
    \begin{math}
        \left( a-c \right)b=ab-cb=0\implies a-c=0.
    \end{math}
\end{proof}
\begin{definition}[ideál]\index{ideál}\index{főideál}\index{főideál-gyűrű}
    Egy $\left( R,+,\cdot \right)$ kommutatív gyűrű egy $J\subseteq R$ nemüres részhalmazát
    \emph{ideálnak} nevezzük,
    ha 
    \begin{enumerate}
        \item 
        minden $a,b\in J$ mellett $a+b\in J$;
    \item
        minden $c\in R$ és minden $a\in J$ mellett $ca\in J$.
    \end{enumerate}
    Ha egy $d\in R$ adott, akkor a 
    \[
        \left\{ da:a\in R \right\}
    \]
    halmaz egy ideálja $R$-nek. 
    Ez a $d$ elem többszöröseiből álló ideál, amelyet \emph{főideálnak}\index{főideál} is nevezünk.
    Ha egy gyűrűben minden ideál egy főideál, akkor a gyűrűt \emph{főideál-gyűrűnek}\index{főideál-gyűrű} mondjuk.
\end{definition}
A generált ideál fogalma nagyon fontos.
\begin{defprop}[generált ideál]\index{generált ideál}
    Legyen adott a  kommutatív, egységelemes $\left( R,+,\cdot \right)$ gyűrűben véges sok $a_1,\dots,a_r$ elem.
    Az e véges sok elemet tartalmazó ideálok közös része maga is ideál, 
    és e metszet az eredeti véges halmazt
    tartalmazó \emph{legszűkebb ideál}.
    Jelöljük ezt $J\left( a_1,a_2,\dots,a_r \right)$ módon.

    Tekintsük a 
    $
\left\{ \sum_{j=1}^ra_jb_j:b_1,\dots,b_r\in R \right\}
    $ 
    halmazt.
    Világos, 
    hogy ez egy ideál az $R$ gyűrűben. 
    A gyűrű egységelemes, 
    ezért ennek $J\left( a_1,\dots,a_r \right)$ egy részhalmaza.
    Másrészt minden az $\left\{ a_1,\dots,a_r \right\}$ elemeket tartalmazó ideál, 
    egyben tartalmazza a
    $
    \left\{ \sum_{j=1}^ra_jb_j:b_1,\dots,b_r\in R \right\}
    $ 
    halmazt is,
    ami azt jelenti, hogy 
    \[
        J\left( a_1,\dots,a_r \right)=
        \left\{ \sum_{j=1}^ra_jb_j:b_1,\dots,b_r\in R \right\}
    \]
    az $a_1,\dots,a_r$ elemeket tartalmazó legszűkebb ideál.
    Nevezzük ezt az ideált az $a_1,\dots,a_r$ elemek \emph{generálta ideálnak} is.
\end{defprop}



Világos, hogy $\left\{ 0 \right\}$ és maga az egész $R$ ideálok.
A legfontosabb struktúrák számunkra a következők:
\begin{itemize}
    \item 
    Egységelemes gyűrű, amelyben a nullosztómentesség nem teljesül: mátrixok.
    \item
    Kommutatív egységelemes gyűrű, amely nullosztómentes de mégsem test: polinomok.
    \item
    Test:
    a valós vagy a komplex számok.
\end{itemize}
\section{Polinomgyűrűk}
\begin{definition}[polinom]\index{polinom}
    Legyen $\mathbf{F}$ egy test.
    E test feletti polinomokon az összes 
    \[
        p\left( t \right)=
        \alpha_0+\alpha_1t+\alpha_2t^2+\dots+\alpha_nt^n
    \]
    alakú formális algebrai kifejezést értjük.
    Itt $n$ tetszőleges nem negatív egész 
    és $\alpha_0,\dots,\alpha_n$ tetszőleges, az $\mathbf{F}$ testbeli elemek.
    Az $\mathbf{F}$ test feletti összes polinomok halmazát $\mathbf{F}\left[ t \right]$ módon jelöljük.
\end{definition}
A fenti definícióban az \emph{algebrai kifejezés} szó arra utal,  hogy az
\begin{math}
        \alpha_0+\alpha_1t+\alpha_2t^2+\dots+\alpha_nt^n
\end{math}
műveletek minden $t\in\mathbf{F}$ mellett értelmesek, 
és eredményük egy újabb $\mathbf{F}$ testbeli elem.
Ha $t\in\mathbf{F}$ konkrétan meg van adva, 
akkor a behelyettesítés után kapott elemet mondjuk a $p$ polinom helyettesítési értékének.

A \emph{formális algebrai kifejezés}\index{formális algebrai kifejezés} arra utal, 
hogy egy polinomot az együtthatói határozzák meg, 
azaz két polinom akkor és csak akkor azonos,
ha az összes együtthatói azonosak.
Ez szemben áll avval, hogy ha a polinomokra mint függvényekre tekintenénk, 
akkor a helyettesítési értékek egyenlősége jelentené a két polinom azonos voltát.
A formális szó tehát azt jelenti, hogy nem mint függvényre gondolunk, 
hanem egyszerűen az adott $\alpha_0,\dots,\alpha_n$ rögzített elemek -- ezeket mondjuk együtthatóknak --,
által előírt műveletekre. 
Az az előírás ugyanis, hogy tetszőleges $t\in\mathbf{F}$ mellett hajtsuk végre az
\[
        \alpha_0+\alpha_1t+\alpha_2t^2+\dots+\alpha_nt^n
\]
műveletsort. 
A műveletsorról és nem annak eredményéről van szó. 


Két műveletsor akkor azonos, ha ugyanazok a műveletsort meghatározó 
$\left( \alpha_0,\alpha_1,\dots,\alpha_n \right)$%
\footnote{Az előbbi zárójellel azt hangsúlyozzuk, hogy az együtthatók sorrendje is számít.}
együtthatók.%
\footnote{Persze felmerül a kérdés, 
    hogy ha két polinom minden helyettesítési értéke azonos,
    akkor igaz-e, 
    hogy mint formális polinomok is azonosak,
    tehát a két polinom együtthatói is rendre azonosak-e?
    A pozitív választ később látjuk nem véges számtest, például a valós vagy a komplex test, feletti polinomok esetén.
    Lásd \aref{pr:polinomfv}. állítás utáni megjegyzést \apageref{pr:polinomfv}. oldalon.%
}
A jelölések megértése is fontos.
$p\left( t \right)\in\mathbf{F}\left[ t \right]$ semmi mást nem jelent, 
minthogy $p\left( t \right)$ egy polinom.
Persze a polinom nem keverendő össze a helyettesítési értékével, 
hiszen az egyik egy algebrai kifejezés-együttes, a másik egy az adott testbeli elem.
Szokásos viszont, hogy ha nincs konkrét $t$ a szövegkörnyezetben, akkor is $p\left( t \right)$ jelöli a polinomot. 
Néha egyszerűbben csak $p$-vel jelöljük, főleg akkor ha nincs szó behelyettesítésről,
emiatt érdektelen a változó jele.
Ritkábban, de előfordul, hogy egy konkrét értékre, mondjuk $s\in\mathbf{F}$-re kell kiértékelnünk a polinomot ilyenkor $p\left( s \right)$ jelöli azt a testbeli elemet,
amelyet $t$ helyett $s$-et téve az előírt műveletek kiértékelése után kapunk.
A szövegkörnyezetben mindig világosnak kell lennie, hogy $p\left( t \right)$ a polinomot jelenti,
vagy egy konkrét $t$-re kiértékelt testbeli elemet.

\begin{definition}[polinom foka]\index{polinom foka}
    Legyen $p\left( t \right)\in\mathbf{F}\left[ t \right]$ egy polinom.
    Azt mondjuk, hogy a $n$ nem negatív egész szám e \emph{polinom fokszáma},
    ha $n$ a legnagyobb nemzérus együttható indexe.
    A legnagyobb nemzérus együtthatót \emph{főegyütthatónak}\index{főegyüttható} nevezzük.
    Azt mondjuk, hogy egy nemzérus polinom \emph{normált}\index{normált polinom}, ha $1$ a főegyütthatója.

    A $p\left( t \right)=0$ konstans zérus polinom foka megállapodás szerint legyen $-\infty$.
    A $p$ polinom fokszámát $\deg p$ módon jelöljük.
\end{definition}
Látni fogjuk, hogy a konstans zérus polinomra $\deg p=-\infty$ csak egy kényelmes jelölés.
Időnként a polinom fokszámával műveleteket is végzünk.
Megegyezés szerint ilyenkor $-\infty+a=-\infty$ minden $a$ nem negatív egész számra, 
és $\left( -\infty \right)+\left( -\infty \right)=-\infty.$
A $-\infty$ szimbólumot minden egész számnál határozottan kisebbnek gondoljuk.

Két polinom összegét és szorzatát a szokásos módon definiáljuk:
\begin{definition}
    Legyen $p,q\in\mathbf{F}[t]$, két polinom.
    \[
        p\left( t \right)
        =
        \sum_{j=0}^n\alpha_jt^j
        \text{ és }
        q\left( t \right)
        =
        \sum_{j=0}^m\beta_jt^j,
        \qquad
        \alpha_j,\beta_j\in\mathbf{F}, 
        0\leq n,m\in\mathbf{Z}.
    \]
    Ekkor a $p$ és $q$ összegének definíciója:
    \[
        \left( p+q \right)\left( t \right)
        =
        \sum_{j=0}^{\max{\left\{ m,n \right\}}}\left( \alpha_j+\beta_j \right)t^j;
    \]
    míg a két polinom szorzatának definíciója:
    \[
        \left( pq \right)\left( t \right)
        =
        \sum_{j=0}^{n+m}c_jt^j
        \text{ ahol }
        c_j
        =
        \sum_{k=0}^j\alpha_k\beta_{j-k}.\qedhere
    \]
\end{definition}
\begin{proposition}
    Legyenek $p,q\in\mathbf{F}[t]$ polinomok az $\mathbf{F}$ test felett.
    Ekkor
    \begin{enumerate}
        \item $\deg \left( pq \right)=\deg p+\deg q$;
        \item $\deg \left( p+q \right)\leq\max\left\{ \deg p,\deg q \right\}$.\qedhere
    \end{enumerate}
\end{proposition}
\begin{proof}
    Figyeljünk arra, hogy a konstans zérus polinom esetében is működik a tétel,
    és vegyük észre, hogy az szorzat polinomra vonatkozó állítás azért igaz, 
    mert a test nullosztómentes.
\end{proof}
\begin{proposition}
    Egy $\mathbf{F}$ test feletti $\mathbf{F}\left[ t \right]$ formális polinomok 
    a fent bevezetett összeadás és szorzás műveletekkel,
    nullosztómentes,
    kommutatív, egységelemes gyűrűt alkotnak.
\end{proposition}

\section{Polinomok oszthatósága és a maradékos osztás}
\begin{definition}[oszthatóság]\index{polinom osztója}
        Azt mondjuk, hogy a $p\in\mathbb{F}\left[ t \right]$ \emph{osztója} az $f\in\mathbb{F}\left[ t \right]$ nem zérus polinomnak,
        ha létezik $h\in\mathbb{F}\left[ t \right]$, hogy $f\left( t \right)=p\left( t \right)h\left( t \right)$.
        Ilyenkor $f$-et a $p$ egy \emph{többszörösének}\index{polinom többszöröse} is mondjuk.
        Jelölés: $p|f$.
\end{definition}
Világos, hogy egy $p$ polinom összes többszörösei -- tehát azok, amelyeknek $p$ osztója --
ideált alkotnak. 
Ez a $p$ generálta legszűkebb ideál, azaz  a $J(p)=\left\{ fp:f\in\mathbb{F}\left[ t \right] \right\}$ főideál.
Ha $q\in J\left( p \right)$, akkor $J\left( q \right)\subseteq J\left( p \right)$, azaz ha $q$ egy többszöröse $p$-nek,
akkor $q$ minden többszöröse $p$-nek is többszöröse.
Ha $p,q$ polinomok, 
amelyekre $p|q$ és $q|p$ akkor a két polinom csak konstans szorzóban különbözik egymástól.
Ha például a két polinom még normált is, akkor $p|q$ és $q|p$ csak $p=q$ esetben lehetséges.
A polinomgyűrű ideáljaira fokuszálva, azt gondoltuk éppen meg, 
hogy \emph{$J\left( p \right)=J\left( q \right)$ normált $p,q$ polinomokra csak úgy teljesülhet, ha $p=q$},
azaz a polinomok gyűrűjében minden főideálnak csak egy generáló eleme van a normált polinomok körében.

A következő állítás szerint a polinomok közt is működik a maradékos osztás,
ahogyan azt az egész számok közt megszoktuk.
\begin{proposition}[maradékos osztás]
    Legyenek $p,q\in\mathbf{F}\left[ t \right]$ polinomok, $q\neq 0$.
    Ekkor létezik egyetlen $h,r\in\mathbf{F}\left[ t \right]$ polinom, amelyre
    \[
        p
        =
        hq+r;
        \quad
        \deg r < \deg q.\qedhere
    \]
\end{proposition}
\begin{proof}
    Először is azt vegyük észre, hogy $\deg p<\deg q$ esetben $r=p$, 
    $h=0$ szereposztással készen is vagyunk.

    Tegyük fel tehát, hogy $n=\deg p\geq \deg q=m$, és lássuk be az állítást $n$ szerinti indukcióval.
    Ha $n=0$, akkor $p\left( t \right)=\alpha_0$ és $q\left( t \right)=\beta_0\neq 0$.
    Ekkor persze
    \[
        \alpha_0=\frac{\alpha_0}{\beta_0}\beta_0+0,
    \]
    ami azt jelenti, hogy $h\left( t \right)=\frac{\alpha_0}{\beta_0}$ és $r\left( t \right)=0$ szereposztás
    megfelelő.

    Most tegyük fel, 
    hogy igaz az állítás $n+1$-nél kisebb fokú $p$ polinomokra ($n\geq 0$),
    és lássuk be egy pontosan $n+1$-ed fokú polinomra.
    Legyen tehát
    \[
        p\left( t \right)=\alpha_{n+1}t^{n+1}+\dots+\alpha_0
        \quad\text{ és }\quad
        q\left( t \right)=\beta_{m}t^m+\dots+\beta_0,
    \]
    ahol $m\leq n+1$.
    Tekintsük a következő polinomot:
    \[
        \frac{\alpha_{n+1}}{\beta_m}t^{n+1-m}q\left( t \right).
    \]
    Világos, hogy ennek főegyütthatója éppen $\alpha_{n+1}$ és foka éppen $n+1=\deg p$.
    Így a 
    \[
        p_1\left( t \right)
        =
        p\left( t \right)-
        \frac{\alpha_{n+1}}{\beta_m}t^{n+1-m}q\left( t \right).
    \]
    polinomra $\deg p_1<\deg p$.
    Na most, ha $\deg p_1<\deg q$, akkor a bizonyítás első mondatában említett helyzetben vagyunk,
    tehát nyilvánvaló szereposztással az állítás igaz $p_1$-re és $q$-ra.
    Ha viszont $\deg p_1\geq \deg q$ még mindig igaz, akkor az indukciós feltétel szerint található
    $h,r\in\mathbf{F}\left[ t \right]$ polinom, amelyre igaz az állítás.
    Mindkét esetben találtunk tehát $h,r$ polinomokat, amelyre
    \[
        p\left( t \right)-
        \frac{\alpha_{n+1}}{\beta_m}t^{n+1-m}q\left( t \right)
        =
        p_1\left( t \right)
        =
        h\left( t \right)q\left( t \right)+r\left( t \right);
        \quad
        \deg r < \deg q
    \]
    teljesül.
    Ekkor persze
    \[
        p\left( t \right)
        =
        \left( h\left( t \right)
        +
        \frac{\alpha_{n+1}}{\beta_m}t^{n+1-m}
        \right)
        q\left( t \right)
        +
        r\left( t \right);
        \quad
        \deg r < \deg q
    \]
    is fennáll. Ezt kellett belátni az állítás egzisztencia részéhez.
    
    Az unicitás részhez tegyük fel, hogy valamely $h,h_1,r,r_1$ polinomokra
    \[
        h\left( t \right)q\left( t \right)+r\left( t \right)
        =
        p\left( t \right)
        =
        h_1\left( t \right)q\left( t \right)+r_1\left( t \right)
    \]
    teljesül, ahol $\deg r<\deg q$ és $\deg r_1<\deg q$.
    Persze átrendezve ekkor
    \[
        \left( h\left( t \right)-h_1\left( t \right) \right)q\left( t \right)
        =
        r_1\left( t \right)-r\left( t \right)
    \]
    is fennáll.
    Ekkor a fokszámokra figyelve
    \[
        \deg\left( h-h_1 \right)+\deg q 
        = 
        \deg\left( r_1-r \right)
        \leq
        \max\left\{ \deg r_1,\deg (-r) \right\}
        <
        \deg q.
    \]
    Ez csak akkor lehetséges, ha $\deg\left( h-h_1 \right)=-\infty$,
    ami azt jelenti, hogy $h=h_1$, 
    amiből persze $r_1=r$ már látszik is.
\end{proof}
\begin{proposition}[a polinomgyűrű egy főideál-gyűrű]
    A polinomok $\mathbf{F}\left[ t \right]$ kommutatív, egységelemes, nullosztómentes gyűrűjében
    minden a $\left\{ 0 \right\}$-tól különböző ideált generál az ideálban lévő egyetlen normált minimális fokszámú polinom.
    Így $\mathbf{F}\left[ t \right]$ egy főideál-gyűrű.
\end{proposition}
\begin{proof}
    Legyen a $J$ egy ideálja $\mathbf{F}\left[ t \right]$-nek, 
    amely nem csak a zérus elemből áll.
    Vegyünk egy minimális fokszámú de nem zérus polinomot $J$-ben,
    tehát olyat, 
    amely maga sem zérus és nála kisebb fokszámú polinom már nincs $J$-ben a $0$ elemen kívül.
    Legyen ez $d$. 
    Most megmutatjuk, hogy minden $p\in J$-re $d|p$.
    A maradékos osztás szerint
    valamely $h,r$ polinomokra
    \[
        p\left( t \right)=
        h\left( t \right)d\left( t \right)+r\left( t \right);
        \text{ ahol }
        \deg r<\deg d.
    \]
    Mivel $p,d\in J$, és $J$ egy ideál, ezért $r\in J$.
    No de $d$ konstrukciója szerint ilyen csak a zérus polinom van,
    ezért valóban $d|p$.
    Ez éppen azt jelenti, hogy
    \(
        J=\left\{ dh:h\in\mathbf{F}\left[ t \right] \right\}
    \)
    azaz $d$ generálja a $J$ ideált.
    Azt viszont már korábban is meggondoltuk, 
    hogy egy főideált csak egyetlen normált polinom generál.

    Megmutattuk tehát, 
    hogy egyetlen normált, minimális fokszámú polinom van $J$-ben,
    és minden $J$-beli polinom ennek többszöröse.
\end{proof}
Érdemes eltenni magunknak, hogy az ideál generáló eleme, tehát az ideálbeli elemek közös osztója
éppen az ideál minimális fokszámú nem zérus polinomja.
Ilyenből a normált polinomok közül csak egy van.

\begin{definition}
        Legyenek most $p_1,\dots,p_k$ polinomok. 
        \begin{enumerate}
            \item A $d$ polinom a \emph{legnagyobb közös osztója}\index{legnagyobb közös osztó} az adott polinomoknak, 
                ha
                \begin{enumerate}
                    \item $d|p_j$ minden $j=1,\dots,k$-ra,
                    \item ha $d_1|p_j$ minden $j=1,\dots,k$ mellett akkor $d_1|d$ is fennáll,
                    \item $d$ normált.
                \end{enumerate}
                A $p_1,\dots,p_k$ polinomokat \emph{relatív prímeknek}\index{relatív prím polinomok} nevezzük, 
                ha közös osztójuk csak a konstans polinomok, 
                azaz a $d\left( t \right)=1$ a legnagyobb közös osztó.
            \item A $d$ polinom a \emph{legkisebb közös többszöröse}\index{legkisebb közös többszörös} az adott polinomoknak, 
                ha
                \begin{enumerate}
                    \item $p_j|d$ minden $j=1,\dots,k$-ra,
                    \item ha $p_j|d_1$ minden $j=1,\dots,k$ mellett akkor $d|d_1$ is fennáll.
                    \item $d$ normált.\qedhere
                \end{enumerate}
        \end{enumerate}
\end{definition}
Persze az első kérdés, hogy van-e a polinomoknak legnagyobb közös osztója vagy legkisebb közös többszöröse, és hány ilyen van?
\begin{proposition}
    Bármely $p_1,\dots,p_r\in\mathbb{F}\left[ t \right]$ nem zérus polinomoknak
    létezik egyetlen legkisebb közös többszörösük.
\end{proposition}
\begin{proof}
    Világos, hogy ideálok metszete is ideál, emiatt 
    \(
        \cap_{j=1}^rJ\left( p_j \right)
    \)
    is ideál $\mathbf{F}\left[ t \right]$ gyűrűben.
    De itt minden ideál főideál, létezik tehát $d\in\mathbf{F}\left[ t \right]$ normált polinom, amelyre
    \[
        J\left( d \right)
        =
        \cap_{j=1}^rJ\left( p_j \right)
    \]
    Világos, hogy $d\in J\left( p_j \right)$ minden $j$-re, 
    ergo $d$ többszöröse minden $p_j$-nek.
    Ha $p_j|d_1$ fennáll, minden $j$-re
    az azt jelenti, hogy $d_1\in J\left( p_j \right)$, minden $j$-re, azaz 
    \(
        d_1
        \in
        \cap_{j=1}^rJ\left( p_j \right)
        =
        J\left( d \right),
    \)
    tehát $d|d_1$ valóban fennáll.

    Ha $d$ mellett $g$ is legkisebb közös többszörös, akkor $d|g$ és $g|d$ szerint $g$ és $d$ foka azonos,
    így csak egymás konstans szorosai lehetnek, de mivel mindketten normáltak, ezért e konstans csak 1 lehet.
\end{proof}
\begin{proposition}
    Bármely $p_1,\dots,p_r\in\mathbb{F}\left[ t \right]$ nem zérus polinomoknak
    létezik egyetlen legnagyobb közös osztójuk.
    A $d$ legnagyobb közös osztó kifejezhető
    \[
        d\left( t \right)=f_1\left( t \right)p_1\left( t \right)+
        \dots+
        f_r\left( t \right)p_r\left( t \right)
    \]
    alakban valamely $f_1,\dots,f_r\in\mathbb{F}\left[ t \right]$ polinomok segítségével.
\end{proposition}
\begin{proof}
    Láttuk, hogy létezik egyetlen normált $d$ polinom, amelyre $J\left( d \right)=J\left( p_1,\dots,p_r \right)$.
    Világos, hogy $d|p_j$ minden $j=1,\dots,r$ és $d\in J\left( p_1,\dots,p_r \right)$,
    azaz
    \[
        d=f_1p_1+\dots+f_rp_r
    \]
    valamely $f_1,\dots,f_r$ polinomokra.
    Ha valamely $d_1$ polinomra $d_1|p_j$ minden $j=1,\dots,r$ mellett,
    akkor a fenti azonosság szerint $d_1|d$ is fennáll.

    Az egyértelműség mint a legkisebb közös többszörösnél.
\end{proof}
A következő állítás azonosságát Bezout--azonosságnak mondjuk.
\begin{proposition}[Bezout--azonosság]\index{Bezout--azonosság}
    Legyenek a $p_1,\dots,p_r\in\mathbb{F}\left[ t \right]$ tetszőleges $\mathbb{F}$ test feletti polinomok.
    Ezek pontosan akkor relatív prímek, 
    ha léteznek $f_1,\dots,f_r\in\mathbb{F}\left[ t \right]$ polinomok, hogy
    \[
        f_1\left( t \right)p_1\left( t \right)+f_2\left( t \right)p_2\left( t \right)+\dots+f_r\left( t \right)p_r\left( t \right)=1\qedhere
    \]
\end{proposition}

A szakaszt a maradékos osztás módszerének másik fontos következményeivel zárjuk.
Azt gondoljuk meg, 
hogy a gyöktényező a polinomból mindig kiemelhető, 
emiatt egy akármilyen test feletti $n$-ed fokú polinom gyökeinek száma $n$-nél nagyobb nem lehet.
\begin{proposition}
    Legyen $p\in\mathbf{F}\left[ t \right]$ egy nem zérus polinom,
    és $t_0$ egy gyöke, azaz $p\left( t_0 \right)=0$.
    Ekkor létezik $h\in\mathbf{F}\left[ t \right]$ polinom,
    amelyre
    \[
        p\left( t \right)=\left( t-t_0 \right)h\left( t \right).\qedhere
    \]
\end{proposition}
\begin{proof}
    Maradékos osztással $p$-re és az elsőfokú $t-t_0$ polinomra
    \[
        p\left( t \right)=h\left( t \right)\left( t-t_0 \right)+r\left( t \right),
        \text{ ahol }
        \deg r<1.
    \]
    No de, $t_0$ egy gyök, tehát $0=p\left( t_0 \right)=r\left( t_0 \right)$. 
    Ez azt jelenti, hogy $\deg r=-\infty$, ami éppen az állítás.
\end{proof}
\begin{definition}[gyök multiplicitása]\index{gyökök multiplicitása}
        Legyen $t_0$ gyöke a $p\left( t \right)$ polinomnak.
        Azt mondjuk, hogy a $k$ pozitív egész e $t_0$ gyök \emph{multiplicitása},
        ha van olyan $h\left( t \right)$ polinom, hogy 
        \begin{math}
            p\left( t \right)=\left( t-t_0 \right)^kh\left( t \right),
        \end{math}
        de $h\left( t_0 \right)\neq 0$.
        Néha azt is mondjuk, hogy $t_0$ egy $k$-szoros gyöke $p$-nek.
\end{definition}
Teljesen világos, hogy a gyöktényező kiemelhetősége miatt minden gyök legalább egyszeres multiplicitású.
A következő állítás szerint 
a gyökök száma még a multiplicitásukkal együtt számolva sem lehet több mint a polinom foka.
\begin{proposition}
    Legyen(ek) a $p\in\mathbf{F}\left[ t \right]$ nem zérus polinom különböző gyökei $t_1,\dots,t_k$,
    és ezen gyökök multiplicitásai rendre $m_1,\dots,m_k$.
    Ekkor $m_1+\dots+m_k\leq\deg p$.
\end{proposition}
\begin{proof}
    A test nullosztó mentessége és a gyöktényező kiemelhetősége miatt 
    \[
    p\left( t \right)=
    \left( t-t_1 \right)^{m_1}\cdot\left( t-t_2 \right)^{m_2}\dots\left( t-t_k \right)^{m_k}\cdot 
    h\left( t \right),
    \]
    ahol $h$ olyan polinom, amelynek már nincsen gyöke.
    A fokszámok összehasonlításából kapjuk, hogy
    $m_1+\dots+m_k\leq m_1+\dots+m_k+\deg h=\deg p$.
\end{proof}
A fenti gondolat szerint, ha egy legfeljebb $n$-ed fokú polinomnak $n+1$ különböző gyöke van, 
akkor csak úgy lehetséges, ha a polinom minden együtthatója nulla.
Ezt úgy is szoktuk fogalmazni, 
hogy egy legfeljebb $n$-ed fokú polinomot $n+1$ helyettesítési értéke már egyértelműen meghatározza:
\begin{proposition}\label{pr:polinomfv}
    Tegyük fel, hogy a $p,q\in\mathbb{F}\left[ t \right]$ polinomok legfeljebb $n$-ed fokúak, ahol $n$ egy nemnegatív egész,
    és tegyük fel, 
    hogy létezik $n+1$ különböző $t_0,t_1,\dots,t_n$ pont a testben,
    amelyekre $p\left( t_j \right)=q\left( t_j \right)$ minden $j=0,\dots,n$.
    Ekkor $p\left( t \right)=q\left( t \right)$, 
    azaz $p$ és $q$ együtthatói azonosak.
\end{proposition}
\begin{proof}
    Legyen $h=p-q$. 
    Világos, hogy $\deg h\leq n$ és $h$-nak van $n+1$ különböző gyöke.
    Az előző állítás szerint ez csak a $h=0$ polinomra igaz, ami azt jelenti, 
    hogy $p$ és $q$ együtthatói azonosak.
\end{proof}
A maradékos osztás tételének szép következménye tehát,
hogy ha $\mathbb{F}$ egy nem véges test, és a $p,q\in\mathbb{F}\left[ t \right]$ polinomok,
akkor $p$-nek és $q$-nak pontosan akkor azonosak az együtthatói, ha 
\(
    p\left( t \right)=q\left( t \right)
\)
fennáll minden $t\in\mathbb{F}$ mellett.

Itt fontos, hogy $\mathbb{F}$ nem véges test,
hiszen például ha $\mathbb{F}=\left\{ 0,1 \right\}$ a két elemű test,
akkor a $p\left( t \right)=t^2+t$ polinomra minden $t\in\left\{ 0,1 \right\}$ mellett $p\left( t \right)=0$,
de a polinom együtthatói rendre a $\left\{ 0,1,1 \right\}$ számok a testből, 
tehát ez nem a összeadásra nézve neutrális eleme az $\mathbb{F}\left[ t \right]$ polinomgyűrűnek.

Konklúzióképpen: megnyugodhatunk, hogy az iménti szörnyűség nem véges testek esetén nem fordulhat elő,
tehát mondjuk a valós vagy a komplex számtest felett mindegy, 
hogy a polinomokat függvényeknek, vagy formális algebrai kifejezéseknek gondoljuk. 
A lényeg hogy egy $n$-ed fokú polinomot az $n+1$ együtthatója, definíció szerint, 
de az $n+1$ különböző helyen felvett helyettesítési értéke is egyértelműen meghatározza.

\section{Az Euklideszi--algoritmus}
Algoritmust keresünk polinomok legnagyobb közös osztójának és legkisebb közös többszörösének meghatározására.
Ha egy pillanatra $\left( p_1,\dots,p_n \right)$ jelöli az adott $p_1,\dots,p_n$ polinomok legnagyobb közös osztóját,
akkor nem nehéz meggondolni, 
hogy 
\[
    \left( \left( p_1,\dots,p_{n-1} \right),p_n \right)=\left( p_1,\dots,p_{n}\right).
\]
Ezt $n=3,4,\dots$ számokra alkalmazva azt kapjuk, 
hogy ha módszerünk van két polinom legnagyobb közös osztójának meghatározására, 
akkor evvel már akárhány -- persze véges sok -- polinom legnagyobb közös osztója is meghatározható.
Analóg módon ugyanez igaz a legkisebb közös többszörösre is.
Azt gondoltuk meg tehát, hogy ha meg tudnánk határozni két polinom legnagyobb közös többszörösét és legkisebb közös osztóját,
akkor ugyan ezt mát meg tudnánk tenni véges sok polinom esetén is.

Az Euklideszi-algoritmus két polinom legnagyobb közös osztójának meghatározására szolgál.
Az eddigi ismereteink szerint a $p,q$ legnagyobb közös osztója a $J\left( p,q \right)$ ideál legalacsonyabb 
fokú, normált tagja. 
Véges sok lépésben végrehajtható, ezért a fentinél sokkal egyszerűbben működő algoritmust ad.
Emlékezzünk arra, hogy ha $p,q\in\mathbf{F}\left[ t \right]$ valamely polinomok, akkor
    \(
        J\left( p,q \right)=\left\{ fp+gq:f,g\in\mathbf{F}\left[ t \right] \right\}
    \)
jelöli a $p$ és a $q$ polinomokat tartalmazó legszűkebb ideált.
Világos, hogy ha $r_1,r_2\in J\left( p,q \right)$, akkor 
$J\left( r_1,r_2 \right)\subseteq J\left( p,q \right)$.
\begin{proposition}[Euklidesz]\index{Euklideszi algoritmus}
    Legyen $p,q\in\mathbb{F}\left[ t \right]$ nem zérus polinomok.
    Definiálja $p_{-1}=p,p_{0}=q$. 
    Folytatva, ha valamely $i\geq 0$ számra $p_{i-1}$ és $p_i$ már definiált és $p_i\neq 0$, 
    akkor a maradékos osztás szerint létezik egyetlen $h_{i+1},p_{i+1}\in\mathbb{F}\left[ t \right]$ polinom,
    amelyre 
    \[
        p_{i-1}=h_{i+1}p_i+p_{i+1};
        \text{ ahol }
        \deg p_{i+1}<\deg p_i.\tag{\dag}
    \]
    Mivel minden egyes lépésben csökken a fokszám,
    ezért van olyan $s>0$, hogy $p_s\neq 0$, de $p_{s+1}=0$.
    Erre a $p_s$ polinomra $p_s\in J\left( p,q \right)$ és
    $p_s$ normáltja a $p\text{ és a }q$ polinomok legnagyobb közös osztója.
\end{proposition}
\begin{proof}
    A fenti algoritmussal olyan
    $p_{-1}, p_0,p_1,\dots,p_s,p_{s+1}$ polinomokat kapunk, 
    amelyekre minden $i=0,\dots,s$ mellett a (\dag) azonosság fennáll, 
    és $\deg p_{s+1}=-\infty$, 
    azaz $p_{s+1}=0$.

    A (\dag) azonosság szerint, 
    ha egy $h$ polinomra $h|p_{i+1},h|p_{i}$ akkor $h|p_{i-1}$ is fennáll.
    Az utolsó azonosságra $p_{s+1}=0$, 
    így $p_s|p_{s-1}$.
    No de $p_s|p_s$, amiből már $p_s|p_{s-2}$ is fennáll.
    Ezt a gondolatot folytatva látjuk, hogy $p_s|p_0$ és $p_s|p_{-1}$,
    ami azt jelenti, hogy a $p_s$ polinom a $p$ és $q$ polinomok egy közös osztója.

    Most a (\dag) azonosságot értelmezzük úgy, 
    \(
    p_{i+1}\in J\left( p_{i-1},p_i \right)\text{ ahol } i=0,\dots,s.
    \)
    Ebből persze $p_i,p_{i+1}\in J\left( p_{i-1},p_i \right)$, 
    ergo $J\left( p_i,p_{i+1} \right)\subseteq J\left( p_{i-1},p_i \right)$.
    Így minden $i=0,\dots,s$-re
    \[
        p_i\in J\left( p_{i-2},p_{i-1} \right)
        \subseteq
        J\left( p_{i-3},p_{i-2} \right)
        \subseteq
        \dots
        \subseteq
        J\left( p_0,p_1 \right)
        \subseteq
        J\left( p_{-1},p_0 \right)
        =
        J\left( p,q \right).
    \]
    Speciálisan $i=s$-re is, tehát $p_s\in J\left( p,q \right)$,
    ergo léteznek $f,g\in\mathbf{F}\left[ t \right]$ polinomok, 
    amelyekre 
    \[
        p_s=fp+gq.
    \]
    Ha $d|p$ és $d|q$, akkor persze $d|p_s$ is teljesül.
    Emiatt,
    a $p_s$ polinomot a p főegyütthatójával elosztva kapjuk a $p$ és a $q$ polinomok legnagyobb közös osztóját.
\end{proof}
Most a legkisebb közös többszörös algoritmikus meghatározására törekszünk.
\begin{proposition}\label{pr:rprim}
    Legyenek a $p,q\in\mathbb{F}\left[ t \right]$ polinomok relatív prímek,
    és tegyük fel, hogy 
    \begin{math}
        p|qr.
    \end{math}
    Ekkor $p|r$.
\end{proposition}
\begin{proof}
    Mivel a $p$ és a $q$ polinomok relatív prímek,
    ezért a Bezout--azonosság szerint van $f$ és $g$ polinom, amelyekre
    \(
        fpr+gqr=r.
    \)
    A feltétel szerint $qr$ a $p$ többszöröse, 
    így az iménti azonosság baloldala a $p$ többszöröse,
    ami éppen azt jelenti, hogy $p|r$.
\end{proof}
\begin{proposition}
    Tekintsük a $p,q\in\mathbb{F}\left[ t \right]$ polinomokat.
    Jelölje $d$ a legnagyobb közös osztót, 
    és
    $m$ a legkisebb közös többszöröst.
    Ekkor
    \begin{displaymath}
        d\left( t \right)m\left( t \right)=p\left( t \right)q\left( t \right).\qedhere
    \end{displaymath}
\end{proposition}
\begin{proof}
    Legyen $p=dr_1$ és $q=dr_2$.
    Először megmutatjuk, hogy $r_1$ és $r_2$ relatív prímek.
    Ha $s$ polinomom közös osztójuk, akkor $ds$ is közös osztója $p$-nek és $q$-nak,
    amiből $ds|s$ következik.
    A fokszámokat összehasonlítva ez csak akkor lehetséges, ha $s$ konstans polinomom, 
    azaz $s|1$ valóban fennáll.

    Most megmutatjuk, hogy az $m$ legkisebb közös többszörösre
    \[
        m=dr_1r_2.\tag{\dag}
    \]
    Világos, hogy $dr_1r_2$ egy közös többszöröse a $p,q$ polinomoknak.
    Tegyük fel, hogy $s$ egy másik közös többszörös, azaz 
    $s=ps_1$ és $s=qs_2$.
    Ekkor 
    \begin{math}
        dr_1s_1=ps_1=s=qs_2=dr_2s_2,
    \end{math}
    amiből a nullosztómentesség szerint 
    \[
        r_1s_1=r_2s_2.
    \]
    Na most,
    a fent kiemelt azonosság szerint szerint $r_1|r_2s_2$, ahol $r_1$ és $r_2$ relatív prímek.
    Ebből azonnal kapjuk, hogy $r_1|s_2$.
    No de $s=qs_2=dr_1s_2$, amiből már látszik, 
    hogy $s$ egy többszöröse a $dr_1r_2$ polinomnak.
    Ez éppen (\dag) azonosságot jelenti.
    Innen
    $d\cdot m=d(dr_1r_2)=(dr_1)(dr_2)=p\cdot q$ már nyilvánvaló.
\end{proof}
A fenti állítás csak két polinomra igaz, többre nem,
de nekünk csak két polinomra kell.
Úgy interpretáljuk, 
hogy ha a szorzatot osztom maradékosan a legnagyobb közös osztóval, akkor a maradék mindig 
zérus, 
és a hányados éppen a legkisebb közös többszörös.
Azt gondoltuk meg tehát, 
hogy az Euklideszi-algoritmus módszert ad két polinom legkisebb közös többszörösének algoritmikus meghatározására is.

Visszatérve a szakasz elején felvetett gondolatra, 
ilyen módon véges sok lépésben végrehajtható algoritmust kapunk véges sok polinom legkisebb közös többszörösének meghatározására.
Például öt polinom legkisebb közös többszöröséhez, 
egy Euklideszi-algoritmussal meghatározzuk az első kettő polinom legnagyobb közös osztóját, 
majd egy újabb maradékos osztással az első kettő legkisebb közös többszörösét.
Ugyanezt teszem az így kapott és a harmadik polinommal, 
az eredmény az első három polinom legkisebb közös többszöröse.
Az így kapott polinommal és a negyedik polinommal egy újabb Euklideszi-algoritmus és egy újabb maradékos osztás után kapjuk az első négy polinom legkisebb közös többszörösét, 
majd ennek eredményével és az ötödik polinommal mint két polinomnak a legkisebb közös többszörösével kapjuk az eredeti öt polinom legkisebb közös többszörösét.
\section{Polinom faktorizáció}
Kicsi korunk óta sulykolják belénk, hogy minden egész szám előáll, méghozzá lényegében csak egyféleképpen
prímek szorzataként.
Ha ismerjük két szám prímtényezős előállítását, akkor nagyon könnyű megmondani a két szám legkisebb közös többszörösét,
vagy a legnagyobb közös osztóját.
Evvel a probléma csak annyi, 
hogy nagyon nehéz megmondani két, esetleg jó nagy, szám prímtényezős előállítását,
így még az egész számok gyűrűjében is az Euklideszi-algoritmus a megfelelő, véges sok lépésben,
végrehajtható módszer a legnagyobb közös osztó és a legkisebb közös többszörös konkrét felírására.

Ebben a szakaszban azt mutatjuk meg, hogy prímtényezős előállításról szóló tétel a polinomok gyűrűjében is igaz marad.
Természetesen konkrét algoritmust nem adunk, hiszen ilyen még számokra sem igen van.%
\footnote{Ha ilyen lenne senki nem tudna interneten két számla közt pénzt mozgatni. Lásd például: \url{https://en.wikipedia.org/wiki/RSA_(cryptosystem)}}
\begin{definition}[reducibilis polinom]\index{reducibilis polinom}\index{irreducibilis polinom}
    Egy polinomot \emph{reducibilisnek} mondunk, 
    ha előáll mint két legalább elsőfokú polinom szorzata.
    Egy nem reducibilis polinomot \emph{irreducibilisnek} nevezünk.
\end{definition}
Világos, hogy minden legfeljebb elsőfokú polinom tetszőleges test felett irreducibilis.
A magasabb fokú polinomok esetében a probléma nagyban függ a testtől is, 
ahonnan a polinom együtthatói származnak.
A következő állítás viszont minden test mellett igaz.
\begin{proposition}[polinom faktorizáció]\label{pr:polfact}
    Tetszőleges test feletti polinomgyűrűben, 
    minden (normált) polinom előáll mint (normált) irreducibilis polinomok szorzata.
\end{proposition}
\begin{proof}
    Az előállítandó polinom foka szerinti teljes indukció.
    Elsőfokú polinom maga irreducibilis.

    Most tegyük fel, hogy az állítás igaz $n$-nél alacsonyabb fokú polinomokra
    és lássuk be $\deg p=n$ mellett, ahol $n>1$.
    Ha $p$ irreducibilis, akkor megint készen vagyunk.
    Ha $p=f g$ valamely $\deg f\geq 1$ és $\deg g\geq 1$ mellett,
    akkor $\deg f<n$ és $\deg g<n$.
    Az indukciós feltevés szerint $f$ és $g$, emiatt $p=fg$ is előáll irreducibilis polinomok szorzataként.
\end{proof}

Érdemes látni, hogy ha $p$ irreducibilis, 
és $f$ egy tetszőleges polinom, akkor vagy $p|f$ vagy $p$ és $f$ relatív prímek.
Ugyanis ha $g$ egy közös osztójuk, akkor $p$ iredducibilitása miatt,
$\deg g=0$, vagy $\deg g=\deg p$. 
Ez utóbbi esetben $g$ a $p$ konstans szorosa, ergo $p|f$,
az előbbi eset pedig éppen azt jelenti, hogy $p$ és $f$ relatív prímek.
\begin{proposition}[prím tulajdonság]
    Legyen $p\in\mathbf{F}\left[ t \right]$ irreducibilis polinom, amelyre
    $p|(f_1\cdots f_n)$, valamely $f_j\in\mathbf{F}\left[ t \right]$ polinomokra, ahol $j=1\dots,n\geq 1$.
    Ekkor létezik $1\leq j\leq n$, amelyre $p|f_j$.\qedhere
\end{proposition}
\begin{proof}
    A polinomok $n$ száma szerinti indukció.
    Az $n=1$ eset semmitmondó módon teljesül.
    Tegyük fel, hogy igaz az állítás $n$-nél kevesebb polinomra,
    és lássuk be $n$-re. Itt $n\geq 2$.
    Induljunk ki tehát abból, hogy 
    \[
        p|\left( f_1\cdots f_{n-1} \right)\cdot f_n
    \]
    Ha $p$ osztója lenne az $f_1\cdots f_{n-1}$ szorzatnak, 
    akkor az indukciós feltevés szerint készen is lennénk.
    Ha $p$ nem osztója a szorzatnak, 
    akkor $p$ irreducibilis volta miatt relatív prímek.
    Ekkor \aref{pr:rprim}. állítás szerint $p|f_n$.
\end{proof}
Az állítás fordítva is igaz, de azt a gyakorlatokra hagyjuk.
Az irreducibilis polinomok prím tulajdonsága segítségével a polinomok faktorizáció egyértelműségét is igazolhatjuk.
\begin{proposition}
    Legyen $p \in\mathbf{F}\left[ t \right]$ egy legalább elsőfokú normált polinom.
    Ekkor ezek sorrendjétől eltekintve egyértelműen léteznek legalább elsőfokú, normált, irreduciblis $q_1,\dots,q_s\in\mathbf{F}\left[ t \right]$ polinomok,
    hogy $p=q_1\cdots q_s$.
\end{proposition}
A ,,sorrendtől eltekintés'' alatt azt értjük, hogy ha $p$ előáll 
\[
    p_1\cdots p_s=p=q_1\cdots q_r
\]
legalább elsőfokú, normált, irreducibilis polinomok szorzataként,
akkor $s=r$ és a $p_1,\dots,p_s$ polinomok alkalmas átindexelése után $p_j=q_j$,
minden $j=1,\dots,s$ mellett.
\begin{proof}
    \Aref{pr:polfact}. állításban már tisztáztuk az egzisztenciális részt, 
    így már csak az unicitás maradt,
    amit a felbontandó polinom fokszáma szerinti indukcióval végzünk most is.
    Ha a polinom elsőfokú, akkor az unicitás is nyilvánvaló.

    Tegyük fel, hogy igaz az egyértelműség $n$-nél alacsonyabb fokszámú polinomokra,
    és tegyük fel, hogy $\deg p=n>1$.
    Nézzünk két lehetséges előállítást
    \[
    p_1\cdots p_s=p=q_1\cdots q_r,
    \]
    ahol $p_1,\dots,p_s,q_1,\dots,q_r$ legalább elsőfokú, normált, irreducibilis polinomok.
    A $q_1$ polinom osztója a jobboldalnak, ezért a baloldalnak is.
    A prím tulajdonság miatt, \ref{pr:rprim}. állítás, $q_1$ osztója az egyik baloldali polinomnak.
    Alkalmas átindexelés után feltehető, hogy $q_1|p_1$.
    No de, $p_1$ is irreducibilis, és $\deg q_1\geq 1$ miatt csak $\deg q_1=\deg p_1$ lehetséges,
    tehát a normáltság szerint $q_1=p_1$.
A polinomgyűrű nullosztó mentessége szerint az első polinomokkal egyszerűsíthetünk, ergo
    \[
    p_2\cdots p_s=p=q_2\cdots q_r,
    \]
    is fennáll. 
    A fenti polinom már $n$-nél alacsonyabb fokú, 
    így az indukciós feltevés szerint $s-1=r-1$,
    és alkalmas átindexelés után minden $j=2,\dots,s$ esetén is teljesül a $p_j=q_j$ egyenlőség.
\end{proof}
Az egész szakaszt összefoglalhatjuk így is:
\begin{proposition}
    Minden legalább elsőfokú normált polinomhoz léteznek,
    méghozzá sorrendjüktől eltekintve egyértelműen léteznek 
    $q_1,\dots,q_s$ normált, irreducibilis polinomok,
    és $n_1,\dots,n_s$ pozitív egészek, 
    amelyekre
    \[
        p\left( t \right)
        =
        q_1^{n_1}\left( t \right)\cdots q_s^{n_s}\left( t \right).\qedhere
    \]
\end{proposition}

\section{Mátrixok}
\begin{definition}[mátrix]
    Egy tetszőleges test feletti mátrixnak nevezzünk,
    a test elemeiből képzett táblázatot.
    Ha $m,n\in\mathbf{N}$ előre rögzített pozitív egészek és 
    az $A$ táblázatnak $m$ sora és $n$ oszlopa van, 
    akkor azt mondjuk, hogy $A$ egy $m\times n$ méretű mátrix.
    Az $\mathbf{F}$ test feletti $m\times n$-es mátrixok halmazát $\mathbf{F}^{m\times n}$
    módon jelöljük.

    Ha $A\in\mathbf{F}^{m\times n}$ egy mátrix, 
    akkor $A_i$ jelöli az $i$-edik sort, ami persze egy $1\times n$-es mátrix;
    $A^j$ jelöli a $j$-edik oszlopot, ami persze egy $m\times 1$-s mátrix;
    $A_i^j$ jelöli az $i$-edik sor $j$-edik elemét.
    Sokszor használjuk az $A_{i,j}=A_i^j$ jelölést is.

    Időnként, azt hangsúlyozandó hogy mátrixokról van szó a mátrixot jelölő betűt kapcsos zárójelbe teszem.
    Pl. $\left[ A \right]\in\mathbf{F}^{m\times n}$.

    A mátrixot a mérete és az elemei határozzák meg.
    Emiatt két mátrix akkor azonos, ha azonos méretűek, és a megfelelő elemeik is azonosak.

    \emph{Diádnak}\index{diád} nevezzük egy oszlop és egy sor szorzatát.
    Ha az oszlopnak és a sornak rendre azonosak az elemei, akkor \emph{szimmetrikus diádról} beszélünk.\index{szimmetrikus diád}
\end{definition}
Az azonos típusú mátrixok közt műveleteket definiálunk:
\begin{definition}[mátrixok összege]\index{mátrixok összege}
    Rögzített $m,n\in\mathbf{N}$ mellett, ha $A,B\in\mathbf{F}^{m\times n}$,
    akkor ezek összege az a $C\in\mathbf{F}^{m\times n}$ mátrix, amelyre
    \[
        C_{i,j}=A_{i,j}+B_{i,j}
    \]
    minden $i=1,\dots,m$ és $j=1,\dots, n$.
    Jelölés: $C=A+B$.
\end{definition}
\begin{proposition}\label{pr:matrixokVS1}
    Az $m\times n$ méretű mátrixok az fent definiált összeadás művelettel
    Abel-csoportot alkotnak.
    A $[0]$-val jelölt neutrális elem az az $m\times n$-s mátrix, 
    amelynek minden eleme a test zérus eleme:
    \[
        [0]_{i,j}=0;
    \]
    az $[A]$ mátrix additív inverze az az $[-A]$-val jelölt $m\times n$ méretű mátrix,
    amelyre
    \[
        [-A]_{i,j}=-([A]_{i,j}).\qedhere
    \]
\end{proposition}

Most definiáljuk egy számnak és egy mátrixnak a szorzatát.
\begin{definition}[szám és mátrix szorzata]\index{szám és mátrix szorzata}
    Ha $\alpha\in\mathbf{F}$ egy szám és $A\in\mathbf{F}^{m\times n}$ egy mátrix, akkor ezek szorzata
    az $\alpha A\in\mathbf{F}^{m\times n}$ módon jelölt mátrix, melynek elemeire
    \[
        [\alpha A]_{i,j}=\alpha[A]_{i,j}.\qedhere
    \]
\end{definition}
Könnyen ellenőrizhetőek a következő számolási szabályok:
\begin{proposition}\label{pr:matrixokVS2}
    Legyenek $A,B\in\mathbf{F}^{m\times n}$ mátrixok, és $\alpha,\beta\in\mathbf{F}$ tetszőleges számok.
    Ekkor
    \begin{enumerate}
        \item $\alpha\left( A+B \right)=\alpha A+\alpha B$;
        \item $\left( \alpha+\beta \right)A=\alpha A+\beta A$;
        \item $\left( \alpha\beta \right)A=\alpha\left( \beta A \right)$;
        \item $1 A=A$.
    \end{enumerate}
\end{proposition}
Az utolsó két állítást, \ref{pr:matrixokVS1} és \ref{pr:matrixokVS2},
együtt később úgy fogjuk fogalmazni, 
hogy adott test feletti tetszőleges méretű mátrixok \emph{vektorteret}\index{vektortér} alkotnak.

Most mátrixok szorzatát definiáljuk:
\begin{definition}[mátrixok szorzata]\index{mátrixok szorzata}
    Legyen $A\in\mathbf{F}^{m\times k}$ és $B\in\mathbf{F}^{k\times n}$ mátrix.
    Fontos, hogy $A$ oszlopainak száma azonos $B$ sorainak számával.
    Ezek $C=AB$ szorzata egy $C\in\mathbf{F}^{m\times n}$ mátrix,
    melynek elemeit az alábbi egyenlőség definiálja
    \[
        [C]_{i,j}=\sum_{s=1}^k[A]_{i,s}[B]_{s,j}.
    \]
    Itt persze $i=1,\dots,m$ és $j=1,\dots,n.$
\end{definition}
Az összeadás és szorzás műveleteket a disztributivitás kapcsolja össze:
\begin{proposition}
    Legyen $A\in\mathbf{F}^{m\times k}$, valamint a $B,C\in\mathbf{F}^{k\times n}$ mátrixok.
    Ekkor
    \begin{displaymath}
        A\left( B+C \right)=AB+AC.
    \end{displaymath}
    Hasonlóan,
    ha $A,B\in\mathbf{F}^{m\times k}$, valamint a $C\in\mathbf{F}^{k\times n}$ mátrixok,
    akkor
    \begin{displaymath}
        \left( A+B \right)C=AC+BC.
    \end{displaymath}
\end{proposition}
\begin{proposition}
    Legyen $A\in\mathbf{F}^{m\times k}$ és $B\in\mathbf{F}^{k\times n}$ mátrix.
    Jelölje $C=AB$ ezek szorzatát ebben a sorrendben.
    Ekkor
    \begin{enumerate}
        \item A szorzat mátrix $i$-edik sorának $j$-edik eleme Az $A$ mátrix $i$-edik sorának és a $B$ mátrix $j$-edik
            oszlopának szorzata. Magyarul: minden $1\leq i\leq m$ és $1\leq j \leq n$ mellett
            \[
                [C]_{i,j}=[A]_i\cdot [B]^j.
            \]
        \item
            A szorzat mátrix minden oszlopa az $A$ mátrix oszlopainak a $B$ mátrix megfelelő oszlopából vett elemekkel képzett
            lineáris kombinációja.
            Magyarul: minden $1\leq j\leq n$ mellett
            \[
                [C]^j=\sum_{s=1}^k[B]_s^j[A]^s.
            \]
        \item
            A szorzat mátrix minden sora a $B$ mátrix sorainak az $A$ mátrix megfelelő sorából vett elemekkel képzett
            lineáris kombinációja.
            Magyarul: minden $1\leq i\leq n$ mellett
            \[
                [C]_i=\sum_{s=1}^k[A]_i^s[B]_s.
            \]
        \item
            A szorzat mátrix az $A$ oszlopaiból, és a $B$ soraiból alkotott diádok összege.
            Magyarul:
            \[
                [C]=\sum_{s=1}^k[A]^s[B]_s.\qedhere
            \]
    \end{enumerate}
\end{proposition}
\begin{proof}[1. bizonyítása]
    $[A]_i\cdot [B]^j=\sum_{s=1}^k[A]_{i,s}[B]_{s,j}=[C]_{i,j}.$
\end{proof}
\begin{proof}[2. bizonyítása]
    \(
    \left[ \sum_{s=1}^k[B]_s^j[A]^s \right]_i=
    \sum_{s=1}^k[B]_s^j[A]_i^s =
    \sum_{s=1}^k[A]_i^s[B]_s^j =
    [C]_{i,j}=
    [C]_i^j
    \)
    minden $i$-re.
\end{proof}
\begin{proof}[3. bizonyítása]
     \(
     \left[ \sum_{s=1}^k[A]_i^s[B]_s \right]^j=
     \sum_{s=1}^k[A]_i^s[B]_s^j=
    [C]_{i,j}=
    [C]_i^j
     \)
     minden $j$-re.
\end{proof}
\begin{proof}[4. bizonyítása]
    \(
    \left[ \sum_{s=1}^k[A]^s[B]_s \right]_{i,j}=
    \sum_{s=1}^k\left[ [A]^s[B]_s \right]_{i,j}=
    \sum_{s=1}^k [A]_i^s[B]_s^j=
    C_{i,j}
    \)
    minden $i$-re $j$-re.
\end{proof}
\begin{definition}[Kronecker-delta, identitás mátrix]\index{Kronceker-delta}\index{identitás mátrix}
    \emph{Kroencker-deltának} nevezzük az alábbi egyszerű szimbólumot:
    \[
        \delta_{i,j}=
        \begin{cases}
            1,\text{ ha }i=j;\\
            0,\text{ egyébként.}
        \end{cases}
    \]
    Adott $n\geq 1$ természetes számra az $n\times n$ méretű identitás mátrix azaz $I\in\mathbf{F}^{n\times n}$ mátrix,
    amelyre
    \[
        [I]_{i,j}=\delta_{i,j}.\qedhere
    \]
\end{definition}
Nyilvánvaló, hogy ha $A\in\mathbf{F}^{m\times n}$ mátrix és $I\in\mathbf{F}^{n\times n}$ méretű identitás mátrix,
akkor $A\cdot I=A$. Hasonlóan, ha most $I$ az $m\times m$ identitás mátrixot jelöli, akkor pedig $I\cdot A=A$ azonosság teljesül.
Persze, ha $A\in\mathbf{F}^{n\times n}$ négyzetes mátrix és $I\in\mathbf{F}^{n\times n}$ az ugyanilyen méretű identitás mátrix,
akkor
\[
    IA=AI=A
\]
is fennáll.

A mátrixok szorzásának legérdekesebb tulajdonsága a szorzás asszociativitása.
\begin{proposition}
    Legyen $A,B,C$ mátrixok úgy, hogy $AB$ értelmes és $BC$ is értelmes, 
    azaz $A\in\mathbf{F}^{m\times k},B\in\mathbf{F}^{k\times l}, C\in\mathbf{F}^{l\times n}$.
    Ekkor
    \[
        A\left( BC \right)=\left( AB \right)C.\qedhere
    \]
\end{proposition}
\begin{proof}
    Először is azt vegyük észre, hogy ha a három mátrix egyike szám, és a másik két mátrix
    összeszorozható, 
    akkor a mátrix szorzás definíciója szerint az állítás nyilvánvaló.

    Másodszor azt vegyük észre, hogy mindkét oldalon azonos méretű,
    konkrétan $m\times n$ méretű mátrixok szerepelnek.
    
    Azt kell tehát még meggondolnunk,
    hogy az $i$-edik sor $j$-edik eleme mindkét oldalon ugyanaz.
    A jobboldalon ez
    \begin{multline*}
        [AB]_i\cdot [C]^j
        =
        \left( \sum_{s=1}^k[A]_i^s[B]_s \right)[C]^j
        =
        \sum_{s=1}^k\left([A]_i^s[B]_s\right)[C]^j
        =
        \sum_{s=1}^k[A]_i^s\left([B]_s[C]^j\right)
        \\
        =
        \sum_{s=1}^k[A]_i^s\left(\sum_{r=1}^l[B]_s^r[C]_r^j\right)
        =
        \sum_{s=1}^k\sum_{r=1}^l[A]_i^s([B]_s^r[C]_r^j).
    \end{multline*}
    A baloldalon
    az $i$-edik sor $j$-edik eleme hasonló számolgatással:
    \begin{multline*}
        [A]_i[BC]^j=
        [A]_i\left( \sum_{r=1}^l[B]^r[C]_r^j \right)
        =
        \sum_{r=1}^l[A]_i\left([B]^r[C]_r^j\right)
        =
        \sum_{r=1}^l\left([A]_i[B]^r\right)[C]_r^j
        \\
        \sum_{r=1}^l\left(\sum_{s=1}^k[A]_i^s[B]_s^r\right)[C]_r^j
        =
        \sum_{r=1}^l\sum_{s=1}^k\left([A]_i^s[B]_s^r\right)[C]_r^j.
    \end{multline*}
    A testben fennálló asszociativitás és kommutativitás miatt a bal- és a jobboldali kifejezés azonos.
\end{proof}
\begin{proposition}
    Legyen $n\in\mathbf{N}$ természetes szám, és tekintsük az $n\times n$ méretű mátrixok
    halmazát, ellátva ezt a halmazt a mátrix összeadással és a mátrixszorzással.
    Az $\left( \mathbf{F}^{n\times n},+,\cdot \right)$ algebrai struktúra egy egységelemes gyűrű.
\end{proposition}
Ez a gyűrű, az $n>1$ esetben biztosan nem kommutatív. 
Például $n=2$ mellett
\[
    \begin{pmatrix}
        0&1\\
        0&0
    \end{pmatrix}
    \cdot
    \begin{pmatrix}
        0&0\\
        1&0
    \end{pmatrix}
    =
    \begin{pmatrix}
        1&0\\
        0&0
    \end{pmatrix},
    \text{ amíg }
    \begin{pmatrix}
        0&0\\
        1&0
    \end{pmatrix}
    \cdot
    \begin{pmatrix}
        0&1\\
        0&0
    \end{pmatrix}
    =
    \begin{pmatrix}
        0&0\\
        0&1
    \end{pmatrix}.
\]
Az sem igaz, hogy ez a gyűrű nullosztómentes lenne, hiszen például $n=2$ mellett az
\[
    A
    =
    \begin{pmatrix}
        0&1\\
        0&0
    \end{pmatrix}
\]
mátrix nyilván nem a zérus mátrix (az összeadásra nézve neutrális elem), 
de $A\cdot A=0$. 
Ebből persze már az is következik, hogy a fenti $A$ mátrixnak nincs a szorzásra nézve inverze,
de ez e nélkül is nagyon egyszerűen látszik.%
\footnote{
    Ha 
    \(
        \begin{pmatrix}
            a&b\\
            c&d
        \end{pmatrix}
    \)
    inverze lenne akkor a jobb alsó sarokra figyelve $0\cdot b +0\cdot d=1$ lenne, de egy testben $1\neq 0$.
}
Gyakorlatként próbáljunk magasabb $n$ számok mellett is a kommutativitás és a nullosztómentesség hiányára 
példát találni.

Nagyon fontos látni, hogy a kommutativitás hiánya, az eddigiektől eltérő számolási gyakorlatot eredményez.
A számolás közben a mátrixok sorrendjén nem változtathatunk. 
Persze előfordul, hogy két mátrix szorzata nem függ a sorrendtől. 
Ilyenkor a két mátrixot egymással \emph{felcserélhetőnek}, vagy \emph{kommutálónak} mondjuk.\index{kommutáló mátrixok}
Például, az identitás mátrixszal minden más mátrix kommutál.
Egy mátrixot \emph{diagonális alakúnak}\index{diagonális mátrix} mondunk, 
ha minden nem zérus eleme a fődiagonálisában van.
Az is világos, hogy a diagonális mátrixok egymással kommutálnak.
Fontos része az első féléves anyagnak, hogy ha két négyzetes mátrix szorzata az identitás mátrix,
akkor e két mátrix egymással kommutál.
Ez az eredmény távolról sem nyilvánvaló, és most nem is tudjuk belátni, ehhez már szükség van a lineáris függetlenség fogalmára, amit majd később vezetünk be.

Nagyon is triviális mégis érdemes észrevenni, hogy a fentiek $n=1$ esetben nem jelentenek problémát.
Ilyenkor az $1\times 1$-es mátrixok tere voltaképpen azonos az $\mathbf{F}$-testtel, hiszen csak az a különbség,
hogy egy testbeli $a$ elemet $[a]$ módon írjuk. A szorzás és az összeadás definíciója ugyanazt adja,
ha mint a testbeli elemre, vagy az ebből képzett $1\times 1$-es mátrixra gondolunk.

Az $n\times n$-es négyzetes mátrixok másik érdemleges részstruktúrája az
\[
    \mathcal{F}=
    \left\{ c\cdot I:c\in\mathbf{F} \right\}.
\]
Itt $I$ az $n\times n$ méretű identitás mátrix, tehát $\mathcal{F}$ elemei azon diagonális alakú mátrixok, 
ahol minden elem a diagonálisban azonos.
Világos, hogy két ilyen mátrix összege és szorzata is ilyen marad:
\[
    aI+bI=\left( a+b \right)I\text{ és } aI\cdot bI=\left( ab \right)I.
\]
Ez azt jelenti, hogy az $\left( \mathcal{F},+,\cdot \right)$ struktúra egy egységelemes gyűrű.
Világos, hogy itt bármely két elem kommutál, ergo egy kommutatív egységelemes gyűrűvel állunk szemben és
az is teljesen nyilván való, hogy minden nem zérus elemnek van a szorzásra nézve inverze.
Azt kaptuk tehát, hogy a fenti $\mathcal{F}$ minden $n$ mellett egy test.

\section{A komplex számok mint mátrixok}
\begin{definition}[Izomorf testek]\index{izomorfizmus}
    Legyenek $\mathbf{F}$ és $\mathbf{G}$ testek.
    Azt mondjuk, hogy a két test \emph{izomorf} egymással, 
    ha létezik köztük \emph{művelettartó bijekció}, azaz létezik
    \[
        \varphi:\mathbf{F}\to\mathbf{G}
    \]
    bijekció, amely tartja a műveleteket is, azaz
    \[
        \varphi\left( a+b \right)=\varphi\left( a \right)+\varphi\left( b \right)
        \text{ és }
        \varphi\left( a\cdot b \right)=\varphi\left( a \right)\cdot \varphi\left( b \right).
    \]
    A művelettartó bijekciót \emph{izomorfizmusnak} nevezzük.
\end{definition}
Az izomorf testek közt nem teszünk különbséget. Úgy tekintjük őket, hogy csak jelölésükben különböznek.
Például, ha az $\mathbf{R}$ valós számokra gondolunk, 
akkor a $2\times 2$-es diagonális alakú valós mátrixok közül azok, ahol a diagonális mindkét eleme azonos,
a valós testtel izomorf testet alkot.
\[
    \mathcal{R}=\left\{ 
        \begin{pmatrix}
            a&0\\
            0&a
        \end{pmatrix}
        :a\in\mathbf{R}
    \right\}
    \text{ és }
    \varphi:\mathbf{R}\to\mathcal{R},
    \text{ ahol } 
    \varphi\left( a \right)
    =
    \begin{pmatrix}
         a&0\\
         0&a
    \end{pmatrix}.
\]

A $2\times 2$-es valós mátrixok egységelemes gyűrűjében tehát $\mathcal{R}$ egy olyan részgyűrű, ami még test is,
és izomorf az $\mathbf{R}$ valós számtesttel.
Voltaképpen azt csináltuk, hogy a valós számtestet beágyaztuk a $2\times 2$-es mátrixok közé,
azaz egy $a$ valós számot az
\begin{math}
    \begin{pmatrix}
        a&0\\
        0&a
    \end{pmatrix}
\end{math}
mátrixszal reprezentálunk (írunk le).

A $2\times 2$ méretű valós mátrixok, még sok-sok más testet is tartalmaznak. 
Ezek közül számunkra a legfontosabb a következő részhalmaz
\begin{defprop}[komplex számtest]
    Jelölje két tetszőleges $a,b\in\mathbf{R}$ valós szám mellett $M_{a,b}$ az 
    az $\left( a,b \right)$ valós számpárhoz tartozó 
    \(
        M_{a,b}
        =
        \begin{pmatrix}
            a&-b\\
            b&a
        \end{pmatrix}
    \)
    mátrixot.
    Tekintsük az ilyen típusú mátrixok $\mathcal{C}$-vel jelölt halmazát:
    \[
        \mathcal{C}
        =
        \left\{ 
            M_{a,b}
            :a,b\in\mathbf{R}
        \right\}
    \]
    E részhalmaz 
    \begin{enumerate}
        \item 
        zárt a mátrix összeadásra és a mátrix szorzásra, 
        így a $2\times 2$-es valós mátrixok egy speciális egységelemes részgyűrűje.
        \item
        E részgyűrűben a mátrix szorzás kommutatív művelet, 
        és 
    \item
        és e részgyűrűben minden nem zérus mátrixnak van inverze is a mátrixszorzás műveletre nézve.
    \end{enumerate}
    Eszerint a $\left( \mathcal{C},+,\cdot \right)$ algebrai struktúra egy test.
    Ezt a testet nevezzük a \emph{komplex számtestnek}\index{komplex számok},
    vagy a \emph{komplex számtest mátrix reprezentációjának}.\index{komplex számok mátrix reprezentációja}
\end{defprop}
\begin{proof}
    Az $a,b,c,d\in\mathbb{R}$ valós számok mellett
    \[
        M_{a,b}+M_{c,d}=
        \begin{pmatrix}
            a&-b\\
            b&a
        \end{pmatrix}
        +
        \begin{pmatrix}
            c&-d\\
            d&c
        \end{pmatrix}
        =
        \begin{pmatrix}
            a+c&-b-d\\
            b+d&a+c
        \end{pmatrix}
        =
        M_{a+c,b+d}
    \]
    és hasonlóan a mátrix szorzás definíciója szerint
    \[
        M_{a,b}\cdot M_{c,d}=
        \begin{pmatrix}
            a&-b\\
            b&a
        \end{pmatrix}
        \cdot
        \begin{pmatrix}
            c&-d\\
            d&c
        \end{pmatrix}
        =
        \begin{pmatrix}
            ac-bd&-ad-bc\\
            bc+ad&-bd+ac
        \end{pmatrix}
        =
        M_{ac-bd,ad+bc}.
    \]
    Mivel $M_{1,0}$ a $2\times 2$-es identitás mátrix, ezért $\mathcal{C}$ a mátrix összeadásra és a mátrixszorzásra nézve
    egységelemes gyűrűt alkot.

    A szorzás kommutativitása is látszik a fenti számolásból, hiszen
    \[
        M_{c,d}\cdot M_{a,b}=M_{ca-db,cb+da}=M_{ac-bd,ad+bc}=M_{a,b}\cdot M_{c,d}
    \]
    a valós számok összeadásának és szorzásának 
    kommutativitása miatt.

    Legyen most $M_{a,b}$ egy nem zérus mátrix, így $a^2+b^2\neq 0$.
    Világos, hogy 
    \[
    M_{a,b}\cdot M_{a,-b}=M_{a^2+b^2,0}=\left( a^2+b^2 \right)M_{1,0}=\left( a^2+b^2 \right)I.
    \]
    Ebből már látszik is, hogy $M_{a,b}\cdot M_{\frac{a}{a^2+b^2},\frac{-b}{a^2+b^2}}=I$.
    Ez a már igazolt kommutativitással éppen azt jelenti, hogy minden nem zérus elemnek van multiplikatív inverze,
    ergo $\mathcal{C}$ valóban test.
\end{proof}
Ebben a $\mathbb{C}$ testben az 
\(
    M_{0,-1}=
    \begin{pmatrix}
        0&-1\\
        1&0
    \end{pmatrix}
\)
elem olyan, hogy a négyzete a szorzásra nézve reprodukáló elemnek az összeadásra nézve képzett inverze:
\[
    M_{0,-1}\cdot M_{0,-1}
    =
    \begin{pmatrix}
        0&-1\\
        1&0
    \end{pmatrix}
    \cdot
    \begin{pmatrix}
        0&-1\\
        1&0
    \end{pmatrix}
    =
    \begin{pmatrix}
        -1&0\\
        0 &-1
    \end{pmatrix}
    =
    M_{-1,0}
    =
    -I.
\]
Ez a tulajdonság azért figyelemre méltó, 
mert ha a szokásoknak megfelelően a test multiplikatív neutrális elemét az $1$ szimbólummal jelöljük, 
akkor olyan elemet találtunk a komplex számtestben, 
amelynek négyzete éppen $-1$.
Tudjuk, hogy a valós számtest esetében ez nem lenne lehetséges.

Tekintsük most valamely $a,b\in\mathbb{R}$ mellett az 
\[
    M_{a,b}=
    \begin{pmatrix}
        a&-b\\
        b&a
    \end{pmatrix}
    =
    \begin{pmatrix}
        a&0\\
        0&a
    \end{pmatrix}
    +
    \begin{pmatrix}
        0&-1\\
        1&0
    \end{pmatrix}
    \cdot
    \begin{pmatrix}
        b&0\\
        0&b
    \end{pmatrix}
\]
felbontást.
Ha bevezetjük az 
\(
    i=M_{0,1}=
    \begin{pmatrix}
        0&-1\\
        1&0
    \end{pmatrix}
\)
jelölést\index{$i$ komplex szám}, akkor minden komplex szám
\[
    M_{a,b}=
    \begin{pmatrix}
        a&0\\
        0&a
    \end{pmatrix}
    +
    i
    \cdot
    \begin{pmatrix}
        b&0\\
        0&b
    \end{pmatrix}
\]
alakban írható.
Emlékezzünk arra, 
hogy a valós számtest is részhalmaza a komplex számoktestnek abban az értelemben, 
ha minden $a$ valós számot az
\begin{math}
    \begin{pmatrix}
        a&0\\
        0&a
    \end{pmatrix}
\end{math}
mátrixszal reprezentálunk. ($\mathcal{R}\subseteq\mathcal{C}$).
Ha tehát megegyezünk abban, hogy az $a$ valós számra nézve mi az
\(
    \begin{pmatrix}
        a&0\\
        0&a
    \end{pmatrix}
\)
mátrixra gondolunk,%
\footnote{Kicsit pontosabban: a valós számok $2\times 2$-es mátrix reprezentációját használjuk.}
akkor azt kapjuk, hogy minden komplex szám
\[
    M_{a,b}=a+ib
\]
alakú, ahol $i$ egy olyan komplex szám, amelyre $i^2=-1$, $a,b\in\mathbb{R}$.
Ezt nevezzük a komplex szám \emph{normálalakjának}.\index{komplex szám normálalakja}

Ne felejtsük a műveleteket:
Láttuk, hogy $M_{a,b}+M_{ c,d}=M_{a+c,b+d}$. 
Ez a normálalak reprezentáció mellett azt jelenti, hogy az összeadás definíciója csak
\[
    (a+ib)+(c+id)=(a+c)+i(b+d)\tag{\dag}
\]
lehet.
Hasonlóan emlékszünk, hogy
\begin{math}
    M_{a,b}\cdot M_{c,d}=M_{ac-bd,ad+bc}
\end{math},
ami normálalak reprezentáció mellett az
\[
    (a+ib)\cdot (c+id)=(ac-bd)+i(ad+bc)\tag{\ddag}
\]
definíciót eredményezi.

A következőket gondoltuk meg:
\begin{defprop}[komplex számtest a normálakkal]
    Definiálja
    \[
        \mathbb{C}
        =
        \left\{ 
            a+ib:a,b\in\mathbb{R}
        \right\}
    \]
    a \emph{komplex számok normálalakját}.
    Az összeadás műveletet definiálja (\dag), és a szorzás műveletet definiálja (\ddag).
    Az így kapott algebrai struktúra test, amely izomorf a komplex számtest mátrix reprezentációjával.
    Az izomorfizmust a 
    \[
        \varphi:\mathcal{C}\to\mathbb{C},\quad 
        \varphi
        \left( 
        \begin{matrix}
            a&-b\\
            b&a
        \end{matrix}
        \right)
        =
        a+ib
    \]
    művelettartó bijekció hozza létre.
\end{defprop}

Ha már megértettük, hogy a komplex számok normálalak reprezentációja testet alkot,
akkor a (\dag) és (\ddag) definíciók megjegyzése nagyon könnyű.
Más nem is lehet:
Az összeadáshoz (\dag) csak el kell végezni a műveletet majd kiemelni $i$-t,
 a szorzás definíciójához (\ddag) 
az $i$ kiemelése után jutunk.

\section{A komplex számok abszolút értéke}
A valós számokra jól ismert abszolút érték függvényt terjesztjük ki komplex számtest elemeire.
\begin{definition}[valós rész, képzetes rész]\index{komplex szám valós része}\index{komplex szám képzetes része}
    Legyen $z\in\mathbf{C}$, $z=a+ib$.
    Ekkor $a\in\mathbf{R}$ a $z$ komplex szám \emph{valós része}, 
    és $b\in\mathbf{R}$ a $z$ komplex szám \emph{képzetes része}.
    $\Re z$ jelöli a valós részt, és $\Im z$ a képzetes részt.
\end{definition}
\begin{definition}[konjugált]\index{komplex szám konjugáltja}
    Legyen $z\in\mathbf{C}$, $z=a+ib$.
    Ekkor $z$ \emph{konjugáltja} $\bar{z}=a-ib$.
\end{definition}
\begin{proposition}
Minden $z\in\mathbf{C}$ komplex szám mellett
\[
    \bar{\bar{z}}=z,\quad
    z+\bar{z}=2\Re z,\quad
    z-\bar{z}=2i\Im z,\quad
    z\in\mathbf{R}\iff z=\bar{z},\quad
    z\bar{z}=(\Re z)^2+(\Im z)^2\geq 0,
\]
    és bármely két $z,w\in\mathbf{C}$ komplex szám esetén
    \[
        \overline{z+w}=\bar{z}+\bar{w},\quad
        \overline{zw}=\bar{z}\overline{w},\quad
        \overline{z-w}=\bar{z}-\bar{w},\quad
        \overline{\left( \frac{z}{w} \right)}=\frac{\bar{z}}{\bar{w}}\quad\text{ feltéve, hogy }w\neq 0.\qedhere
    \]
\end{proposition}
Most használjuk először a valós számok rendezését.
Az $\mathbf{R}$ testen a $\geq $ relációt az algebrai műveletekkel a következő két axióma kapcsolja össze:
Minden $a,b,c\in\mathbf{R}$ mellett
\[
    a\geq b\text{ esetén }a+c\geq b+c,\qquad a,b\geq 0\text{ esetén }ab\geq 0.
\]
Az $a^2-b^2=\left( a+b \right)\left( a-b \right)$ azonosság szerint,
ha $a\geq b\geq 0$ akkor $a^2\geq b^2$ is fennáll.

\begin{definition}[komplex szám abszolút értéke]\index{abszolút érték}
    Legyen $z\in\mathbf{C}$ egy komplex szám.
    Láttuk, hogy 
    $z\bar{z}\in\mathbf{R}$ és  
    $z\bar{z}\geq 0$.
    E szám négyzetgyökét nevezzük a $z$ komplex szám \emph{abszolút értékének}.
    Jelölés: $|z|=\sqrt{z\bar{z}}$.
\end{definition}
Fontos látni, hogy ha speciálisan $\Im z=0$, 
tehát ha $z$ egy valós szám,
akkor e definíció szerint
\[
    |z|=
    \sqrt{z^2}=
    \begin{cases}
        z&\text{, ha }z\geq 0\\
        -z&\text{, ha }z<0\text{,}
    \end{cases}
\]
ami egybeesik a valós számok abszolút értékének definíciójával.
Világos, hogy $|\Re z|^2\leq(\Re z)^2+(\Im z)^2=z\bar{z}$, emiatt
\[
    \Re z\leq |z|.
\]
Az abszolút érték legfontosabb tulajdonságai:
\begin{proposition}
    Legyen $z,w\in\mathbf{C}$ komplex szám.
    Ekkor
    \begin{enumerate}
        \item $|z|=0$ akkor és csak akkor, ha $z=0$,
        \item $|zw|=|z||w|$,
        \item $|z+w|\leq |z|+|w|$.\qedhere
    \end{enumerate}
\end{proposition}
Az utolsó egyenlőtlenséget \emph{háromszög egyenlőtlenségnek}\index{háromszög egyenlőtlenség} nevezik.
Egy kevésbé népszerű, de ekvivalens alakja
\[
    \left|\left( |z|-|w| \right)\right|\leq|z-w|.
\]
Vegyük észre, hogy a valós abszolút érték függvény tulajdonságait sem használtuk,
és a fenti tulajdonságokat valós esetre is újra igazoltuk.
\section{A komplex számok trigonometrikus alakja}
Láttuk, hogy $z,w\in\mathbf{C}$ komplex szám mellett
\[
    \Re\left( z+w \right)=\Re z+\Re w
    \qquad
    \text{ és }
    \qquad
    \Im\left( z+w \right)=\Im z+\Im w.
\]
Ez azt jelenti, hogy ha az $a+ib$ komplex számot azonosítjuk az $\mathbf{R}^2$ sík
$\left( a,b \right)$ pontjával, 
akkor egyszerűen koordinátánként kell összeadni a komplex számokat, mintha a $z$ komplex szám
az origóból az $\left( a,b \right)$ pontra mutató vektor lenne.

A kérdés, hogy ha így képzeljük a komplex számokat, 
akkor a komplex számok szorzása mit jelent a komplex számoknak megfeleltetett vektorok körében?

\begin{defprop}[komplex szám trigonometrikus alakja]\index{komplex szám trigonometrikus alakja}
    Legyen $z\in \mathbf{C}$ egy nem zérus komplex szám.
    Ekkor létezik $\varphi\in\mathbf{R}$ valós szám, hogy
    \[
        z=
        |z|\left( \cos\varphi+i\sin\varphi \right)\tag{\dag}
    \]
    Ezt a $\varphi$ számot nevezzük a $z$ komplex szám \emph{argumentumának}, és $\arg z$ módon jelöljük.
    \index{komplex szám argumentuma}
    A (\dag) alak a komplex szám \emph{trigonometrikus alakja}.

    Két nem zérus komplex szám egyenlősége azt jelenti, hogy az abszolút értékük azonos,
    és az argumentumaik különbsége $2\pi$ többszöröse.
\end{defprop}
\begin{proof}
    Világos, hogy ha $z=a+ib\neq 0$, akkor $a^2+b^2>0$, így
    \[
        z=a+ib
        =
        \sqrt{a^2+b^2}\left( \frac{a}{\sqrt{a^2+b^2}}+i\frac{b}{\sqrt{a^2+b^2}} \right).
    \]
    Ha $x$ jelöli a fenti zárójelben a valós részt és $y$ a képzetes részt,
    akkor $x^2+y^2=1$.
    Így az $\left( x,y \right)$ pár a sík egységkörének egy pontja.
    A $\cos$ és $\sin$ függvény definíciója szerint,
    ha $\varphi$ jelöli az origót az $\left( x,y \right)$ 
    ponttal a körcikk peremén mért ív hosszát,
    akkor $x=\cos\varphi$ és $y=\sin\varphi$.
\end{proof}
\begin{proposition}
    Legyenek a $z,w\in \mathbf{C}$ nem nulla komplex számok a trigonometrikus alakjukban felírva, 
    azaz
    \[
        z=|z|\left( \cos\varphi+i\sin\varphi \right),\qquad 
        w=|w|\left( \cos\psi+i\sin\psi \right).
    \]
    Ekkor a két szám szorzatának abszolútértéke az abszolútértékek szorzata
    és a szorzat argumentuma az argumentumok összege, azaz
    \[
        zw=
        |z||w|\left( \cos\left( \varphi+\psi \right)+i\sin\left( \varphi+\psi \right) \right).
    \]
    Speciálisan minden $n\in\mathbf{Z}$ egész számra\index{Moivre-formula}
    \[
        z^n=|z|^n\left( \cos n\varphi+i\sin n\varphi \right).\qedhere
    \]
\end{proposition}

\section{Az algebra alaptétele}

%\chapter{Gauss-Jordan elimináció}
%\scwords Lineáris egyenletrendszerek megoldását automatizáljuk.\index{lineáris egyenletrendszer}


\chapter{Vektortér fogalma}
\chapter{Altér}
\chapter{Lineárisan független rendszerek és generátor rendszerek}
\begin{definition}[lineárisan összefüggő vektorrendszer]\index{lineárisan összefüggő rendszer}
    Egy véges $\left\{ y_1,\dots,y_n \right\}$ vektorrendszert \emph{lineárisan összefüggőnek}
    mondunk, ha egyik vektora kifejezhető a többi vektor lineáris kombinációjaként.
\end{definition}
Úgy is fogalmazhatnánk, hogy az $\left\{ y_1,\dots,y_n \right\}$ rendszer pontosan akkor
lineárisan összefüggő, ha létezik $1\leq k\leq n$ index, amelyre
\[
    y_k\in\lin\left\{ y_1,\dots,y_{k-1},y_{k+1},\dots,y_n \right\}.
\]
\begin{proposition}
    Legyen $\left\{ y_1,\dots,y_n \right\}$ vektorrendszer rögzítve a $V$ vektortérben.
    A vektorrendszerre tett alábbi feltevések egymással ekvivalensek.
    \begin{enumerate}
        \item Lineárisan összefüggő;
        \item Van olyan elem a vektortérben, amely nem csak egyféleképpen áll elő mint az $y_1,\dots,y_n$
            vektorok lineáris kombinációja,\\
            azaz formálisabban:
            létezik z\in V, amelyre $z=\sum_{j=1}^n\xi_jy_j$ és $z=\sum_{j=1}^n\eta_jy_j$
            és létezik $1\leq k\leq n$, amelyre $\xi_k\neq\eta_k$.
        \item Vannak olyan nem mind zérus $\alpha_1,\dots,\alpha_n$ skalárok, amelyekkel
            \[
                \sum_{j=1}^n\alpha_jy_j=0.\qedhere
            \]
    \end{enumerate}
\end{proposition}
\begin{proof}
    Körben bizonyítunk.
    \begin{description}
        \item[$1.\Rightarrow 2.$] 
            Tegyük fel, hogy $y_k=\sum_{j=1}^{k-1}\eta_jy_j+\sum_{j=k+1}^n\eta_jy_j$.
            Ekkor az alábbi együttható rendszerek
            \[
                \left( \alpha_1,\dots,\alpha_{k-1},0,\alpha_{k+1},\dots,\alpha_n \right)
                \qquad
                \left( 0,\dots,0,1,0,\dots,0 \right)
            \]
            a $k$-adik helyen biztosan különböznek, 
            hiszen $0\neq 1$, 
            és mind a két együttható rendszerrel képzett lineáris kombináció ugyanazt az $y_k$ vektort eredményezi.
        \item[$2.\Rightarrow 3.$]
            Világos, hogy 
            \[
                0=z-z=
                \sum_{j=1}^n\left( \xi_j-\eta_j \right)y_j
            \]
            és a $k$-adik skalár nem zérus.
        \item[$3.\Rightarrow 1.$]
            Tegyük fel most, hogy 
            \(
            \sum_{j=1}^n\alpha_jy_j=0,
            \)
            és, hogy $\alpha_k\neq 0.$
            Ekkor 
            \[
                y_k=\sum_{j=1,j\neq k}^n-\frac{1}{\alpha_k}\alpha_jy_j
            \]
            azaz a $k$-adik vektor tekinthető mint a többi vektor valamely lineáris kombinációja.
    \end{description}
    Ezt kellett belátni. 
\end{proof}
Fontos észrevétel a következő.
\begin{proposition}
    Minden, 
    valamely lineárisan összefüggő vektorrendszert tartalmazó vektorrendszer maga is lineárisan összefüggő.
\end{proposition}
\begin{definition}
    Egy nem feltétlen véges vektorrendszert lineárisan összefüggőnek nevezünk,
    ha van véges részrendszere, amely lineárisan összefüggő.
\end{definition}
\begin{definition}[lineárisan független vektorrendszer]\index{lineárisan független rendszer}
    Egy vektorrendszer \emph{lineárisan független}, ha nem lineárisan összefüggő.
\end{definition}
Így egy nem véges vektorrendszer akkor lineárisan független, ha minden véges részrendszere is az.
Egy véges vektorrendszer lineárisan függetlenségét, pedig a következő egymással ekvivalens állítások karakterizálják.
\begin{proposition}
    Legyen $\left\{ y_1,\dots,y_n \right\}$ vektorrendszer rögzítve a $V$ vektortérben.
    A vektorrendszerre tett alábbi feltevések egymással ekvivalensek.
    \begin{enumerate}
        \item Lineárisan független;
        \item A vektorrendszer lineáris burkában minden elem egyetlen egyféleképpen áll elő,
            mint az $y_1,\dots,y_n$ vektorok lineáris kombinációja.
        \item Az $y_1,\dots,y_n$ vektoroknak csak a triviális lineáris kombinációja zérus,
            azaz
            \[
                \sum_{j=1}^n\alpha_jy_j=0\text{ esetén }\alpha_1=\alpha_2=\dots=\alpha_n=0.\qedhere
            \]
    \end{enumerate}
\end{proposition}
\begin{proof}
    Nyilvánvaló a lineáris összefüggés karakterizációjából.
\end{proof}
\begin{definition}[maximális lineárisan független-- és minimális generátorrendszer]\index{maximális lineárisan független rendszer}\index{minimális generátorrendszer}
    Egy lineárisan független rendszert \emph{maximális lineárisan független rendszernek} nevezünk,
    ha nem lehet bővíteni úgy, hogy lineárisan független maradjon.

    Egy generátorrendszert \emph{minimális generátor rendszernek} mondunk, ha nem lehet szűkíteni úgy,
    hogy generátorrendszer maradjon.
\end{definition}
\begin{proposition}
    Legyen $\left\{ x_1,\dots,x_m \right\}$ egy vektorrendszere a $V$ vektortérnek.
    Az alábbi feltevések ekvivalensek.
    \begin{enumerate}
        \item A vektorrendszer maximális lineárisan független rendszer.
        \item A vektorrendszer egyszerre lineárisan független és generátorrendszer.
        \item A vektorrendszer minimális generátorrendszer.\qedhere
    \end{enumerate}
\end{proposition}
\begin{proof}
    Az alábbi lépéseket követjük.
    \begin{itemize}
        \item[1.\Rightarrow 2.]
            Ha a vektorrendszer nem lenne generátorrendszer is,
            akkor a lineáris burkán kívül lenne egy vektor.
            Ezt a vektorrendszerhez illesztve, a vektorrendszer egy valódi lineárisan független bővítését kapjuk,
            ami ellentmond a maximalitás feltételének.
        \item[3.\Rightarrow 2.]
            Ha a vektorrendszer egyik eleme a többi elem lineáris kombinációja,
            akkor azt az elemet elhagyva is generátorrendszert kapunk,
            ami ellentmond a minimalitás feltételének.
        \item[2.\Rightarrow 1.]
            Ha nem lenne maximális a lineárisan független tulajdonságra nézve,
            akkor létezne egy vektor a lineáris burkán kívül is,
            ami ellentmond a generátorrendszer tulajdonságnak.
        \item[2.\Rightarrow 3.]
            Mivel a vektorrendszer egyik eleme, sincs a többi lineáris burkában,
            ezért egyetlen elemet sem hagyhatunk el a generátorrendszer tulajdonság megtartásával,
            ami azt jelenti, hogy ez egy minimális generátorrendszer.\qedhere
    \end{itemize}
\end{proof}
Egy a zéró vektort tartalmazó vektorrendszer persze lineárisan összefüggő, 
és egy nem zérus vektorból álló egyelemű vektorrendszer lineárisan független.
A következő állítás sokszor teszi kényelmessé a gondolatmenetünket. 
\begin{proposition}
    Legyen $\left\{ y_1,\dots,y_n \right\}$ egy olyan legalább két elemű vektorrendszer,
    amelynek első eleme nem a zérus vektor, tehát $y_1\neq 0$.
    A vektorrendszer pontosan akkor lineárisan összefüggő,
    ha létezik olyan eleme, 
    amely pusztán az előző elemek lineárisan kombinációja.

    Formálisabban: akkor és csak akkor, 
    ha 
    $\exists k\quad 2\leq k\leq n : y_k\in\lin\left\{ y_1,\dots,y_{k-1} \right\}$
\end{proposition}
\begin{proof}
    Tegyük fel, hogy a vektorrendszer lineárisan összefüggő.
    Ekkor van olyan a zérus vektort eredményező lineáris kombinációja
    \(
    \alpha_1y_1+\dots+\alpha_ny_n=0,
    \)
    ahol nem az összes együttható nulla.
    Legyen $k$ a lineáris kombinációban a legnagyobb nem nulla együttható indexe.
    Világos, hogy $k\neq 1$, 
    hiszen $y_1\neq 0$.
    Persze a $k$ feletti együtthatók mind nullák,
    emiatt
    \[
        \alpha_1y_1+\dots+\alpha_ky_k=0.
    \]
    Itt már $\alpha_k\neq 0$, tehát $y_k$ kifejezhető az előző vektorok segítségével:
    \[
        y_k=
        \frac{-1}{\alpha_k}\alpha_1y_1+\dots+\frac{-1}{\alpha_{k-1}}\alpha_{k-1}y_{k-1}.\qedhere
    \]
\end{proof}
\chapter{Vektortér bázisa}
\scwords A Steinitz-lemma fontosságát kell kiemelni.
A Steinitz-lemma legegyszerűbb megfogalmazásában azt állítja, 
hogy \emph{egy lineárisan független rendszer elemszáma, soha nem nagyobb mint egy generátorrendszer elemszáma.}
Ez a tény vezet a bázis és a dimenzió fogalmához.

A generátorrendszer cseréről szóló legelső lemmának is fontos szerepe van az itt választott felépítésben.
Egyrészt használjuk a Steinitz-lemma igazolásában, másrészt ennek segítségével tisztázzuk majd azt a kérdést,
hogy hogyan alakulnak egy vektor koordinátái, ha a bázist, tehát a vonatkoztatási rendszert változtatjuk.

\begin{lemma}[generátorrendszer csere]
    Legyen $\left\{ x_1,\dots,x_m \right\}$ egy generátor rendszere valamely vektortérnek,
    és tegyük fel, hogy valamely $y$ vektorra
    \[
        y=\sum_{j=1}^m\eta_jx_j,
    \]
    ahol $\eta_k\neq 0$ valamely $1\leq k\leq m$ mellett. 
    Ekkor $y$ becserélhető a $k$-adik helyen a generátor rendszerbe, 
    úgy hogy az generátorrendszer maradjon, azaz a
    \[
        \left\{ x_1,\dots,x_{k-1},y,x_{k+1},\dots,x_m \right\}
    \]
    vektorrendszer is generátorrendszer.
    \label{le:gencsere}
\end{lemma}
\begin{proof}
    Fejezzük ki $x_k$-t az $y$-ra felírt formulából:
    \(
    x_k=\frac{1}{\eta_k}y+\sum_{j=1,j\neq k}^m\frac{-1}{\eta_k}\eta_jx_j.
    \)
    Ha $z$ eredetileg 
    \[
        z=\sum_{j=1}^m\zeta_jx_j
    \]
    alakú, akkor $x_k$ helyére betéve, a fent kifejezett formulát és bevezetve a 
    $\delta=\frac{\zeta_k}{\eta_k}$ jelölést, azt kapjuk hogy:
    \begin{multline*}
        z=\zeta_kx_k+\sum_{j=1,j\neq k}^m\zeta_jx_j=
        \\
        =
        \zeta_k
        \left( 
        \frac{1}{\eta_k}y+\sum_{j=1,j\neq k}^m\frac{-1}{\eta_k}\eta_jx_j
        \right)
        +\sum_{j=1,j\neq k}^m\zeta_jx_j
        =
        \frac{\zeta_k}{\eta_k}y+
        \sum_{j=1,j\neq k}^m\left( \zeta_j-\frac{\zeta_k}{\eta_k}\eta_j \right)x_j=
        \\
        =\delta y+
        \sum_{j=1,j\neq k}^m\left( \zeta_j-\delta\eta_j \right)x_j.
    \end{multline*}
    Azt kaptuk tehát, hogy ha egy vektor kifejezhető az eredeti vektorrendszerből az 
    \[
        \left( \zeta_1,\dots,\zeta_m \right) 
    \]
    együtthatókkal, akkor ugyanez a vektor a módosított vektorrendszerből is kifejezhető,
    méghozzá az 
    \[
        \left( 
        \underbrace{\zeta_1-\delta\eta_1}_{1.},
        \underbrace{\zeta_2-\delta\eta_2}_{2.},
        \dots,
        \underbrace{\zeta_{k-1}-\delta\eta_{k-1}}_{k-1.},
        \underbrace{\delta}_{k.},
        \underbrace{\zeta_{k+1}-\delta\eta_{k+1}}_{k+1.},\dots,
        \underbrace{\zeta_m-\delta\eta_m}_{m.}
        \right)
    \]
    együtthatókkal.
\end{proof}
\begin{SL}\index{Steinitz--lemma}
    Tegyük fel, hogy az $\left\{ y_1,\dots,y_n \right\}$ egy lineárisan független rendszer és az
    $\left\{ x_1,\dots,x_m \right\}$ vektorrendszer egy generátor rendszer.
    Ekkor
    \begin{enumerate}
        \item $n\leq m$;
        \item Az $x_1,\dots,x_m$ vektorok alkalmas átindexelésével az
            \[
                \left\{ y_1,\dots,y_n,x_{n+1},\dots,x_m \right\}
            \]
            vektorrendszer is generátorrendszer.%
            \footnote{Úgy kell a jelöléseket érteni, hogy az $n=0$, de az $n=m$ eset is lehetséges. 
                Az $n=0$ esetben az $y$-okkal jelölt vektorok egyike sem,
                míg az $n=m$ esetben az $x$-el jelölt vektorok egyike sem szerepel az 
                \(
                \left\{ y_1,\dots,y_n,x_{n+1},\dots,x_m \right\}
                \)
            vektorrendszer elemei közt.}%
            \qedhere
    \end{enumerate}
    \label{le:Stienitz}
\end{SL}
\begin{proof}
    Legyen $k$ a legnagyobb egész a $\left\{ 0,\dots,n \right\}$ egészek közül, amelyre
    \begin{enumerate}
        \item $k\leq m$, és
        \item az $x_1,\dots,x_m$ vektorok alkalmas átindexelésével az
            \[
                \left\{ y_1,\dots,y_k,x_{k+1},\dots,x_m \right\}\tag{\dag}
            \]
            vektorrendszer is generátorrendszer.
    \end{enumerate}
    Ilyen $k$ biztosan van van, hiszen $k=0$ triviálisan jó.
    Összesen azt kell meggondolnunk, hogy $k=n$.
    Ha $k<n$ lenne, 
    \begin{itemize}
        \item 
            akkor létezne $y_{k+1}$ vektor.
            No de, ez az $y_{k+1}$ nem szerepel az $\left\{ y_1,\dots,y_k \right\}$ lineáris burkában,
            ami (\dag) generátorrendszer volta miatt csak úgy lehetséges, hogy $k<m$, 
            ergo $k+1\leq m$.
        \item
            \Aref{le:gencsere}. lemma szerint a (\dag) vektorrendszerben az $y_{k+1}$ vektor 
            avval az $x$-el
            --- a generátorrendszer tulajdonság megtartásával is --- 
            kicserélhető, 
            amely $x$ szerepel az $y_{k+1}$ vektornak a (\dag)-beli
            vektorokkal képzett lineáris kombinációjában. 
    \end{itemize}
    Ez ellentmondás, hiszen $k$ a legnagyobb olyan szám, 
    amelyre a bizonyítás elején szereplő 1. és 2. feltételek egyszerre állnak fenn.
\end{proof}
\begin{corollary}
    Egy vektortérben bármely két véges egyszerre lineárisan független és egyszerre generátorrendszer elemszáma azonos.
    Konkrétabban, ha
    \[
        \left\{ x_1,\dots,x_m \right\} \text{ és } \left\{ y_1,\dots,y_n \right\}
    \]
    lineárisan független generátor rendszerek, akkor $n=m$.
    \label{co:baziselemszam}
\end{corollary}
\begin{definition}[végesen generált vektortér]\index{végesen generált vektortér}
    Egy vektorteret \emph{végesen generáltnak} nevezünk,
    ha létezik véges elemszámú generátor rendszere.
\end{definition}
Teljesen világos, hogy ha van egy vektortérben véges generátorrendszer,
akkor van minimális generátorrendszer is, azaz van a térben lineárisan független generátorrendszer.
Ezt rögzítjük a következőekben.
\begin{proposition}
    Minden végesen generált vektortérnek van olyan vektorrendszere, 
    amely egyszerre lineárisan független és generátorrendszer.
    \label{pr:bazisletezik}
\end{proposition}
\begin{proof}
    Tekintsünk egy véges generátorrendszert.
    Ha minden elem kívül esik a többi elem lineáris burkában, akkor a rendszer lineárisan független, és készen is vagyunk.
    Ha van olyan elem, amely a többi elem lineáris burkában van, akkor dobjuk el ezt az elemet, és tekintsük, a most már
    eggyel kevesebb elemből álló vektorrendszert. 
    Világos, hogy ez is generátorrendszer marad.

    Folytassuk az eljárást.
    Mivel véges sok vektor van az eredeti generátor rendszerben az algoritmus előbb-utóbb megáll,
    ami azt jelenti, hogy olyan generátorrendszert kapunk, 
    ahol már minden elem a többi lineáris burkán kívül van,
    ergo lineárisan független.
\end{proof}
A lineárisan független generátor rendszerek olyan sűrűn fordulnak elő a tárgyalásban,
hogy rövidebb külön nevet adni nekik.
\begin{definition}[bázis]\index{bázis}
    Egy vektorrendszert \emph{bázisnak} nevezünk, ha ez egyszerre lineárisan független és generátorrendszer.
\end{definition}
\Aref{pr:bazisletezik}. állítást tehát úgy fogalmazhatjuk, hogy végesen generált vektortérnek van bázisa,
és hasonlóan \aref{co:baziselemszam}. következmény pedig azt jelenti, 
hogy egy vektortérben bármely két bázis azonos elemszámú.
Ez utóbbi tény ad értelmet a következő definíciónak:
\begin{proposition}
    Egy végesen generált vektortérről azt mondjuk, hogy $n$ dimenziós, vagy $n$ a dimenzió száma,
    ha a vektortérben van $n$ elemű bázis.
\end{proposition}
Fontos látni, hogy éppen azt gondoltuk meg, hogy \emph{minden végesen generált vektortérben van bázis}, 
\footnote{Ez nem végesen generált vektorterekre is igaz, de itt nem igazoljuk}
és \emph{bármely két bázis pontosan annyi vektorból áll mint a tér dimenziója.}
A végesen generált vektortereket sokszor szinonimaként \emph{véges dimenziósnak} is mondjuk.\index{véges dimenziós}

Az eddigiek összefoglalásaként is tekinthető a következő állítás.
\begin{proposition}
    Tekintsünk egy $m$-dimenziós vektorteret, és abban egy $m$-elemű
    $\left\{ x_1,\dots,x_m \right\}$
    vektorrendszert.
    E vektorrendszerre tett alábbi feltevések ekvivalensek.
    \begin{enumerate}
        \item Lineárisan független;
        \item Maximális lineárisan független rendszer;
        \item Generátorrendszer;
        \item Minimális generátorrendszer;
        \item Bázis.\qedhere
    \end{enumerate}
\end{proposition}
\begin{proof}
    Az első négy feltétel ekvivalenciájával kezdünk.
    \begin{itemize}
        \item[1.\Rightarrow 2.]
            Mivel a tér $m$-dimenziós, ezért van $m$-elemű generátor rendszere,
            így a Steinitz-lemma szerint nincs $m$-nél több elemet tartalmazó lineárisan független
            rendszere, ergo bármely $m$ elemet tartalmazó lineárisan független rendszer maximális is.
        \item[2.\Rightarrow 3.]
            Láttuk korábban.
        \item[3.\Rightarrow 4.]
            Mivel a tér $m$-dimenziós, ezért van $m$-elemű lineárisan független rendszere,
            így a Steinitz-lemma szerint nincs $m$-nél kevesebb elemet tartalmazó generátor rendszere,
            ergo bármely $m$ elemet tartalmazó generátorrendszer rendszer minimális is.
        \item[4.\Rightarrow 1.]
            Láttuk korábban.
    \end{itemize}

    Az első négy feltétel tehát ugyanazt jelenti. 
    Így ha 1.-et feltesszük, akkor 3. is fennáll, ami azt jelenti, hogy 1. feltétel és 5. feltétel is ekvivalensek.
\end{proof}
A Steinitz-lemma kulcs szerepet játszott dimenzió fogalmának megértésében,
hiszen a bázis elemszáma nem lehetne a tér dimenziója, anélkül hogy tudnánk a tényt: 
bármely két bázis azonos elemszámú! 
Márpedig egy vektortérben nagyon sok bázis van. 
A Steinitz-lemma 2. pontja segít ennek megértéséhez.

\begin{proposition}
    Egy végesen generált vektortér bármely lineárisan független rendszere kiegészíthető bázissá.
    \label{pr:lfgtenbazissa}
\end{proposition}
\begin{proof}
    Tegyük fel, hogy a tér $m$ dimenziós, ami azt jelenti, hogy van 
    \[
        \left\{ x_1,\dots,x_m \right\}
    \]
    $m$ elemű lineárisan független generátor rendszere.
    Legyen $\left\{ y_1,\dots,y_n \right\}$ egy lineárisan független.
    A Steinitz-lemma szerint ez a rendszer kiterjeszthető egy 
    \[
        \left\{ y_1,\dots,y_n,x_{n+1},\dots,x_m \right\}
    \]
    generátor rendszerré, ami persze bázis is.
    Ezt kellett belátni. 
\end{proof}

Meggondoltuk tehát, hogy bármely véges generátor rendszerből elhagyható néhány elem úgy, 
hogy a rendszer lineárisan független generátor rendszerré váljon,
és hasonlóan bármely lineárisan független rendszerhez, hozzátehető néhány elem úgy, hogy a
rendszer lineárisan független generátor rendszerré váljon.





\chapter{Koordinátázás}
\section{Rang-tétel}

\begin{definition}[rang, oszloprang, sorrang, feszítőrang]\index{vektorrendszer rangja}\index{oszloprang}\index{sorrang}\index{feszítőrang}
    Egy véges vektorrendszer \emph{rangján} a vektorrendszer generálta altér dimenzióját értjük.
    Egy mátrix \emph{oszloprangján} a mátrix oszlopai mint vektorrendszer rangját értjük.
    Egy mátrix \emph{sorrangján} a mátrix sorai mint vektorrendszer rangját értjük.
    Ha $A\in\mathbb{F}^{n\times m}$ nemzérus mátrix,
    akkor legkisebb olyan $r$ számot, amelyre
    létezik $B\in\mathbb{F}^{n\times r}$ és $C\in\mathbb{F}^{r\times m}$ mátrix úgy, hogy 
    $A=BC$ szorzatfelbontás teljesül,
    az $A$ mátrix \emph{feszítőrangjának} nevezzük.

    Jelölések: Ha $\left\{ x_1,\dots,x_m \right\}$ a szóban forgó vektorrendszer, akkor
    \[
        \rank\left\{ x_1,\dots,x_m \right\}=\dim\lin\left\{ x_1,\dots,x_m \right\}
    \]
    Ha $A\in\mathbb{F}^{n\times m}$ egy mátrix,
    akkor 
    \[
        \crank{A}=\rank\left\{ [A]^{j}:j=1,\dots,m \right\},
        \quad
        \rrank{A}=\rank\left\{ [A]_k:k=1,\dots,n\right\},
    \]
    továbbá $\srank{A}$ jelöli a feszítőrangját $A$-nak.
\end{definition}
\begin{proposition}(Rang-tétel)\label{pr:rang}
    Tetszőleges test feletti tetszőleges mátrix mellett, a fent bevezetett három rang-koncepció azonos.

    Formálisabban: Minden $A\in\mathbb{F}^{n\times m}$ mellett
    \[
        \crank{A}=\srank{A}=\rrank{A}\qedhere
    \]
\end{proposition}
\begin{proof}
    Induljunk ki a feszítőrang fogalmából.
    Legyen $r=\srank{A}$, és 
    \[
        A=BC,\tag{\dag}
    \]
    ahol $B\in\mathbf{F}^{n\times r},C\in\mathbf{F}^{r\times m}$.
    Azt mutatjuk meg, hogy ekkor $B$ oszloprendszere minimális generátor rendszere $A$ oszlop-vektorterének
    és $C$ sorrendszere minimális generátor rendszere $A$ sor-vektorterének.

    Vegyük észre, hogy $\srank{A}\leq \crank{A}$.
    Ugyanis ha az $A$ mátrix oszlop-vektorterének veszünk egy tetszőleges választott
    $b_1,\dots,b_k$ generátor rendszerét, 
    akkor van $B\in\mathbf{F}^{n\times k}$ és $C\in\mathbf{F}^{k\times m}$ mátrix, hogy $A=BC$.
    Az $r=\srank{A}$ szám az ilyen $k$ számok legkisebbike, tehát valóban $r\leq\crank{A}$.
    \\
    Most tekintsük a (\dag)-ben rögzített szorzatot.
    Jelölje $W$ a $B$ mátrix és $V$ az $A$ mátrix oszlopvektorterét.
    Mivel $BC$ oszlopai $B$ oszlopainak lineáris kombinációja, 
    ezért $A$ minden oszlopa beleesik $W$-be, 
    így az $A$ oszlopainak lineáris burka is részhalmaza $W$-nek,
    azaz 
    \[
        V\subseteq W.
    \]
    A $B$ mátrixnak $r$ darab oszlopa van, 
    tehát $\dim W\leq r$.
    Látjuk tehát, hogy 
    \[\dim W\leq r\leq\crank{A}=\dim V,
    \]
    ami csak úgy lehetséges, 
    hogy $V=W$.
    A $B$ oszlopai tehát $V$-nek is generátor rendszerét alkotják,
    és $r$ minimalitása szerint egy elem sem elhagyható a generátorrendszer tulajdonság
    elvesztése nélkül.

    A sorokra vonatkozó indoklás analóg.
    Először is $\srank{A}\leq \rrank{A}$.
    Ugyanis ha az $A$ mátrix sorvektorterének veszünk egy tetszőleges választott
    $c_1,\dots,c_k$ generátor rendszerét, 
    akkor van $B\in\mathbf{F}^{n\times k}$ és $C\in\mathbf{F}^{k\times m}$ mátrix, hogy $A=BC$.
    Az $r=\srank{A}$ szám az ilyen $k$ számok legkisebbike, tehát valóban $r\leq\rrank{A}$.
    \\
    Most tekintsük a (\dag)-ben rögzített szorzatot.
    Jelölje $W$ a $C$ mátrix és $V$ az $A$ mátrix sorvektorterét.
    Mivel $BC$ sorai $C$ sorainak lineáris kombinációja, 
    ezért $A$ minden sora $W$-be esik,
    így az $A$ sorainak lineáris burka is részhalmaza $W$-nek,
    azaz 
    \[
        V\subseteq W.
    \]
    A $C$ mátrixnak $r$ sora van, 
    tehát $\dim W\leq r$.
    Látjuk tehát, hogy 
    \[\dim W\leq r\leq\crank{A}=\dim V,
    \]
    ami csak úgy lehetséges, 
    hogy $V=W$.
    A $C$ oszlopai tehát $V$-nek is generátor rendszerét alkotják,
    és $r$ minimalitása szerint egy elem sem elhagyható a generátorrendszer tulajdonság
    elvesztése nélkül.

    Ezt kellett belátni. 
\end{proof}
\begin{definition}[mátrix rangja]\index{mátrix rangja}
    Mivel a sorrang, az oszloprang, a feszítőrang minden mátrix mellett azonos,
    ezért a továbbiakban a közös értékre a \emph{rang} szót is használjuk.\footnote{Lásd: \citep{Wardlaw2005}}
\end{definition}
\begin{note}
    Érdemes a rang-tétel következő összegzését megjegyezni.
    Legyen $A\in\mathbb{F}^{n\times m}$ mátrix, amelynek $r$ a rangja.
    Ekkor létezik $A=BC$ felbontása, ahol $B\in\mathbb{F}^{n\times r},C\in\mathbb{F}^{r\times m}$.
    Ez a felbontás persze nem egyértelmű, hiszen $A$ oszlop-vektorterének nagyon sok bázisa van.
    Viszont minden ilyen felbontásban $B$ oszloprendszere az $A$ oszlop-vektorterének, 
    míg $C$ sorrendszere az $A$ sorvektorterének minimális generátor rendszerét, ergo bázisát alkotja.
\end{note}
Következményképpen érdemes meggondolni a mátrix és inverzének felcserélhetőségére vezető állítást.
\begin{proposition}
    Legyenek $A,B\in\mathbb{F}^{n\times n}$ négyzetes mátrixok, amelyekre AB=I.
    Ekkor BA=I is teljesül.
\end{proposition}
\begin{proof}
    Az identitás mátrix rangja nyilván $n$.
    E mátrix feszítőrangjának definíciójára gondolva, 
    az előző megjegyzés szerint $A$ oszlopai $\mathbb{F}^n$ lineárisan független rendszerét alkotják.
    Vegyük észre, hogy a mátrix szorzás asszociativitását is kihasználva
    \[
        A\left( BA-I \right)=A\left( BA \right)-AI=\left( AB \right)A-AI=IA-AI=A-A=0.
    \]
    Na most, 
    ha $BA\neq I$ lenne, 
    akkor a $BA-I$ mátrixnak lenne egy olyan nem zérus oszlopa, 
    melynek elemeivel mint együtthatókkal képzett lineáris kombinációja az $A$ oszlopainak 
    a zéró vektort eredményezi.
    Ez persze ellentmond az $A$ oszloprendszer lineáris függetlenségének,
    tehát $BA=I$ valóban fennáll.%
    \footnote{%
        A feszítőrang fogalmának ismerete nélküli -- talán még elemibb -- bizonyítás: \citep{doi:10.4169/college.math.j.48.5.366}%
    }%
\end{proof}
\begin{defprop}[invertálható mátrix]
    Legyen $A\in\mathbb{F}^{n\times n}$ egy négyzetes mátrix.
    Az alábbi feltételek egymással ekvivalensek.
    \begin{enumerate}
        \item Van egyetlen olyan $B\in\mathbb{F}^{n\times n}$ mátrix,
            amelyre $AB=I$,
        \item Van olyan $B\in\mathbb{F}^{n\times n}$ mátrix,
            amelyre $AB=I$,
        \item Van egyetlen olyan $B\in\mathbb{F}^{n\times n}$ mátrix,
            amelyre $BA=I$,
        \item Van olyan $B\in\mathbb{F}^{n\times n}$ mátrix,
            amelyre $BA=I$,
        \item $\rank A=n$,
        \item $A$ oszlopai lineárisan független rendszer alkotnak,
        \item $A$ sorai lineárisan független rendszert alkotnak.
    \end{enumerate}
    Ha a fenti feltételek egyike (ergo mindegyike) fennáll, 
    akkor azt mondjuk, hogy $A$ mátrix \emph{invertálható}\index{invertálható mátrix}.
    Szinonimaként használjuk még a \emph{nemszinguláris}\index{nemszinguláris mátrix}, 
    vagy az \emph{reguláris}\index{reguláris mátrix} szavakat.
    Ha egy mátrix nem invertálható, akkor \emph{szingulárisnak}\index{szinguláris mátrix} nevezzük.

    Egy invertálható négyzetes mátrix esetén azt az egyetlen $B$ mátrixot, amelyre
    \(
        AB=I
    \)
    fennáll az $A$ inverzének mondjuk, és $A^{-1}=B$-vel jelöljük.
    Világos, hogy
\[
    AA^{-1}=I=A^{-1}A,\quad (A^{-1})^{-1}=A.\qedhere
\]
\end{defprop}
\begin{proof}
    A 2., 4., 5., 6., 7. állítások ekvivalenciája nyilvánvaló az előzőek szerint.
    Ha $AB=I=AC$, akkor $A\left( B-C \right)=0$ így $A$ oszloprendszere lineáris függetlensége 
    miatt $B=C$. 
    Ezzel $2.\Rightarrow 1.$ implikációt is beláttuk.
    Az 1. és 3. feltevések ekvivalenciája az előző állítás miatt teljesül.
\end{proof}
\chapter{Elemi fogalmak}
Tegyük fel, hogy, hogy ismerjük az alábbi fogalmakat:
\begin{enumerate}
    \item Vektorrendszer lineáris függetlensége;
    \item Altér, lineáris burok, generátorrendszer;
    \item Mátrix szorzás,
    \item Gauss-Jordan elimináció. Pontosan azt tesszük fel, hogy ha $Q$ egy négyzetes mátrix, melynek oszlopai
        lineárisan függetlenek, akkor az elemi sor műveletekkel az identikus mátrixszá transzformálható.
        Mivel az elemi sorműveletek, bal-szorzások alkalmas mátrixszal, 
        azt kapjuk, hogy létezik $P$ mátrix, amelyre $PQ=I$.
\end{enumerate}
Amit határozottan kerülünk, és azt tesszük fel, hogy nem ismerjük,
\begin{enumerate}
    \item Bázisok fogalma,
    \item Különböző bázisok azonos számossága,
    \item determináns.
\end{enumerate}
Külön probléma a mátrix rangjának definíciója.
Nem definiálhatom, 
mint az oszlop vagy sorvektortér dimenzióját, hiszen a felépítés jelen szintjén még nincs bázis.
Természetesen azt sem tudjuk még, hogy a maximális lineárisan független sor- vagy oszloprendszer választástól függetlenül mindig azonos elemszámú.
A mátrix feszítő rangja viszont definiálható.
\begin{definition}[mátrix feszítőrangja]
    Legyen $A\in\mathbf{F}^{n\times m}$ egy tetszőleges nem zérus mátrix.
    Azt mondjuk, hogy feszítő rangja $r$, ha $r$ a legkisebb olyan pozitív egész, amelyre $A$ előáll
    \[
        A=BC
    \]
    alakban, ahol $B\in\mathbf{F}^{n\times r}$ és $C\in\mathbb{F}^{r\times n}$.
\end{definition}
Világos, hogy tetszőleges nemzérus négyzetes mátrixra ez jól definiált és $1\leq r \leq \min\{n,m\}$.
A rang-tételnek a szokásosnál egy kicsit erősebb formáját lehet megfogalmazni a dimenzió fogalmának bevezetése nélkül,
ami a lenti 3. állítás.

\begin{proposition}
    Az alábbi állítások egymás következményei:
    \begin{enumerate}
        \item Homogén lineáris egyenletrendszernek, amelynek több ismeretlene van, mint egyenlete
            mindig létezik nem triviális megoldása.
        \item Lineárisan független vektorrendszer elemszáma nem nagyobb mint egy generátorrendszer elemszáma.
        \item Minden nemzérus mátrixban 
            a maximális lineárisan független oszloprendszerek 
            és maximális lineárisan független sorrendszerek azonos elemszámúak, 
            és ez a szám egybeesik a mátrix feszítőrangjával.
        \item
            Legyen $A,B\in\mathbb{F}^{n\times n}$ négyzetes mátrixok.
            Ekkor $AB=I$ esetén $BA=I$ is fennáll.\qedhere
    \end{enumerate}
\end{proposition}
\begin{proof}[1. \Rightarrow 2.]
    Legyen $\left\{ y_1,\dots,y_n \right\}$ egy generátorrendszer,
    és $\left\{ x_1,\dots,x_m \right\}$ olyan vektorrendszer a vektortérben, ahol $m>n$.
    Meg kell mutatnunk, hogy ez utóbbi egy lineárisan összefüggő.
    Világos, hogy minden $1\leq k\leq m$ mellett
    \[
        x_k=\sum_{j=1}^n\alpha_{j,k}y_j.
    \]
    Egyenlőre tetszőleges $\xi_1,\dots,\xi_m$ együtthatók mellett
    \begin{eqnarray}
        \sum_{k=1}^m\xi_kx_k=
        \sum_{k=1}^m\sum_{j=1}^n\xi_k\alpha_{j,k}y_j=
        \sum_{j=1}^n\left( \sum_{k=1}^m\alpha_{j,k}\xi_k \right)y_j
        \label{eq:sys}
    \end{eqnarray}
    Tekintsük az $\left( \alpha_{j,k} \right)$ együtthatók generálta
    homogén lineáris egyenletrendszert. 
    Itt $j=1,\dots,n$ és $k=1,\dots,m$.
    Mivel $m>n$, ezért az ismeretlenek száma több mint az egyenletek száma.
    Létezik tehát nem triviális megoldás, azaz léteznek nem mind nulla
    $\xi_1,\dots,\xi_m$ számok, amelyekre minden $j=1,\dots,n$ esetén
    \[
        \sum_{k=1}^m\alpha_{j,k}\xi_k=0.
    \]
    Találtunk tehát az $\left\{ x_1,\dots,x_m \right\}$ vektorrendszernek egy
    nem triviális, 
    de a zéró vektort eredményező,
    lineáris kombinációját (\ref{eq:sys}).
\end{proof}
\begin{proof}[2.\Rightarrow 3.]
    Jelölje $r$ az $A\in\mathbb{F}^{n\times m}$ mátrix feszítőrangját.
    Legyen $r_c$ a mátrix egyik rögzített maximális lineárisan független oszloprendszerének elemszáma.
    \begin{itemize}
        \item 
        Ezen oszlopokat egy $B\in\mathbb{F}^{n\times r_c}$ mátrixba téve 
        -- a maximalitás miatt -- létezik olyan $C\in\mathbb{F}^{r_c\times m}$ mátrix, 
        amelyre $A=BC$, azaz $r\leq r_c$.
        \item
        Most tekintsünk egy tetszőleges olyan
        \(
            A=BC
        \)
        felbontást, 
        ahol $B\in\mathbb{F}^{n\times r}$ és $C\in\mathbb{F}^{r\times m}$.
        Jelölje $W$ a $B$ mátrix oszlopai lineáris burkát. 
        Az $A$ mátrix fent rögzített maximális lineárisan független oszloprendszere egy lineárisan független rendszer a 
        $W$ vektortérben,
        és $B$ oszlopai pedig egy generátorrendszer ugyanebben a vektortérben.
        A 2. állitás szerint $r_c\leq r$.
    \end{itemize}
    Evvel megmutattuk, hogy bármely két maximális lineárisan független oszloprendszer azonos elemszámú, és számuk megegyezik a mátrix feszítőrangjával.


    Legyen $r_w$ az $A$ mátrix egyik rögzített maximális lineárisan független sorrendszerének elemszáma.
    \begin{itemize}
        \item 
        Ezen sorokat egy $C\in\mathbb{F}^{r_w\times m}$ mátrixba téve 
        -- a maximalitás miatt -- létezik olyan $B\in\mathbb{F}^{n\times r_w}$ mátrix, 
        amelyre $A=BC$, azaz $r\leq r_w$.
        \item
        Most tekintsünk egy tetszőleges olyan
        \(
            A=BC
        \)
        felbontást, 
        ahol $B\in\mathbb{F}^{n\times r}$ és $C\in\mathbb{F}^{r\times m}$.
        Jelölje most $V$ a $C$ mátrix sorai lineáris burkát. 
        Az $A$ mátrix fent rögzített maximális lineárisan független sorrendszere egy lineárisan független rendszer e
        $V$ vektortérben,
        és $C$ sorai pedig egy generátorrendszert alkotnak ugyanebben a $V$ vektortérben.
        A 2. állitás szerint $r_w\leq r$.
    \end{itemize}
    Evvel azt is megmutattuk, 
    hogy bármely két maximális lineárisan független sorrendszer azonos elemszámú, 
    és számuk megegyezik a mátrix feszítőrangjával.
\end{proof}
\begin{proof}[3.\Rightarrow 4.]
    Tegyük fel, hogy $AB=I$.
    Az identitás mátrix rangja nyilván $n$.
    A 3. állitás miatt a feszítőrang is $n$.
    Ha $A$ oszlopai nem lennének lineárisan függetlenek,
    akkor lenne $A=CD$ felbontás, ahol $C\in\mathbb{F}^{n \times r}$ és $D\in\mathbb{F}^{r\times n}$ valamely $r<n$ mellett.
    Ekkor persze $I=AB=\left( CD \right)B=C\left( DB \right)$ is teljesülne, 
    ahol $DB\in\mathbb{F}^{r\times n}$ ellentmondva az identitás mátrix feszítőrangja definíciójának.
    Világos, hogy
    \[
        A\left( BA-I \right)=
        A\left( BA \right)-AI=
        \left( AB \right)A-AI=IA-AI=0.
    \]
    Figyelembe véve, hogy $A$ oszlopai lineáris függetlenek, ez csak úgy lehetséges, ha $BA-I$ a zéró mátrix, ergo $BA=I$.
\end{proof}
\begin{proof}[4.\Rightarrow 1.]
    Legyen $A\in\mathbb{F}^{n\times m}$ a homogén lineáris egyenletrendszer együttható mátrixa,
    ahol $n$ az egyenletek száma, $m$ az ismeretlenek száma.
    Azt kell megmutatnunk, hogy az oszloprendszer lineárisan összefüggő.
    Ha független lenne, akkor
    egészítsük ki e mátrixot alulról $m-n$ darab csupa nullákat tartalmazó sorral.
    Mivel $m>n$ ezért a kiegészített $Q\in\mathbb{F}^{m\times m}$ mátrix legalsó sora csak nullát tartalmaz.
    Mivel $A$ oszlopai lineárisan függetlenek, ezért $Q$ oszlopai is azok.
    Emiatt létezik $P\in\mathbb{F}^{m\times m}$ mátrix, amelyre $PQ=I$.
    A 3. állitás szerint $I=QP$ is teljesül, 
    ami azt jelenti, hogy $I$ utolsó sora a csupa nullákat tartalmaz, ami ellentmondás.
    Beláttuk tehát, hogy $A$ oszlopai lineárisan összefüggőek, azaz az eredeti egyenletrendszernek van nemtriviális megoldása.
\end{proof}

%%\part{E-dúr hegedűverseny No. 1, Op. 8, RV 269, ,,La primavera''}
%% a vége következik
\backmatter
\pagestyle{empty}
\bibliography{la.bib}
\printindex

\end{document}
% arara: latexmk: { 
% arara: --> engine: lualatex,
% arara: --> options: [ '-pvc' ]
% arara: --> }

\begin{thebibliography}{MMMMMMMMM}
    \bibitem[Dancs István, Puskás Csaba (2006)]{PCSDI}
        Dancs István, Puskás Csaba: \textit{Vektorterek}, Aula kiadó 2001, Budapest, ISBN:963 9345 53 9, BCE Catalogue: bcek.379187.
\end{thebibliography}

Created: Sat 20 Jul 2019 05:02:12 AM CEST
Last Modified : Sat 14 Sep 2019 08:28:02 AM CEST
