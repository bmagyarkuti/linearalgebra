\documentclass[a4paper]{memoir}
\let\Aref\relax
\usepackage[x11names]{xcolor}

\usepackage{polyglossia,amsmath,amsthm,fixme,natbib}\setdefaultlanguage{magyar}\frenchspacing
\fxsetup{status=draft, theme=color, layout={inline}}
\renewcommand{\fixmelogo}{\textcolor{black}{\colorbox{Firebrick1}{\textsf{\textbf{FIX}}}}}

\usepackage[unicode]{hyperref}\hypersetup{final}\usepackage{memhfixc}

\usepackage{fontspec}
\defaultfontfeatures{Ligatures=TeX}
\setmainfont{TeX Gyre Pagella}
\setsansfont{Kurier}[Scale=MatchLowercase]
\setmonofont{inconsolata}[Scale=MatchLowercase]

\usepackage[a4paper]{geometry}
\usepackage[missing={(Oh my!)},dirty={Oh no!},mark,marknotags]{gitinfo2}

\setsecheadstyle{\Large\sffamily\bfseries\raggedright}
\setsubsecheadstyle{\large\sffamily\bfseries\raggedright}
\setsubsubsecheadstyle{\sffamily\bfseries\raggedright}
\makeatletter
\newcommand\arraybslash{\let\\\@arraycr}
\patchcmd{\@makechapterhead}
 {\printchaptername \chapternamenum \printchapternum}
 {\printchapternum.\@\chapternamenum \printchaptername}
 {}{}
\makeatother

\def\scwords #1 #2 #3 {\textsc{#1} \textsc{#2} \textsc{#3} }
\citeindextrue
\makeindex

\begin{document}
\chapter*{Előszó}
\scwords Hogy nekem mennyi bajom van.
A legfontosabb forrás \citep{PCSDI}.

\noindent MGy


\noindent Budapest, \today
\tableofcontents*
\chapter{Gauss-Jordan elimináció}
\scwords Lineáris egyenletrendszerek megoldását automatizáljuk.\index{lineáris egyenletrenszer}

\begin{thebibliography}{MMMMMMMMM}
    \bibitem[Dancs István, Puskás Csaba (2006)]{PCSDI}
        Dancs István, Puskás Csaba: \textit{Vektorterek}, Aula kiadó 2001, Budapest, ISBN:963 9345 53 9, BCE Catalogue: bcek.379187.
\end{thebibliography}
\printindex
\end{document}
Created: Sat 20 Jul 2019 05:02:12 AM CEST
Last Modified : Sat 20 Jul 2019 01:38:43 PM CEST
