\documentclass[9pt,showtrims]{memoir}
\let\Aref\relax
\usepackage[x11names]{xcolor}
%%%%%%% pdflatex %%%%%%%%%%
%\usepackage[T1]{fontenc}
%\usepackage[utf8]{inputenc}
%\usepackage[hungarian]{babel}[2015/11/24]
%%%%%%% pdflatex %%%%%%%%%%

%%%%%% lualatex %%%%%%%%%%%
%\usepackage{polyglossia}\setdefaultlanguage{magyar}
\usepackage[hungarian]{babel}[2015/11/24]
\usepackage{fontspec}
\defaultfontfeatures{Ligatures=TeX}
\setmainfont{TeX Gyre Pagella}
\setsansfont{Kurier}[Scale=MatchLowercase]
\setmonofont{inconsolata}[Scale=MatchLowercase]
%%%%%% lualatex %%%%%%%%%%%

\frenchspacing
\usepackage{amsmath,amsthm,amsfonts,amssymb,fixme}
\usepackage{natbib}\bibliographystyle{natdin-es}
\fxsetup{status=draft, theme=color, layout={inline}}
\renewcommand{\fixmelogo}{\textcolor{black}{\colorbox{Firebrick1}{\textsf{\textbf{FIX}}}}}

\usepackage[unicode]{hyperref}\hypersetup{final}\usepackage{memhfixc}


\usepackage[a4paper]{geometry}
%\usepackage[missing={(Oh my!)},dirty={Oh no!},mark]{gitinfo2}
\usepackage[mark,dirty={(Dirty)}]{gitinfo2}
\renewcommand{\gitMark}{References: \gitReferences\,@\,\gitFirstTagDescribe{}
    \textbullet{}
    Date: \gitAuthorIsoDate
}


%\setsecheadstyle{\Large\sffamily\bfseries\raggedright}
%\setsubsecheadstyle{\large\sffamily\bfseries\raggedright}
%\setsubsubsecheadstyle{\sffamily\bfseries\raggedright}


\nouppercaseheads
%\makeoddhead{myheadings}{\footnotesize\selectfont\sffamily\leftmark}{}{\footnotesize\selectfont\sffamily\thepage}
%\makeevenhead{myheadings}{\footnotesize\selectfont\sffamily\thepage}{}{\footnotesize\selectfont\sffamily\rightmark}
\makeoddhead{myheadings}{\footnotesize\leftmark}{}{\footnotesize\thepage}
\makeevenhead{myheadings}{\footnotesize\thepage}{\footnotesize\myBotmark}{\footnotesize\rightmark}
\makepsmarks{myheadings}{%
    \renewcommand\chaptermark[1]{%
    \markboth{%
%      \ifnum \value{secnumdepth} > 1
%      \if@mainmatter %
            \thechapter.~\chaptername:~%
%      \fi
%      \fi
        ##1}{\thepart.~\partname}}%
    }

\makeatletter
\newcommand\arraybslash{\let\\\@arraycr}
\patchcmd{\@makechapterhead}
    {\printchaptername \chapternamenum \printchapternum}
    {\printchapternum.\@\chapternamenum \printchaptername}
    {}{}
\renewenvironment{proof}[1][\proofname]
    {\par\pushQED{\qed}%
    \normalfont \topsep6\p@\@plus6\p@\relax
    \trivlist
    \item[\hskip\labelsep
        \itshape
    #1\@addpunct{:}]\ignorespaces}
    {\popQED\endtrivlist\@endpefalse}
\makeatother

\newcommand{\addQEDstyle}[2]{\AtBeginEnvironment{#1}{\pushQED{\qed}\renewcommand{\qedsymbol}{#2}}\AtEndEnvironment{#1}{\popQED}}
%% qed trükkök:
%% https://tex.stackexchange.com/questions/16453/denoting-the-end-of-example-remark
%\swapnumbers %% a magyar.ldf megfordítja. A polyglossia nem. De a magyar ldf pontot is tesz a számcimke után

\renewcommand{\qedsymbol}{$\centerdot$}
\newcommand{\myqedsymbol}{$\lrcorner$}
\theoremstyle{plain}

\newtheorem{proposition}{állítás}[chapter]
\newtheorem{lemma}[proposition]{lemma}
\newtheorem*{SL}{Steinitz-lemma}
%
\theoremstyle{remark}
\newtheorem{note}[proposition]{megjegyzés}

\theoremstyle{definition}
\newtheorem{definition}[proposition]{definíció}
\newtheorem{corollary}[proposition]{következmény}
\newtheorem{defprop}[proposition]{definíció-állítás}
\addQEDstyle{definition}{\myqedsymbol}\addQEDstyle{proposition}{\myqedsymbol}\addQEDstyle{lemma}{\myqedsymbol}\addQEDstyle{note}{\myqedsymbol}\addQEDstyle{corollary}{\myqedsymbol}\addQEDstyle{SL}{\myqedsymbol}\addQEDstyle{defprop}{\myqedsymbol}


%%% https://tex.stackexchange.com/questions/319474/put-current-theorem-like-items-name-number-in-header
% \myBotmark feltöltése a lapon lévő utolsó tétel környezettel
\makeatletter
    \@ifdefinable\@my@claim@mark{\newmarks\@my@claim@mark}
    \newcommand*\myMark[1]{\marks\@my@claim@mark{#1}}
    \newcommand*\myBotmark{\botmarks\@my@claim@mark}
    \patchcmd{\@begintheorem}{% search for:
        \thm@swap\swappedhead\thmhead % more specific than before
    }{% replace with:
        \myMark{#2.\@ifnotempty{#1}{\ #1}\@ifnotempty{#3}{\ (#3)}}%
        \thm@swap\swappedhead\thmhead
    }{
        \typeout{>>> Made patch specific for amsthm.}
    }{
        \typeout{>>> Patch specific for amsthm FAILED!}
    }

 %part
\long\def\@part[#1]#2{%
  \M@gettitle{#1}%
  \def\f@rtoc{#1}%
  \@nameuse{part@f@rtoc@before@write@hook}%
  \phantomsection
  \mempreaddparttotochook
  \ifnum \c@secnumdepth >-2\relax
    \refstepcounter{part}%
    \addcontentsline{toc}{part}%
      {\protect\partnumberline{\thepart}\f@rtoc}%
    \mempartinfo{\thepart}{\f@rtoc}{#2}%
  \else
    \addcontentsline{toc}{part}{\f@rtoc}%
    \mempartinfo{}{\f@rtoc}{#2}%
  \fi
  \mempostaddparttotochook
  \partmark{#1}%
  {\centering
   \interlinepenalty \@M
   \parskip\z@
   \normalfont
   \ifnum \c@secnumdepth >-2\relax
     \printpartnum.\ \printpartname \partnamenum%%MGy
     \midpartskip
   \fi
   \printparttitle{#2}\par}%
  \@endpart}
\makeatother

\renewcommand{\mathbf}{\mathbb}
\DeclareMathOperator{\lin}{lin}
\DeclareMathOperator{\crank}{crank}
\DeclareMathOperator{\rrank}{rrank}
\DeclareMathOperator{\srank}{srank}
\DeclareMathOperator{\rank}{rank}
%\DeclareMathOperator{\dim}{dim}
\def\scwords #1 #2 #3 {\textsc{#1} \textsc{#2} \textsc{#3} }
\citeindextrue
\makeindex
\synctex=1
\begin{document}
\frontmatter*
\section*{Verzió információk}
\begin{center}
\begin{tabular}{l|r}
    \hline
References & \gitReferences\\
Branch & \gitBranch\\
Dirty & \gitDirty\\
Hash&\gitHash\\
Author Iso Date & \gitAuthorIsoDate\\
\hline
First Tag Describe & \gitFirstTagDescribe \\
Reln& \gitReln \\
Roff & \gitRoff \\
Tags & \gitTags \\
Describe & \gitDescribe \\
\hline
\end{tabular}
\end{center}
\chapter*{Előszó}
\scwords%
A legfontosabb forrás \citep{DancsPuskas2001}.

\dots

Igyekszem struktúráltan írni.
Ennek oka, hogy evvel hangsúlyozzam, hogy az olvasónak igyekeznie kell struktúráltan gondolkodni.
A hátulütője, hogy hibásan azt a helytelen képzetet keltheti, 
Nem, nem erről van szó. 
A puzzle-ban minden elem egyenrangú, az egyik elem hiánya éppen annyira fájdalmas mint a másiké.
Itt nem erről van szó, ez egyetlen matematikai diszciplína esetében sem igaz!
Az olvasónak igyekeznie kell, 
hogy meglássa mi a legfontosabb gondolat a sok-sok állításnak, 
mint építménynek egy-egy ,,nyilvánvaló következményében''.
Kicsi, atomszerű építőkövek egymásutáni megértése visz az anyagban előre, 
ezek az egymástól feltűnő módon szeparált állítások és azok érvekkel való alátámasztása,
amit bizonyításnak is szokás mondani.
Hogy e kis lépések egymástól még határozottabban válljanak el azt az írás
typográfiája is erősíti azzal, 
az állítás-szerű környeteket a \,\myqedsymbol~, 
és a bizonyítás környezetet a \,\qedsymbol~ karakterekkel zárom le.


\bigskip\noindent 
Magyarkuti Gyula
\hfill{Budapest, \ontoday}


\clearpage
\tableofcontents*
\pagestyle{myheadings}
\mainmatter*
\part{F-dúr hegedűverseny No. 3, Op. 8, RV 293, ,,L'autunno''}
\chapter{Előzmények}
\section{Algebrai struktúrák}
\begin{definition}[$n$-változós művelet]\index{művelet}\index{algebrai struktúra}
    Legyen $H$ egy halmaz. Egy 
    \[
        \varphi\colon H^n\to H
    \]
    függvényt $n$-változós \emph{műveletnek} nevezünk.
    Egy halmazt és rajta véges sok műveletet együtt \emph{algebrai struktúrának} mondunk.
    Jelölés: 
    $$\left(H,\varphi_1,\dots,\varphi_n  \right),$$ 
    ahol $H$ a halmaz és
    $\varphi_1,\dots,\varphi_n$ a $H$ halmazon értelmezett műveletek.
\end{definition}
\begin{definition}[félcsoport]\index{félcsoport}
    Egy $\left( S,\star \right)$ algebrai struktúrát \emph{félcsoportnak} mondjuk,
    ha $\star$ egy kétváltozós \emph{asszociatív}\index{asszociatív}
    művelete az $S$ halmaznak,
    azaz minden $a,b,c\in S$ mellett
    \[
        a\star\left( b\star c \right)=\left( a\star b\right)\star c.\qedhere
    \]
\end{definition}
Lefordítva ez azt jelenti, hogy 
\begin{enumerate}
    \item minden $a,b\in S$ mellett $a\star b\in S$, és
    \item minden $a,b,c\in S$ esetén $a\star\left( b\star c \right)=a\star\left( b\star c \right)$
\end{enumerate}
\begin{definition}[neutrális elem]\index{neutrális elem}\index{neutrális elemes félcsoport}
    Az $\left( S,\star \right)$ félcsoportban az $s\in S$ elem \emph{balról (jobbról) neutrális},
    ha $s\star t=t$ ($t\star s=t$) minden $t\in S$ mellett.
    Ha $s\in S$ balról is és jobbról is neutrális, akkor $s$-et egy \emph{neutrális elemnek}
    mondjuk.
    A félcsoportot \emph{neutrális elemes félcsoportnak} nevezzük, ha van benne neutrális elem.
\end{definition}
\begin{proposition}
    Ha egy félcsoportban, van egy balról neutrális elem és egy jobbról neutrális elem, 
    akkor ezek megegyeznek. 
    Emiatt egy neutrális elemes félcsoportban neutrális elem csak egy van.
\end{proposition}
\begin{proof}
    Legyen $s_1$ balról-- és $s_2$ jobbról neutrális elem.
    Ekkor
    \(
        s_1=s_1\star s_2=s_2.
    \)
\end{proof}
A félcsoport additív írásmódja esetén természetes a neutrális elemet \emph{zérusnak},
míg multiplikatív írásmód esetén \emph{egységnek} nevezni.
\begin{definition}[csoport]\index{csoport}
    Egy $\left( G,\star \right)$ algebrai struktúrát \emph{csoportnak} nevezünk,
    ha neutrális elemes félcsoport, amlyben minden $g\in G$-hez létezik $g'\in G$, hogy
    \[
        g\star g'=e=g'\star g.\tag{\dag}
    \]
    Itt $e\in G$ jelöli a $G$ csoport neutrális elemét.
\end{definition}
\begin{defprop}[inverz elem]\index{inverz}
    Legyen $\left( G,\star \right)$ egy csoport.
    Ekkor minden $g\in G$-hez, csak egyetlen $g'\in G$ létezik, 
    amelyre a fenti ($\dag$) azonosság fennáll.
    Adott $g$-hez ezt ez egyetlen $g'\in G$ elemet, 
    amelyre ($\dag$) teljesül a $g$ elem \emph{inverzének} mondjuk.
\end{defprop}
\begin{proof}
    Tegyük fel, hogy $g',g''$ inverz elemei $g$-nek.
    Azt mutatjuk meg, hogy ha $g'$ baloldali-- és $g''$ jobboldali inverze $g$-nek,
    akkor a két elem megegyezik:
    \[
        g'=g'\star e=
        g'\star\left( g\star g'' \right)=
        \left(g'\star g\right)\star g'' =
        e\star g''=g''.\qedhere
    \]
\end{proof}
Példaként gondoljuk meg, hogy a $H\to H$ függvény halmaza a kompozíció művelettel
neutrális elemes félcsoport, és a $H\to H$ kölcsönösen egyértelmű függvények halmaza a kompozíció művelettel csoportot alkotnak. 
Ez utóbbi csoportot mondjuk \emph{permutáció csoportnak}\index{permutációk}.
\begin{proposition}[egyszerűsítési szabály]\index{egyszerűsítési szabály}
    Csoportban igaz az egyszerűsítési szabály, azaz
    \[
        a\star c=b\star c\implies a=b.\qedhere
    \]
\end{proposition}
\begin{proof}
    \begin{math}
        a=a\star e
        =
        a\star \left( c\star c'\right)=
        \left( a\star c \right)\star c'=
        \left( b\star c \right)\star c'=
        b\star\left( c\star c' \right)=
        b\star e=
        b.
    \end{math}
\end{proof}
\begin{definition}[Abel--csoport]\index{Abel--csoport}
    Egy $\left( G,\star \right)$ csoportot \emph{Abel--csoportnak} nevezünk,
    ha a művelete \emph{kommutatív}\index{kommutatív} is, 
    azaz minden $s,t\in G$ mellett $s\star t=t\star s$.
\end{definition}
\begin{definition}[gyűrű]\index{gyűrű}
    A kétműveletes $\left( R,+,\cdot \right)$ algebrai struktúrát \emph{gyűrűnek} nevezzük,
    ha
    \begin{enumerate}
        \item $\left( R,+ \right)$ Abel--csoport;
        \item $\left( R,\cdot \right)$ félcsoport;
        \item és a két műveletet összeköti a következő két disztributivitás:\index{disztributív}
            \[
                a\cdot\left( b+c \right)=a\cdot b + a\cdot c\qquad
                \left( a\cdot b \right)\cdot c=a\cdot c+a\cdot b.
            \]
    \end{enumerate}
    Ha $\left( R,\cdot \right)$ neutrális elemes félcsoport, akkor azt mondjuk, hogy $R$ egy
    \emph{egységelemes gyűrű}, és ha $\left( R,\cdot \right)$ kommutatív félcsoport, akkor
    azt mondjuk, hogy $R$ egy \emph{kommutatív gyűrű}.
\end{definition}
\begin{definition}[test]
    Egy $\left( \mathbb{F},+,\cdot \right)$ kétműveletes algebrai struktúrát \emph{testnek}
    nevezünk, 
    ha olyan kommutatív egységelemes gyűrű, 
    amelyben minden nemzérus\footnote{értsd: minden elemnek, amely a $+$ műveletre nézve neutrális elemtől különbözik}
    elemnek van inverze\footnote{értsd: a $\cdot$ szorzás neutrális elemére mint egységelemre nézve}, 
    és $0\neq 1$\footnote{érts: az összeadásra nézve és a szorzásra nézve képzett neutrális elemek nem azonosak.}.
\end{definition}
A test az elgebrai struktúra, ahol a az összeadás és sorzás műveletekkel úgy számolhatunk, mint amit a valós számok során megszoktuk.
Példaként néhány tulajdonság.
\begin{proposition}
    Az $\left( R,+,\cdot \right)$ gyűrűben minden $a\in R$ mellett 
    \begin{equation*}
        a\cdot 0=0\text{ és }
        \left( -1 \right)a=-a.\qedhere
    \end{equation*}
\end{proposition}
\begin{proof}
    \begin{math}
        0+a\cdot 0=
        a\cdot 0=
        a\left( 0+0 \right)=
        a\cdot 0+a\cdot 0.
    \end{math}
    A jobboldali $a0$-val való egszerűsítés után kapjuk, 
    hogy $a=a\cdot0$.
    A második azonosságot az első felhasználásával kapjuk:
    \begin{math}
        0
        =
        0a
        =
        \left( 1+\left( -1 \right) \right)\cdot a
        =
        1\cdot a + \left( -1 \right)\cdot a
        =
        a +\left( -1 \right)a.
    \end{math}
    Az additív inverz definíciója szerint ez éppen azt jelenti, hogy $-a=\left( -1 \right)\cdot a$.
\end{proof}
Ami nagyon fontos, hogy egy gyűrűben nem feltétlen teljesül, 
hogy elemek szorzata csak úgy lehet zérus, ha legalább az egyik elem zérus.
Számunkra a legfontosabb példa  a mátrixok gyűrűje\footnote{lásd kicsit később},
ahol pont ennek a hiánya jelenti nehézséget.

Egy testben ilyen nem fordulhat elő.
\begin{definition}[nullosztómentes gyűrű]
    Egy gyűrűt \emph{nullosztómentesnek} nevezzük, 
    ha két elem szorzata csak úgy lehet nulla, 
    ha legalább az egyik elem nulla.
\end{definition}
\begin{proposition}
    Egy test egyben nullosztómentes gyűrű, azaz
    ha $\mathbf{F}$ egy test, és $a,b\in\mathbf{F}$.
    Akkor 
    \[
        ab=0\implies a=0\text{ vagy }b=0.\qedhere
    \]
\end{proposition}
\begin{proof}
    Tegyük fel, hogy $ab=0$.
    Ha $b\neq 0$, akkor létezik $b'\in\mathbf{F}$, hogy $bb'=1$.
    Így 
    \[
        0= 0b'=\left( ab \right)b'=a\left( bb' \right)=a1=a.\qedhere
    \]
\end{proof}
\begin{proposition}
    Nullosztómentes gyűrűben nem zérus elemmel való szorzatot egyszerűsíteni lehet
    azaz, ha $a,b,c\in R,b\neq 0$ esetén
    \[
        ab=cb\implies a=c.\qedhere
    \]
\end{proposition}
\begin{proof}
    \begin{math}
        \left( a-c \right)b=ab-cb=0\implies a-c=0.
    \end{math}
\end{proof}
A legfontosabb struktúrák számunkra a következők:
\begin{itemize}
    \item 
    Egységelemes gyűrű, amelyben a nullosztómentesség nem teljesül: mátrixok.
    \item
    Kommutatív egységelemes gyűrű, amely nullosztómentes de mégsem test: polinomok.
    \item
    Test:
    a valós vagy a komplex számok.
\end{itemize}
\section{Polinomgyűrűk}
\begin{definition}[polinom]\index{polinom}
    Legyen $\mathbf{F}$ egy test.
    E test feletti polinomokon az összes 
    \[
        p\left( t \right)=
        \alpha_0+\alpha_1t+\alpha_2t^2+\dots+\alpha_nt^n
    \]
    alakú formális algebrai kifejezést értjük.

    Itt $n$ tetszőleges nem negatív egész 
    és $\alpha_0,\dots,\alpha_n$ tetszőleges, az $\mathbf{F}$ test beli elemek.

    Az $\mathbf{F}$ test felletti összes polinomok halmazát $\mathbf{F}\left[ t \right]$ módon jelöljük.
\end{definition}
A fenti definícióban az \emph{algebrai kifejezés} szó arra utal,  hogy az
\begin{math}
        \alpha_0+\alpha_1t+\alpha_2t^2+\dots+\alpha_nt^n
\end{math}
műveletek minden $t\in\mathbf{F}$ mellett értelmesek, 
és eredményük egy újabb $\mathbf{F}$ test beli elem.
Ezt az elemet mondjuk a $p$ polinom helyettesítési értékének.

A \emph{formális algebrai kifejezés}\index{formális algebrai kifejezés} arra utal, 
hogy egy polinomot az együtthatói határozzák meg, 
azaz két polinom akkor és csak akkor azonos,
ha az összes együtthatói azonosak.
Ez szemben áll avval, hogy ha a polinomokra mint függvényekre tekintenénk, 
akkor a helyettesítési értékek egyenlősége jelentené két polinom azonos voltát.
A formális szó tehát azt jelenti, hogy nem mint függvényre gondolunk, 
hanem egyszerűen adott $\alpha_0,\dots,\alpha_n$ rögzített elemek -- ezeket mondjuk együtthatóknak --,
által előírt műveletekre. 
Az az előírás ugyanis, hogy tetszőleges $t\in\mathbf{F}$ mellett hajtsuk végre az
\[
        \alpha_0+\alpha_1t+\alpha_2t^2+\dots+\alpha_nt^n
\]
műveletsort. 
A műveletsorrol és nem annak eredményéről van szó. 
Két műveletsor akkor azonos, ha ugyanazok a műveletsort meghatározó 
$\left( \alpha_0,\alpha_1,\dots,\alpha_n \right)$%
\footnote{Az előbbi zárójellel azt hangsúlyozzuk, hogy az együtthatók sorrendje is számít.}
együtthatók.%
\footnote{Persze felmerül a kérdés, 
    hogy ha két polinom minden helyettesítési értéke azonos,
    akkor igaz-e, 
    hogy mint formális polinomok is azonosak,
    tehát a két polinom együtthatói is rendre azonosak-e?
    A pozitív választ később látjuk valós vagy komplex szám test feletti polinomok esetén.%
}

\begin{definition}[polinom foka]\index{polinom foka}
    Legyen $p\left( t \right)\in\mathbf{F}\left[ t \right]$ egy polinom.
    Azt mondjuk, hogy a $n$ nem negatív egész szám e \emph{polinom fokszáma},
    ha $n$ a legnagyobb nemzérus együttható indexe.

    A $p\left( t \right)=0$ konstans zérus polinom esetén a fok legyen $-\infty$.
    A $p$ polinom fokszámát $\deg p$ módon jelöljük.
\end{definition}
Látni fogjuk, hogy a konstans zérus polinomra $\deg p=-\infty$ csak egy kényelmes jelölés.
Időnként a polinom fokszámával műveleteket is végzünk.
Megegyezés szerint ilyenkor $-\infty+a=-\infty$ minden $a$ nem negatív egész számra, 
és $\left( -\infty \right)+\left( -\infty \right)=-\infty.$
A $-\infty$ szimbólumot minden egész számnál határozottan kisebbnek gondoljuk.

Két polinom összegét és szorzatát a szokásos módon definiáljuk:
\begin{definition}
    Legyen $p,q\in\mathbf{F}[t]$, két polinom.
    \[
        p\left( t \right)
        =
        \sum_{j=0}^n\alpha_jt^j
        \text{ és }
        q\left( t \right)
        =
        \sum_{j=0}^m\beta_jt^j,
        \qquad
        \alpha_j,\beta_j\in\mathbf{F}, 
        0\leq n,m\in\mathbf{Z}.
    \]
    Ekkor a $p$ és $q$ összegének definíciója:
    \[
        \left( p+q \right)\left( t \right)
        =
        \sum_{j=0}^{\max{\left\{ m,n \right\}}}\left( \alpha_j+\beta_j \right)t^j;
    \]
    míg a két polinom szorzatának definíciója:
    \[
        \left( pq \right)\left( t \right)
        =
        \sum_{j=0}^{n+m}c_jt^j
        \text{ ahol }
        c_j
        =
        \sum_{k=0}^j\alpha_k\beta_{j-k}.\qedhere
    \]
\end{definition}
\begin{proposition}
    Legyenek $p,q\in\mathbf{F}[t]$ polinomok az $\mathbf{F}$ test felett.
    Ekkor
    \begin{enumerate}
        \item $\deg \left( pq \right)=\deg p+\deg q$;
        \item $\deg \left( p+q \right)\leq\max\left\{ \deg p,\deg q \right\}$.\qedhere
    \end{enumerate}
\end{proposition}
\begin{proof}
    Figyeljünk arra, hogy a konstans zérus polinom esetében is működik a tétel,
    és vegyük észre, hogy az szorzat polinomra vonatkozó állítás azért igaz, 
    mert a test nullosztómentes.
\end{proof}
\begin{proposition}
    Egy $\mathbf{F}$ test feletti $\mathbf{F}\left[ t \right]$ formális polinomok 
    a fent bevezetett összeadás és szorzás műveletekkel,
    nullosztómentes,
    kommutatív, egységelemes gyűrűt alkotnak.
\end{proposition}

\section{Polinomok oszthatósága és a maradékos osztás}
A következő állítás szerint a polinomok közt is működik a maradékos osztás,
ahogyan azt az egész számok közt megszoktuk.
\begin{proposition}[maradékos osztás]
    Legyen $p,q\in\mathbf{F}\left[ t \right]$ polinomok, $p\neq 0$.
    Ekkor létezik egyetlen $h,r\in\mathbf{F}\left[ t \right]$ polinom, amelyre
    \[
        p\left( t \right)
        =
        h\left( t \right)q\left( t \right)+r\left( t \right)
        \quad
        \deg r < \deg q.\qedhere
    \]
\end{proposition}
\begin{proof}
    Először is azt vegyük észre, hogy $\deg p<\deg q$ esetben $r=p$, 
    $h=0$ szereposztással készen is vagyunk.

    Tegyük fel tehát, hogy $n=\deg p\geq \deg q=m$, és lássuk be az állítást $n$ szerinti indukcióval.
    Ha $n=0$, akkor $p\left( t \right)=\alpha_0$ és $q\left( t \right)=\beta_0\neq 0$.
    Ekkor persze
    \[
        \alpha_0=\frac{\alpha_0}{\beta_0}\beta_0+0,
    \]
    ami azt jelenti, hogy $h\left( t \right)=\frac{\alpha_0}{\beta_0}$ és $r\left( t \right)=0$ szereposztás
    megfelelő.

    Most tegyük fel, 
    hogy igaz az állítás $n+1$-nél kisebb fokú $p$ polinomokra ($n\geq 0$),
    és lássuk be egy pontosan $n+1$-ed fokú polinomra.
    Legyen tehát
    \[
        p\left( t \right)=\alpha_{n+1}t^{n+1}+\dots+\alpha_0
        \quad\text{ és }\quad
        q\left( t \right)=\beta_{m}t^m+\dots+\beta_0,
    \]
    ahol $m\leq n+1$.
    Tekintsük a következő polinomot:
    \[
        \frac{\alpha_{n+1}}{\beta_m}t^{n+1-m}q\left( t \right).
    \]
    Világos, hogy ennek főegyütthatója éppen $\alpha_{n+1}$ és foka éppen $n+1=\deg p$.
    Így a 
    \[
        p_1\left( t \right)
        =
        p\left( t \right)-
        \frac{\alpha_{n+1}}{\beta_m}t^{n+1-m}q\left( t \right).
    \]
    polinomra $\deg p_1<\deg p$.
    Na most, ha $\deg p_1<\deg q$, akkor a bizonyítás első mondatában említett helyzetben vagyunk,
    tehát nyilvánvaló szereposztással az állítás igaz $p_1$-re és $q$-ra.
    Ha viszont $\deg p_1\geq \deg q$ még mindig igaz, akkor az indukciós feltétel szerint található
    $h,r\in\mathbf{F}\left[ t \right]$ polinom, amelyre igaz az állítás.
    Mindkét esetben találtunk tehát $h,r$ polinomokat, amelyre
    \[
        p\left( t \right)-
        \frac{\alpha_{n+1}}{\beta_m}t^{n+1-m}q\left( t \right)
        =
        p_1\left( t \right)
        =
        h\left( t \right)q\left( t \right)+r\left( t \right);
        \quad
        \deg r < \deg q
    \]
    teljesül.
    Ekkor persze
    \[
        p\left( t \right)
        =
        \left( h\left( t \right)
        +
        \frac{\alpha_{n+1}}{\beta_m}t^{n+1-m}
        \right)
        q\left( t \right)
        +
        r\left( t \right);
        \quad
        \deg r < \deg q
    \]
    is fennáll. Ezt kellett belátni az állítás egzisztencia részéhez.
    
    Az unicitás részhez tegyük fel, hogy valamely $h,h_1,r,r_1$ polinomokra
    \[
        h\left( t \right)q\left( t \right)+r\left( t \right)
        =
        p\left( t \right)
        =
        h_1\left( t \right)q\left( t \right)+r_1\left( t \right)
    \]
    teljesül, ahol $\deg r<\deg q$ és $\deg r_1<\deg q$.
    Persze átrendezve ekkor
    \[
        \left( h\left( t \right)-h_1\left( t \right) \right)q\left( t \right)
        =
        r_1\left( t \right)-r\left( t \right)
    \]
    is fennáll.
    Ekkor a fokszámokra figyelve
    \[
        \deg\left( h-h_1 \right)+\deg q 
        = 
        \deg\left( r_1-r \right)
        \leq
        \max\left\{ \deg r_1,\deg (-r) \right\}
        <
        \deg q.
    \]
    Ez csak akkor lehetséges, ha $\deg\left( h-h_1 \right)=-\infty$,
    ami azt jelenti, hogy $h=h_1$, 
    amiből persze $r_1=r$ már látszik is.
\end{proof}

\section{Euklideszi algoritmus}

\chapter{Gauss-Jordan elimináció}
\scwords Lineáris egyenletrendszerek megoldását automatizáljuk.\index{lineáris egyenletrenszer}


\chapter{Vektortér fogalma}
\chapter{Altér}
\chapter{Lineárisan független rendszerek és generátorrendszerek}
\begin{definition}[lineárisan összefüggő vektorrendszer]\index{lineárisan összefüggő rendszer}
    Egy véges $\left\{ y_1,\dots,y_n \right\}$ vektorrendszert \emph{lineárisan összefüggőnek}
    mondunk, ha egyik vektora kifejezhető a többi vektor lineáris kombinációjaként.
\end{definition}
Úgy is fogalmazhatnánk, hogy az $\left\{ y_1,\dots,y_n \right\}$ rendszer pontosan akkor
lineárisan összefüggő, ha létezik $1\leq k\leq n$ index, amelyre
\[
    y_k\in\lin\left\{ y_1,\dots,y_{k-1},y_{k+1},\dots,y_n \right\}.
\]
\begin{proposition}
    Legyen $\left\{ y_1,\dots,y_n \right\}$ vektorrendszer rögzítve a $V$ vektortérben.
    A vektorrendszerre tett alábbi feltevések egymással ekvivalensek.
    \begin{enumerate}
        \item Lineárisan összefüggő;
        \item Van olyan elem a vektortérben, amely nem csak egyféleképpen áll elő mint az $y_1,\dots,y_n$
            vektorok lineáriskombinációja,\\
            azaz formálisabban:
            létezik z\in V, amelyre $z=\sum_{j=1}^n\xi_jy_j$ és $z=\sum_{j=1}^n\eta_jy_j$
            és létezik $1\leq k\leq n$, amelyre $\xi_k\neq\eta_k$.
        \item Vannak olyan nem mind zérus $\alpha_1,\dots,\alpha_n$ skalárok, amellyekkel
            \[
                \sum_{j=1}^n\alpha_jy_j=0.\qedhere
            \]
    \end{enumerate}
\end{proposition}
\begin{proof}
    Körben bizonyítunk.
    \begin{description}
        \item[$1.\Rightarrow 2.$] 
            Tegyük fel, hogy $y_k=\sum_{j=1}^{k-1}\eta_jy_j+\sum_{j=k+1}^n\eta_jy_j$.
            Ekkor az alábbi együtthatórendszerek
            \[
                \left( \alpha_1,\dots,\alpha_{k-1},0,\alpha_{k+1},\dots,\alpha_n \right)
                \qquad
                \left( 0,\dots,0,1,0,\dots,0 \right)
            \]
            a $k$-adik helyen biztosan különböznek, 
            hiszen $0\neq 1$, 
            és mind a két együtthatórendszerrel képzett lineáris kombináció ugyanazt az $y_k$ vektort eredményezi.
        \item[$2.\Rightarrow 3.$]
            Világos, hogy 
            \[
                0=z-z=
                \sum_{j=1}^n\left( \xi_j-\eta_j \right)y_j
            \]
            és a $k$-adik skalár nem zérus.
        \item[$3.\Rightarrow 1.$]
            Tegyük fel most, hogy 
            \(
            \sum_{j=1}^n\alpha_jy_j=0,
            \)
            és, hogy $\alpha_k\neq 0.$
            Ekkor 
            \[
                y_k=\sum_{j=1,j\neq k}^n-\frac{1}{\alpha_k}\alpha_jy_j
            \]
            azaz a $k$-adik vektor tekinthető mint a többi vektor valamely lineáris kombinációja.
    \end{description}
    Ezt kellett belátni. 
\end{proof}
Fontos észrevétel a következő.
\begin{proposition}
    Minden, 
    valamely lineárisan összefüggő vektorrendszert tartalmazó vektorrendszer maga is lineárisan összefüggő.
\end{proposition}
\begin{definition}
    Egy nem feltétetlen véges vektorrendszert lineárisan összefüggőnek nevezünk,
    ha van véges részrendszere, amely lineárisan összefüggő.
\end{definition}
\begin{definition}[lineárisan független vektorrendszer]\index{lineárisan független rendszer}
    Egy vektorrendszer \emph{lineárisan független}, ha nem lineárisan összefüggő.
\end{definition}
Így egy nem véges vektorrendszer akkor lineárisan független, ha minden véges részrendszere is az.
Egy véges vektorrendszer lineárisan függetlenségét, pedig a következő egymással ekvivalens állítások karakterizálják.
\begin{proposition}
    Legyen $\left\{ y_1,\dots,y_n \right\}$ vektorrendszer rögzítve a $V$ vektortérben.
    A vektorrendszerre tett alábbi feltevések egymással ekvivalensek.
    \begin{enumerate}
        \item Lineárisan független;
        \item A vektorrendszer lineáris burkában minden elem egyetlen egy féle képpen áll elő,
            mint az $y_1,\dots,y_n$ vektorok lineáris kombinációja.
        \item Az $y_1,\dots,y_n$ vektoroknak csak a triviális lináris kombinációja zérus,
            azaz
            \[
                \sum_{j=1}^n\alpha_jy_j=0\text{ esetén }\alpha_1=\alpha_2=\dots=\alpha_n=0.\qedhere
            \]
    \end{enumerate}
\end{proposition}
\begin{proof}
    Nyilvánvaló a lineáris összefüggés karakterizációjából.
\end{proof}
\begin{definition}[maximális lineárisan független-- és minimális generátorrendszer]\index{maximális lineárisan független rendszer}\index{minimális generátorrendszer}
    Egy lineárisan független rendszert \emph{maximális lineárisan független rendszernek} nevezünk,
    ha nem lehet bővíteni úgy, hogy lineárisan független maradjon.

    Egy generátorrendszert \emph{minimális generátorrendszernek} mondunk, ha nem lehet szűkíteni úgy,
    hogy generátorrendszer maradjon.
\end{definition}
\begin{proposition}
    Legyen $\left\{ x_1,\dots,x_m \right\}$ egy vektorrendszere a $V$ vektortérnek.
    Az alábbi feltevések ekvivalensek.
    \begin{enumerate}
        \item A vektorrendszer maximális lineárisan független rendszer.
        \item A vektorrendszer egyszerre lineárisan független és generátorrendszer.
        \item A vektorrendszer minimális generátorrendszer.\qedhere
    \end{enumerate}
\end{proposition}
\begin{proof}
    Az alábbi lépéseket követjük.
    \begin{itemize}
        \item[1.\Rightarrow 2.]
            Ha a vektorrendszer nem lenne generátorrendszer is,
            akkor a lineáris burkán kívül lenne egy vektor.
            Ezt a vektorrendszerhez illesztve, a vektorrendszer egy valódi lineárisan független bővítését kapjuk,
            ami ellentmond a maximalitás feltételének.
        \item[3.\Rightarrow 2.]
            Ha a vektorrendszer egyik eleme a többi elem lineáris kombinációja,
            akkor azt az elemet elhagyva is generátorrendszert kapunk,
            ami ellentmond a minimalitás feltételének.
        \item[2.\Rightarrow 1.]
            Ha nem lenne maximális a lineárisan független tulajdonságra nézve,
            akkor létezne egy vektor a lineáris burkán kívül is,
            ami ellentmond a generátorrendszer tulajdonságnak.
        \item[2.\Rightarrow 3.]
            Mivel a vektorrendszer egyik eleme, sincs a többi lineáris burkában,
            ezért egyetlen elemet sem hagyhatunk el a generátorrendszer tulajdonság megtartásával,
            ami azt jelenti, hogy ez egy minimális generátorrendszer.\qedhere
    \end{itemize}
\end{proof}
Egy a zéró vektort tartalmazó vektorrendszer persze lineárisan összefüggő, 
és egy nem zérus vektorból álló egyelemű vektorrendszer lineárisan független.
A következő állítás sokszor teszi kényelmessé a gondolatmenetünket. 
\begin{proposition}
    Legyen $\left\{ y_1,\dots,y_n \right\}$ egy olyan legalább két elemű vektorrendszer,
    amelynek első eleme nem a zérus vektor, tehát $y_1\neq 0$.
    A vektorrendszer pontosan akkor lineárisan összefüggő,
    ha létezik olyan eleme, 
    amely pusztán az előző elemek lineárisan kombinációja.

    Formálisabban: akkor és csak akkor, 
    ha 
    $\exists k\quad 2\leq k\leq n : y_k\in\lin\left\{ y_1,\dots,y_{k-1} \right\}$
\end{proposition}
\begin{proof}
    Tegyük fel, hogy a vektorrendszer lineárisan összefüggő.
    Ekkor van olyan a zérus vektort eredményező lineáris kombinációja
    \(
    \alpha_1y_1+\dots+\alpha_ny_n=0,
    \)
    ahol nem az összes együttható nulla.
    Legyen $k$ a lineáris kombinációban a legnagyobb nem nulla együttható indexe.
    Világos, hogy $k\neq 1$, 
    hiszen $y_1\neq 0$.
    Persze a $k$ feletti együtthatók mind nullák,
    emiatt
    \[
        \alpha_1y_1+\dots+\alpha_ky_k=0.
    \]
    Itt már $\alpha_k\neq 0$, tehát $y_k$ kifejezhető az előző vektorok segítségével:
    \[
        y_k=
        \frac{-1}{\alpha_k}\alpha_1y_1+\dots+\frac{-1}{\alpha_{k-1}}\alpha_{k-1}y_{k-1}.\qedhere
    \]
\end{proof}
\chapter{Vektortér bázisa}
\scwords A Steinitz-lemma fontosságát kell kiemelni.
A Steinitz-lemma legegyszerűbb megfogalmazásában azt állítja, 
hogy \emph{egy lineárisan független rendszer elemszáma, soha nem nagyobb mint egy generátorrendszer elemszáma.}
Ez a tény vezet a bázis és a dimenzió fogalmához.

A generátorrendszer cseréről szóló legelső lemmának is fontos szerepe van az itt választott felépítésben.
Egyrészt használjuk a Steinitz-lemma igazolásában, másrészt ennek segítségével tisztázzuk majd azt a kérdést,
hogy hogyan alakulnak egy vektor koordinátái, ha a bázist, tehát a vonatkoztatási rendszert változtatjuk.

\begin{lemma}[generátorrendszer csere]
    Legyen $\left\{ x_1,\dots,x_m \right\}$ egy generátorrendszere valamely vektortérnek,
    és tegyük fel, hogy valamely $y$ vektorra
    \[
        y=\sum_{j=1}^m\eta_jx_j,
    \]
    ahol $\eta_k\neq 0$ valamely $1\leq k\leq m$ mellett. 
    Ekkor $y$ becserélhető a $k$-adik helyen a generátorrendszerbe, 
    úgy hogy az generátorrendszer maradjon, azaz a
    \[
        \left\{ x_1,\dots,x_{k-1},y,x_{k+1},\dots,x_m \right\}
    \]
    vektorrendszer is generátorrendszer.
    \label{le:gencsere}
\end{lemma}
\begin{proof}
    Fejezzük ki $x_k$-t az $y$-ra feírt formulából:
    \(
    x_k=\frac{1}{\eta_k}y+\sum_{j=1,j\neq k}^m\frac{-1}{\eta_k}\eta_jx_j.
    \)
    Ha $z$ eredetileg 
    \[
        z=\sum_{j=1}^m\zeta_jx_j
    \]
    alakú, akkor $x_k$ helyére betéve, a fent kifejezett formulát és bevezetve a 
    $\delta=\frac{\zeta_k}{\eta_k}$ jelölést, azt kapjuk hogy:
    \begin{multline*}
        z=\zeta_kx_k+\sum_{j=1,j\neq k}^m\zeta_jx_j=
        \\
        =
        \zeta_k
        \left( 
        \frac{1}{\eta_k}y+\sum_{j=1,j\neq k}^m\frac{-1}{\eta_k}\eta_jx_j
        \right)
        +\sum_{j=1,j\neq k}^m\zeta_jx_j
        =
        \frac{\zeta_k}{\eta_k}y+
        \sum_{j=1,j\neq k}^m\left( \zeta_j-\frac{\zeta_k}{\eta_k}\eta_j \right)x_j=
        \\
        =\delta y+
        \sum_{j=1,j\neq k}^m\left( \zeta_j-\delta\eta_j \right)x_j.
    \end{multline*}
    Azt kaptuk tehát, hogy ha egy vektor kifejezhető az eredeti vektorrendszerből az 
    \[
        \left( \zeta_1,\dots,\zeta_m \right) 
    \]
    együtthatókkal, akkor ugyanez a vektor a módosított vektorrendszerből is kifejezhető,
    méghozzá az 
    \[
        \left( 
        \underbrace{\zeta_1-\delta\eta_1}_{1.},
        \underbrace{\zeta_2-\delta\eta_2}_{2.},
        \dots,
        \underbrace{\zeta_{k-1}-\delta\eta_{k-1}}_{k-1.},
        \underbrace{\delta}_{k.},
        \underbrace{\zeta_{k+1}-\delta\eta_{k+1}}_{k+1.},\dots,
        \underbrace{\zeta_m-\delta\eta_m}_{m.}
        \right)
    \]
    együtthatókkal.
\end{proof}
\begin{SL}\index{Steinitz}
    Tegyük fel, hogy az $\left\{ y_1,\dots,y_n \right\}$ egy lineárisan független rendszer és az
    $\left\{ x_1,\dots,x_m \right\}$ vektorrendszer egy generátor rendszer.
    Ekkor
    \begin{enumerate}
        \item $n\leq m$;
        \item Az $x_1,\dots,x_m$ vektorok alkalmas átindexelésével az
            \[
                \left\{ y_1,\dots,y_n,x_{n+1},\dots,x_m \right\}
            \]
            vektorrendszer is generátorrendszer.%
            \footnote{Úgy kell a jelöléseket érteni, hogy az $n=0$, de az $n=m$ eset is lehetséges. 
                Az $n=0$ esetben az $y$-okkal jelölt vektorok egyike sem,
                míg az $n=m$ esetben az $x$-el jelölt vektorok egyike sem szerepel az 
                \(
                \left\{ y_1,\dots,y_n,x_{n+1},\dots,x_m \right\}
                \)
            vektorrendszer elemei közt.}%
            \qedhere
    \end{enumerate}
    \label{le:Stienitz}
\end{SL}
\begin{proof}
    Legyen $k$ a legnagyobb egész a $\left\{ 0,\dots,n \right\}$ egészek közül, amelyre
    \begin{enumerate}
        \item $k\leq m$, és
        \item az $x_1,\dots,x_m$ vektorok alkalmas átindexelésével az
            \[
                \left\{ y_1,\dots,y_k,x_{k+1},\dots,x_m \right\}\tag{\dag}
            \]
            vektorrendszer is generátorrendszer.
    \end{enumerate}
    Ilyen $k$ biztosan van van, hiszen $k=0$ triviálisan jó.
    Összesen azt kell meggondolnunk, hogy $k=n$.
    Ha $k<n$ lenne, 
    \begin{itemize}
        \item 
            akkor létezne $y_{k+1}$ vektor.
            No de, ez az $y_{k+1}$ nem szerepel az $\left\{ y_1,\dots,y_k \right\}$ lineáris burkában,
            ami (\dag) generátorrendszer volta miatt csak úgy lehetséges, hogy $k<m$, 
            ergo $k+1\leq m$.
        \item
            \Aref{le:gencsere}. lemma szerint a (\dag) vektorrendszerben az $y_{k+1}$ vektor 
            avval az $x$-el
            --- a generátorrendszer tulajdonság megtartásával is --- 
            kicserélhető, 
            amely $x$ szerepel az $y_{k+1}$ vektornak a (\dag)-beli
            vektorokkal képzett lineáris kombinációjában. 
    \end{itemize}
    Ez ellentmondás, hiszen $k$ a legnagyobb olyan szám, 
    amelyre a bizonyítás elején szereplő 1. és 2. feltételek egyszerre állnak fenn.
\end{proof}
\begin{corollary}
    Egy vektortérben bármely két véges egyszerre lineárisan független és egyszerre generátorrendszer elemszáma azonos.
    Konkrétabban, ha
    \[
        \left\{ x_1,\dots,x_m \right\} \text{ és } \left\{ y_1,\dots,y_n \right\}
    \]
    lineárisan független generátorrendszerek, akkor $n=m$.
    \label{co:baziselemszam}
\end{corollary}
\begin{definition}[végesen generált vektortér]\index{végesen generált vektortér}
    Egy vektorteret \emph{végesen generáltnak} nevezünk,
    ha létezik véges elemszámú generátorrendszere.
\end{definition}
Teljesen világos, hogy ha van egy vektortérben véges generátorrendszer,
akkor van minimális generátorrendszer is, azaz van a térben lineárisan független generátorrendszer.
Ezt rögzítjük a következőekben.
\begin{proposition}
    Minden végesen generált vektortérnek van olyan vektorrendszere, 
    amely egyszerre lineárisan független és generátorrendszer.
    \label{pr:bazisletezik}
\end{proposition}
\begin{proof}
    Tekintsünk egy véges generátorrendszert.
    Ha minden elem kívül esik a többi elem lineáris burkában, akkor a rendszer lineárisan független, és készen is vagyunk.
    Ha van olyan elem, amely a többi elem lineáris burkában van, akkor dobjuk el ezt az elemet, és tekintsük, a most már
    eggyel kevesebb elemből álló vektorrendszert. 
    Világos, hogy ez is generátorrendszer marad.

    Folytassuk az eljárást.
    Mivel véges sok vektor van az eredeti generátorrendszerben az algoritmus előbb-utóbb megáll,
    ami azt jelenti, hogy olyan generátorrendszert kapunk, 
    ahol már minden elem a többi lineáris burkán kívül van,
    ergo lineárisan független.
\end{proof}
A lineárisan független generátorrendszerek olyan sűrűn fordulnak elő a tárgyalásban,
hogy rövidebb külön nevet adni nekik.
\begin{definition}[bázis]\index{bázis}
    Egy vektorrendszert \emph{bázisnak} nevezünk, ha ez egyszerre lineárisan független és generátorrendszer.
\end{definition}
\Aref{pr:bazisletezik}. állítást tehát úgy fogalmazhatjuk, hogy végesen generált vektortérnek van bázisa,
és hasonlóan \aref{co:baziselemszam}. következmény pedig azt jelenti, 
hogy egy vektortérben bármely két bázis azonos elemeszámú.
Ezutóbbi tény ad értelmet a következő definíciónak:
\begin{proposition}
    Egy végesen generált vektortérről azt mondjuk, hogy $n$ dimenziós, vagy $n$ a dimenzió száma,
    ha a vektortérben van $n$ elemű bázis.
\end{proposition}
Fontos látni, hogy éppen azt gondoltuk meg, hogy \emph{minden végesen generált vektortérben van bázis}, 
\footnote{Ez nem végesen generált vektorterekre is igaz, de itt nem igazoljuk}
és \emph{bármely két bázis pontosan annyi vektorból áll mint a tér dimenziója.}
A végesen generált vektortereket sokszor szinonimaként \emph{véges dimenziósnak} is mondjuk.\index{véges dimenziós}

Az eddigiek összefoglalásaként is tekinthető a következő állítás.
\begin{proposition}
    Tekintsünk egy $m$-dimenziós vektorteret, és abban egy $m$-elemű
    $\left\{ x_1,\dots,x_m \right\}$
    vektorrendszert.
    E vektorrendszerre tett alábbi feltevések ekvivalensek.
    \begin{enumerate}
        \item Lineárisan független;
        \item Maximális lineárisan független rendszer;
        \item Generátorrendszer;
        \item Minimális generátorrendszer;
        \item Bázis.\qedhere
    \end{enumerate}
\end{proposition}
\begin{proof}
    Az első négy feltétel ekvivalenciájával kezdünk.
    \begin{itemize}
        \item[1.\Rightarrow 2.]
            Mivel a tér $m$-dimenziós, ezért van $m$-elemű generátorrendszere,
            így a Steinitz-lemma szerint nincs $m$-nél több elemet tartalmazó lineárisan független
            rendszere, ergo bármely $m$ elemet tartalmazó lineárisan független rendszer maximális is.
        \item[2.\Rightarrow 3.]
            Láttuk korábban.
        \item[3.\Rightarrow 4.]
            Mivel a tér $m$-dimenziós, ezért van $m$-elemű lineárisan független rendszere,
            így a Steinitz-lemma szerint nincs $m$-nél kevesebb elemet tartalmazó generátorrendszere,
            ergo bármely $m$ elemet tartalmazó generátorrendszer rendszer minimális is.
        \item[4.\Rightarrow 1.]
            Láttuk korábban.
    \end{itemize}

    Az első négy feltétel tehát ugyanazt jelenti. 
    Így ha 1.-et feltesszük, akkor 3. is fennáll, ami azt jelenti, hogy 1. feltétel és 5. feltétel is ekvivalensek.
\end{proof}
A Steinitz-lemma kulcs szerepet játszott dimenzió fogalmának megértésében,
hiszen a bázis elemszáma nem lehetne a tér dimenziója, anélkül hogy tudnánk a tényt: 
bármely két bázis azonos elemszámú! 
Márpedig egy vektortérben nagyon sok bázis van. 
A Steinitz-lemma 2. pontja segít ennek megértéséhez.

\begin{proposition}
    Egy végesen generált vektortér bármely lineárisan független rendszere kiegészíthető bázissá.
    \label{pr:lfgtenbazissa}
\end{proposition}
\begin{proof}
    Tegyük fel, hogy a tér $m$ dimenziós, ami azt jelenti, hogy van 
    \[
        \left\{ x_1,\dots,x_m \right\}
    \]
    $m$ elemű lineárisan független generátorrendszere.
    Legyen $\left\{ y_1,\dots,y_n \right\}$ egy lineárisan független.
    A Steinitz-lemma szerint ez a rendszer kiterjesztehő egy 
    \[
        \left\{ y_1,\dots,y_n,x_{n+1},\dots,x_m \right\}
    \]
    generátorrendszerré, ami persze bázis is.
    Ezt kellett belátni. 
\end{proof}

Meggondoltuk tehát, hogy bármely véges generátorrendszerből elhagyható néhány elem úgy, 
hogy a rendszer lineárisan független generátorrendszerré váljon,
és hasonlóan bármely lineárisan független rendszerhez, hozzátehető néhány elem úgy, hogy a
rendszer lineárisan független generátorrendszerré váljon.





\chapter{Koordinátázás}
\section{Rang-tétel}

\begin{definition}[rang, oszloprang, sorrang, feszítőrang]\index{vektorrendszer rangja}\index{oszloprang}\index{sorrang}\index{feszítőrang}
    Egy véges vektorrendszer \emph{rangján} a vektorrendszer generálta altér dimenzióját értjük.
    Egy mátrix \emph{oszloprangján} a mátrix oszlopai mint vektorrendszer rangját értjük.
    Egy mátrix \emph{sorrangján} a mátrix sorai mint vektorrendszer rangját értjük.
    Ha $A\in\mathbb{F}^{n\times m}$ nemzérus mátrix,
    akkor legkisebb olyan $r$ számot, amelyre
    létezik $B\in\mathbb{F}^{n\times r}$ és $C\in\mathbb{F}^{r\times m}$ mátrix úgy, hogy 
    $A=BC$ szorzatfelbontás teljesül,
    az $A$ mátrix \emph{feszítőrangjának} nevezzük.

    Jelölések: Ha $\left\{ x_1,\dots,x_m \right\}$ a szóbanforgó vektorrendszer, akkor
    \[
        \rank\left\{ x_1,\dots,x_m \right\}=\dim\lin\left\{ x_1,\dots,x_m \right\}
    \]
    Ha $A\in\mathbb{F}^{n\times m}$ egy mátrix,
    akkor 
    \[
        \crank{A}=\rank\left\{ [A]^{j}:j=1,\dots,m \right\},
        \quad
        \rrank{A}=\rank\left\{ [A]_k:k=1,\dots,n\right\},
    \]
    továbbá $\srank{A}$ jelöli a feszítőrangját $A$-nak.
\end{definition}
\begin{proposition}(Rang-tétel)\label{pr:rang}
    Tetszőleges test feletti tetszőleges mátrix mellett, a fent bevezett három rang-koncepció azonos.

    Formálisabban: Minden $A\in\mathbb{F}^{n\times m}$ mellett
    \[
        \crank{A}=\srank{A}=\rrank{A}\qedhere
    \]
\end{proposition}
\begin{proof}
    Induljunk ki a feszítőrang fogalmából.
    Legyen $r=\srank{A}$, és 
    \[
        A=BC,\tag{\dag}
    \]
    ahol $B\in\mathbf{F}^{n\times r},C\in\mathbf{F}^{r\times m}$.
    Azt mutatjuk meg, hogy ekkor $B$ oszloprendszere minimális generátorrendszere $A$ oszlopvektorterének
    és $C$ sorrendszere minimális generátorrendszere $A$ sorvektorterének.

    Vegyük észre, hogy $\srank{A}\leq \crank{A}$.
    Ugyanis ha az $A$ mátrix oszlopvektorterének veszünk egy tetszőleges választott
    $b_1,\dots,b_k$ generátorrendszerét, 
    akkor van $B\in\mathbf{F}^{n\times k}$ és $C\in\mathbf{F}^{k\times m}$ mátrix, hogy $A=BC$.
    Az $r=\srank{A}$ szám az ilyen $k$ számok legkisebbike, tehát valóban $r\leq\crank{A}$.
    \\
    Most tekintsük a (\dag)-ben rögzített szorzatot.
    Jelölje $W$ a $B$ mátrix és $V$ az $A$ mátrix oszlopvektorterét.
    Mivel $BC$ oszlopai $B$ oszlopainak lineáris kombinációja, 
    ezért $A$ minden oszlopa beleesik $W$-be, 
    így az $A$ oszlopainak lineáris burka is részhalmaza $W$-nek,
    azaz 
    \[
        V\subseteq W.
    \]
    A $B$ mátrixnak $r$ darab oszlopa van, 
    tehát $\dim W\leq r$.
    Látjuk tehát, hogy 
    \[\dim W\leq r\leq\crank{A}=\dim V,
    \]
    ami csak úgy lehetséges, 
    hogy $V=W$.
    A $B$ oszlopai tehát $V$-nek is generátorrendszerét alkotját,
    és $r$ minimalitása szerint egy elem sem elhagyható a generátorrendszer tulajdonság
    elvesztése nélkül.

    A sorokra vonatkozó indoklás analóg.
    Először is $\srank{A}\leq \rrank{A}$.
    Ugyanis ha az $A$ mátrix sorvektorterének veszünk egy tetszőleges választott
    $c_1,\dots,c_k$ generátorrendszerét, 
    akkor van $B\in\mathbf{F}^{n\times k}$ és $C\in\mathbf{F}^{k\times m}$ mátrix, hogy $A=BC$.
    Az $r=\srank{A}$ szám az ilyen $k$ számok legkisebbike, tehát valóban $r\leq\rrank{A}$.
    \\
    Most tekintsük a (\dag)-ben rögzített szorzatot.
    Jelölje $W$ a $C$ mátrix és $V$ az $A$ mátrix sorvektorterét.
    Mivel $BC$ sorai $C$ sorainak lineáris kombinációja, 
    ezért $A$ minden sora $W$-be esik,
    így az $A$ sorainak lineáris burka is részhalmaza $W$-nek,
    azaz 
    \[
        V\subseteq W.
    \]
    A $C$ mátrixnak $r$ sora van, 
    tehát $\dim W\leq r$.
    Látjuk tehát, hogy 
    \[\dim W\leq r\leq\crank{A}=\dim V,
    \]
    ami csak úgy lehetséges, 
    hogy $V=W$.
    A $C$ oszlopai tehát $V$-nek is generátorrendszerét alkotját,
    és $r$ minimalitása szerint egy elem sem elhagyható a generátorrendszer tulajdonság
    elvesztése nélkül.

    Ezt kellett belátni. 
\end{proof}
\begin{definition}[mátrix rangja]\index{mátrix rangja}
    Mivel a sorrang, az oszloprang, a feszítőrang minden mátrix mellett azonos,
    ezért a továbbiakban a közös értékre a \emph{rang} szót is használjuk.\footnote{Lásd: \citep{Wardlaw2005}}
\end{definition}
\begin{note}
    Érdemes a rang-tétel következő összegzését megjegyezni.
    Legyen $A\in\mathbb{F}^{n\times m}$ mátrix, amelynek $r$ a rangja.
    Ekkor létezik $A=BC$ felbontása, ahol $B\in\mathbb{F}^{n\times r},C\in\mathbb{F}^{r\times m}$.
    Ez a felbontás persze nem egyértelmű, hiszen $A$ oszlopvektorterének nagyon sok bázisa van.
    Viszont minden ilyen felbontásban $B$ oszloprendszere az $A$ oszlopvektorterének, 
    míg $C$ sorrendszere az $A$ sorvektorterének minimális generátorrendszerét, ergo bázisát alkotja.
\end{note}
Következményképpen érdemes meggondolni a mátrix és inverzének felcserélhetőségére vezető állítást.
\begin{proposition}
    Legyenek $A,B\in\mathbb{F}^{n\times n}$ négyzetes mátrixok, amelyekre AB=I.
    Ekkor BA=I is teljesül.
\end{proposition}
\begin{proof}
    Az identitás mátrix rangja nyilván $n$.
    E mátrix feszítőrangjának definíciójára gondolva, 
    az előző megjegyzés szerint $A$ oszlopai $\mathbb{F}^n$ lineárisan független rendszerét alkotják.
    Vegyük észre, hogy a mátrix szorzás asszociativitását is kihasználva
    \[
        A\left( BA-I \right)=A\left( BA \right)-AI=\left( AB \right)A-AI=IA-AI=A-A=0.
    \]
    Namost, 
    ha $BA\neq I$ lenne, 
    akkor a $BA-I$ mátrixnak lenne egy olyan nem zérus oszlopa, 
    melynek elemeivel mint együtthatókkal képzett lineáris kombinációja az $A$ oszlopainak 
    a zéro vektort eredményezi.
    Ez persze ellentmond az $A$ oszloprendszer lineáris függetlenségének,
    tehát $BA=I$ valóban fennáll.%
    \footnote{%
        A feszítőrang fogalmának ismerete nélküli -- talán még elemibb -- bizonyítás: \citep{doi:10.4169/college.math.j.48.5.366}%
    }%
\end{proof}
\begin{defprop}[invertálható mátrix]
    Legyen $A\in\mathbb{F}^{n\times n}$ egy négyzetes mátrix.
    Az alábbi feltételek egymással ekvivalensek.
    \begin{enumerate}
        \item Van egyetlen olyan $B\in\mathbb{F}^{n\times n}$ mátrix,
            amelyre $AB=I$,
        \item Van olyan $B\in\mathbb{F}^{n\times n}$ mátrix,
            amelyre $AB=I$,
        \item Van egyetlen olyan $B\in\mathbb{F}^{n\times n}$ mátrix,
            amelyre $BA=I$,
        \item Van olyan $B\in\mathbb{F}^{n\times n}$ mátrix,
            amelyre $BA=I$,
        \item $\rank A=n$,
        \item $A$ oszlopai lineárisan független rendszer alkotnak,
        \item $A$ sorai lineárisan független rendszert alkotnak.
    \end{enumerate}
    Ha a fenti feltételek egyike (ergo mindegyike) fennáll, 
    akkor azt mondjuk, hogy $A$ mátrix \emph{invertálható}\index{invertálható mátrix}.
    Szinonímaként használjuk még a \emph{nemszinguláris}\index{nemszinguláris mátrix}, 
    vagy az \emph{reguláris}\index{reguláris mátrix} szavakat.
    Ha egy mátrix nem invertálható, akkor \emph{szingulárisnak}\index{szinguláris mátrix} nevezzük.

    Egy invertálható négyzetes mátrix esetén azt az egyetlen $B$ mátrixot, amelyre
    \(
        AB=I
    \)
    fennáll az $A$ inverzének mondjuk, és $A^{-1}=B$, vel jelöljük.
    Világos, hogy
\[
    AA^{-1}=I=A^{-1}A,\quad (A^{-1})^{-1}=A.\qedhere
\]
\end{defprop}
\begin{proof}
    A 2., 4., 5., 6., 7. állítások ekvivalenciája nyilvánvaló az előzőek szerint.
    Ha $AB=I=AC$, akkor $A\left( B-C \right)=0$ így $A$ oszloprendszere lineáris függetlensége 
    miatt $B=C$. 
    Ezzel $2.\Rightarrow 1.$ implikációt is beláttuk.
    Az 1. és 3. feltevések ekvivalenciája az előző állítás miatt teljesül.
\end{proof}
\chapter{Elemi fogalmak}
Tegyük fel, hogy, hogy ismerjük az alábbi fogalmakat:
\begin{enumerate}
    \item Vektorrendszer lineáris függetlensége;
    \item Altér, lineáris burok, generátorrendszer;
    \item Mátrix szorzás,
    \item Gauss-Jordan elimináció. Pontosan azt tesszük fel, hogy ha $Q$ egy négyzetes mátrix, melynek oszlopai
        lineárisan függetlenek, akkor az elemi sor műveletekkel az identikus mátrixszá transformálható.
        Mivel az elemi sorműveletek, balszorzások alkalmas mátrixszal, 
        azt kapjuk, hogy létezik $P$ mátrix, amelyre $PQ=I$.
\end{enumerate}
Amit határozottan kerülünk, és azt tesszük fel, hogy nem ismerjük,
\begin{enumerate}
    \item Bázisok fogalma,
    \item Különböző bázisok azonos számossága,
    \item determináns.
\end{enumerate}
Külön probléma a mátrix rangjának definíciója.
Nem definálhatom, mint az oszlop vagy sorvektortér dimenzióját, hiszen a felépítés jelen szintjén még nincs bázis.
Természetesen azt sem tudjuk még, hogy a maximális lineárisan független sor- vagy oszloprendszer választástól függetlenül mindig azonos elemszámú.
A mátrix feszítő rangja viszont definiálható.
\begin{definition}[mátrix feszítőrangja]
    Legyen $A\in\mathbf{F}^{n\times m}$ egy tetszőleges nem zérus mátrix.
    Azt mondjuk, hogy feszítő rangja $r$, ha $r$ a legkisebb olyan pozitív egész, amelyre $A$ előáll
    \[
        A=BC
    \]
    alakban, ahol $B\in\mathbf{F}^{n\times r}$ és $C\in\mathbb{F}^{r\times n}$.
\end{definition}
Világos, hogy tetszőleges nemzérus négyzetes mátrixra ez jól definiált és $1\leq r \leq \min\{n,m\}$.
A rang-tételnek a szokásosnál egy kicsit erősebb formáját lehet megfogalmazni a dimenzió fogalmának bevezetése nélkül,
ami a lenti 3. állítás.

\begin{proposition}
    Az alábbi állítások egymás következményei:
    \begin{enumerate}
        \item Homogén lineáris egyenletrendszernek, amelynek több ismeretlene van mint egyenlete,
            mindig létezik nem triviális megoldása.
        \item Lineárisan független vektorrendszer elemszáma nem nagyobb mint egy generátorrendszer elemszáma.
        \item Minden nemzérus mátrixban 
            a maximális lineárisan független oszloprendszerek 
            és maximális lineárisan független sorrendszerek azonos elemszámúak, 
            és ez a szám egybeesik a mátrix feszítőrangjával.
        \item
            Legyen $A,B\in\mathbb{F}^{n\times n}$ négyzetes mátrixok.
            Ekkor $AB=I$ esetén $BA=I$ is fennáll.\qedhere
    \end{enumerate}
\end{proposition}
\begin{proof}[1. \Rightarrow 2.]
    Legyen $\left\{ y_1,\dots,y_n \right\}$ egy generátorrendszer,
    és $\left\{ x_1,\dots,x_m \right\}$ olyan vektorrendszer a vektortérben, ahol $m>n$.
    Meg kell mutatnunk, hogy ez utóbbi egy lineárisan összefüggő.
    Világos, hogy minden $1\leq k\leq m$ mellett
    \[
        x_k=\sum_{j=1}^n\alpha_{j,k}y_j.
    \]
    Egyenlőre tetszőleges $\xi_1,\dots,\xi_m$ együtthatók mellett
    \begin{eqnarray}
        \sum_{k=1}^m\xi_kx_k=
        \sum_{k=1}^m\sum_{j=1}^n\xi_k\alpha_{j,k}y_j=
        \sum_{j=1}^n\left( \sum_{k=1}^m\alpha_{j,k}\xi_k \right)y_j
        \label{eq:sys}
    \end{eqnarray}
    Tekintsük az $\left( \alpha_{j,k} \right)$ együtthatók generálta
    homogén lineáris egyenletrendszert. 
    Itt $j=1,\dots,n$ és $k=1,\dots,m$.
    Mivel $m>n$, ezért az ismeretlenek száma több mint az egyenletek száma.
    Létezik tehát nem triviális megoldás, azaz léteznek nem mind nulla
    $\xi_1,\dots,\xi_m$ számok, amelyekre minden $j=1,\dots,n$ esetén
    \[
        \sum_{k=1}^m\alpha_{j,k}\xi_k=0.
    \]
    Találtunk tehát az $\left\{ x_1,\dots,x_m \right\}$ vektorrendszernek egy
    nem triviális, 
    de a zéro vektort eredményező,
    lineáris kombinációját (\ref{eq:sys}).
\end{proof}
\begin{proof}[2.\Rightarrow 3.]
    Jelölje $r$ az $A\in\mathbb{F}^{n\times m}$ mátrix feszítőrangját.
    Legyen $r_c$ a mátrix egyik rögzített maximális lineárisan független oszloprendszerének elemszáma.
    \begin{itemize}
        \item 
        Ezen oszlopokat egy $B\in\mathbb{F}^{n\times r_c}$ mátrixba téve 
        -- a maximalitás miatt -- létezik olyan $C\in\mathbb{F}^{r_c\times m}$ mátrix, 
        amelyre $A=BC$, azaz $r\leq r_c$.
        \item
        Most tekintsünk egy tetszőleges olyan
        \(
            A=BC
        \)
        felbontást, 
        ahol $B\in\mathbb{F}^{n\times r}$ és $C\in\mathbb{F}^{r\times m}$.
        Jelölje $W$ a $B$ mátrix oszlopai lineáris burkát. 
        Az $A$ mátrix fent rögzített maximális lineárisan független oszloprendszere egy lineárisan független rendszer a 
        $W$ vektortérben,
        és $B$ oszlopai pedig egy generátorrendszer ugyanebben a vektortérben.
        A 2. állitás szerint $r_c\leq r$.
    \end{itemize}
    Evvel megmutattuk, hogy bármely két maximális lineárisan független oszloprendszer azonos elemszámú, és számuk megegyezik a mátrix feszítőrangjával.


    Legyen $r_w$ az $A$ mátrix egyik rögzített maximális lineárisan független sorrendszerének elemszáma.
    \begin{itemize}
        \item 
        Ezen sorokat egy $C\in\mathbb{F}^{r_w\times m}$ mátrixba téve 
        -- a maximalitás miatt -- létezik olyan $B\in\mathbb{F}^{n\times r_w}$ mátrix, 
        amelyre $A=BC$, azaz $r\leq r_w$.
        \item
        Most tekintsünk egy tetszőleges olyan
        \(
            A=BC
        \)
        felbontást, 
        ahol $B\in\mathbb{F}^{n\times r}$ és $C\in\mathbb{F}^{r\times m}$.
        Jelölje most $V$ a $C$ mátrix sorai lineáris burkát. 
        Az $A$ mátrix fent rögzített maximális lineárisan független sorrendszere egy lineárisan független rendszer e
        $V$ vektortérben,
        és $C$ sorai pedig egy generátorrendszert alkotnak ugyanebben a $V$ vektortérben.
        A 2. állitás szerint $r_w\leq r$.
    \end{itemize}
    Evvel azt is megmutattuk, 
    hogy bármely két maximális lineárisan független sorrendszer azonos elemszámú, 
    és számuk megegyezik a mátrix feszítőrangjával.
\end{proof}
\begin{proof}[3.\Rightarrow 4.]
    Tegyük fel, hogy $AB=I$.
    Az identitás mátrix rangja nyilván $n$.
    A 3. állitás miatt a feszítőrang is $n$.
    Ha $A$ oszlopai nem lennének lineárisan függetlenek,
    akkor lenne $A=CD$ felbontás, ahol $C\in\mathbb{F}^{n \times r}$ és $D\in\mathbb{F}^{r\times n}$ valamely $r<n$ mellett.
    Ekkor persze $I=AB=\left( CD \right)B=C\left( DB \right)$ is teljesülne, 
    ahol $DB\in\mathbb{F}^{r\times n}$ ellentmondva az identitás mátrix feszítőrangja definíciójának.
    Világos, hogy
    \[
        A\left( BA-I \right)=
        A\left( BA \right)-AI=
        \left( AB \right)A-AI=IA-AI=0.
    \]
    Figyelembe véve, hogy $A$ oszlopai lineáris függetlenek, ez csak úgy lehetséges, ha $BA-I$ a zéró mátrix, ergo $BA=I$.
\end{proof}
\begin{proof}[4.\Rightarrow 1.]
    Legyen $A\in\mathbb{F}^{n\times m}$ a homogén lineáris egyenletrendszer együttható mátrixa,
    ahol $n$ az egyenletek száma, $m$ az ismeretlenek száma.
    Azt kell megmutatnunk, hogy az oszloprendszer lineárisan összefüggő.
    Ha független lenne, akkor
    egészítsük ki e mátrixot alulról $m-n$ darab csupa nullákat tartalmazó sorral.
    Mivel $m>n$ ezért a kiegészített $Q\in\mathbb{F}^{m\times m}$ mátrix legalsó sora csak nullát tartalmaz.
    Mivel $A$ oszlopai lineárisan függetlenek, ezért $Q$ oszlopai is azok.
    Emiatt létezik $P\in\mathbb{F}^{m\times m}$ mátrix, amelyre $PQ=I$.
    A 3. állitás szerint $I=QP$ is teljesül, 
    ami azt jelenti, hogy $I$ utolsó sora a csupa nullákat tartalmaz, ami ellentmondás.
    Beláttuk tehát, hogy $A$ oszlopai lineárisan összefüggőek, azaz az eredeti egyenletrendszernek van nemtriviális megoldása.
\end{proof}

%%\part{E-dúr hegedűverseny No. 1, Op. 8, RV 269, ,,La primavera''}
%% a vége következik
\backmatter
\pagestyle{empty}
\bibliography{la.bib}
\printindex

\end{document}
% arara: latexmk: { 
% arara: --> engine: lualatex,
% arara: --> options: [ '-pvc' ]
% arara: --> }

\begin{thebibliography}{MMMMMMMMM}
    \bibitem[Dancs István, Puskás Csaba (2006)]{PCSDI}
        Dancs István, Puskás Csaba: \textit{Vektorterek}, Aula kiadó 2001, Budapest, ISBN:963 9345 53 9, BCE Catalogue: bcek.379187.
\end{thebibliography}

Created: Sat 20 Jul 2019 05:02:12 AM CEST
Last Modified : Sat 10 Aug 2019 10:35:20 PM CEST
