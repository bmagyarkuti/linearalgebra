\documentclass[9pt,showtrims]{memoir}
\let\Aref\relax
\usepackage[x11names]{xcolor}
%%%%%%% pdflatex %%%%%%%%%%
%\usepackage[T1]{fontenc}
%\usepackage[utf8]{inputenc}
%\usepackage[hungarian]{babel}[2015/11/24]
%%%%%%% pdflatex %%%%%%%%%%

%%%%%% lualatex %%%%%%%%%%%
%\usepackage{polyglossia}\setdefaultlanguage{magyar}
\usepackage[hungarian]{babel}[2015/11/24]
\usepackage{fontspec}
\defaultfontfeatures{Ligatures=TeX}
\setmainfont{TeX Gyre Pagella}
\setsansfont{Kurier}[Scale=MatchLowercase]
\setmonofont{inconsolata}[Scale=MatchLowercase]
%%%%%% lualatex %%%%%%%%%%%

\frenchspacing
\usepackage{amsmath,amsthm,fixme,natbib}
\fxsetup{status=draft, theme=color, layout={inline}}
\renewcommand{\fixmelogo}{\textcolor{black}{\colorbox{Firebrick1}{\textsf{\textbf{FIX}}}}}

\usepackage[unicode]{hyperref}\hypersetup{final}\usepackage{memhfixc}


\usepackage[a4paper]{geometry}
\usepackage[missing={(Oh my!)},dirty={Oh no!},mark]{gitinfo2}

\setsecheadstyle{\Large\sffamily\bfseries\raggedright}
\setsubsecheadstyle{\large\sffamily\bfseries\raggedright}
\setsubsubsecheadstyle{\sffamily\bfseries\raggedright}
\makeatletter
\newcommand\arraybslash{\let\\\@arraycr}
\patchcmd{\@makechapterhead}
 {\printchaptername \chapternamenum \printchapternum}
 {\printchapternum.\@\chapternamenum \printchaptername}
 {}{}
\makeatother

%\swapnumbers %% a magyar.ldf megfordítja. A polyglossia nem. De a magyar ldf pontot is tesz a számcimke után
\theoremstyle{plain}
\newtheorem{proposition}{állítás}[chapter]
\newtheorem{lemma}[proposition]{lemma}
\newtheorem*{SL}{Steinitz-lemma}
%
\theoremstyle{remark}
\newtheorem{note}[proposition]{megjegyzés}

\theoremstyle{definition}
\newtheorem{definition}[proposition]{definíció}
\newtheorem{corollary}[proposition]{Következmény}

\def\scwords #1 #2 #3 {\textsc{#1} \textsc{#2} \textsc{#3} }
\citeindextrue
\makeindex
\synctex=1
\begin{document}
\chapter*{Előszó}
\scwords Hogy nekem mennyi bajom van.
A legfontosabb forrás \citep{PCSDI}.

\noindent MGy


\noindent Budapest, \today
\tableofcontents*
\chapter{Gauss-Jordan elimináció}
\scwords Lineáris egyenletrendszerek megoldását automatizáljuk.\index{lineáris egyenletrenszer}

\chapter{Vektortér bázisa}
\begin{lemma}[generátorrendszer csere]
    Legyen $\left\{ x_1,\dots,x_m \right\}$ egy generátorrendszere valamely vektortérnek,
    és tegyük fel, hogy valamely $y$ vektorra
    \[
        y=\sum_{j=1}^m\eta_jx_j,
    \]
    ahol $\eta_k\neq 0$ valamely $1\leq k\leq m$ mellett. 
    Ekkor $y$ becserélhető a $k$-adik helyen a generátorrendszerbe, 
    úgy hogy az generátorrendszer maradjon, azaz a
    \[
        \left\{ x_1,\dots,x_{k-1},y,x_{k+1},\dots,x_m \right\}
    \]
    vektorrendszer is generátorrendszer.
    \label{le:gencsere}
\end{lemma}
\begin{proof}
    Fejezzük ki $x_k$-t az $y$-ra feírt formulából:
    \(
        x_k=\frac{1}{\eta_k}y+\sum_{j=1,j\neq k}^m\frac{-1}{\eta_k}\eta_jx_j.
    \)
    Ha $z$ eredetileg 
    \[
        z=\sum_{j=1}^m\zeta_jx_j
    \]
    alakú, akkor $x_k$ helyére betéve, a fent kifejezett formulát és bevezetve a 
    $\delta=\frac{\zeta_k}{\eta_k}$ jelölést, azt kapjuk hogy:
    \begin{multline*}
        z=\zeta_kx_k+\sum_{j=1,j\neq k}^m\zeta_jx_j=
        \\
        =
        \zeta_k
            \left( 
                \frac{1}{\eta_k}y+\sum_{j=1,j\neq k}^m\frac{-1}{\eta_k}\eta_jx_j
            \right)
            +\sum_{j=1,j\neq k}^m\zeta_jx_j
        =
        \frac{\zeta_k}{\eta_k}y+
        \sum_{j=1,j\neq k}^m\left( \zeta_j-\frac{\zeta_k}{\eta_k}\eta_j \right)x_j=
        \\
        =\delta y+
        \sum_{j=1,j\neq k}^m\left( \zeta_j-\delta\eta_j \right)x_j.
    \end{multline*}
    Azt kaptuk tehát, hogy ha egy vektor kifejezhető az eredeti vektorrendszerből az 
    \[
        \left( \zeta_1,\dots,\zeta_m \right) 
    \]
    együtthatókkal, akkor ugyanez a vektor a módosított vektorrendszerből is kifejezhető,
    méghozzá az 
    \[
        \left( 
        \underbrace{\zeta_1-\delta\eta_1}_{1.},
        \underbrace{\zeta_2-\delta\eta_2}_{2.},
        \dots,
        \underbrace{\zeta_{k-1}-\delta\eta_{k-1}}_{k-1.},
        \underbrace{\delta}_{k.},
        \underbrace{\zeta_{k+1}-\delta\eta_{k+1}}_{k+1.},\dots,
        \underbrace{\zeta_m-\delta\eta_m}_{m.}
        \right)
    \]
    együtthatókkal.
\end{proof}
\begin{SL}\index{Steinitz}
    Tegyük fel, hogy az $\left\{ y_1,\dots,y_n \right\}$ egy lineárisan független rendszer és az
    $\left\{ x_1,\dots,x_m \right\}$ vektorrendszer egy generátor rendszer.
    Ekkor
    \begin{enumerate}
        \item $n\leq m$;
        \item Az $x_1,\dots,x_m$ vektorok alkalmas átindexelésével az
            \[
                \left\{ y_1,\dots,y_n,x_{n+1},\dots,x_m \right\}
            \]
            vektorrendszer is generátorrendszer.
            \footnote{Úgy kell a jelöléseket érteni, hogy az $n=0$, de az $n=m$ eset is lehetséges. 
            Az $n=0$ esetben az $y$-okkal jelölt vektorok egyike sem,
            míg az $n=m$ esetben az $x$-el jelölt vektorok egyike sem szerepel az 
            \(
                \left\{ y_1,\dots,y_n,x_{n+1},\dots,x_m \right\}
            \)
            vektorrendszer elemei közt.}
    \end{enumerate}
    \label{le:Stienitz}
\end{SL}
\begin{proof}
    Legyen $k$ a legnagyobb egész a $\left\{ 0,\dots,n \right\}$ egészek közül, amelyre
    \begin{enumerate}
        \item $k\leq m$, és
        \item az $x_1,\dots,x_m$ vektorok alkalmas átindexelésével az
            \[
                \left\{ y_1,\dots,y_k,x_{k+1},\dots,x_m \right\}\tag{\dag}
            \]
            vektorrendszer is generátorrendszer.
    \end{enumerate}
    Ilyen $k$ biztosan van van, hiszen $k=0$ triviálisan jó.
    Összesen azt kell meggondolnunk, hogy $k=n$.
    Ha $k<n$ lenne, 
    \begin{itemize}
        \item 
        akkor létezne $y_{k+1}$ vektor.
        No de, ez az $y_{k+1}$ nem szerepel az $\left\{ y_1,\dots,y_k \right\}$ lineáris burkában,
        ami (\dag) generátorrendszer volta miatt csak úgy lehetséges, hogy $k<m$, 
        ergo $k+1\leq m$.
        \item
        \Aref{le:gencsere}. lemma szerint a (\dag) vektorrendszerben az $y_{k+1}$ vektor 
        avval az $x$-el
        --- a generátorrendszer tulajdonság megtartásával is --- 
        kicserélhető, 
        amely $x$ szerepel az $y_{k+1}$ vektornak a (\dag)-beli
        vektorokkal képzett lineáris kombinációjában. 
    \end{itemize}
    Ez ellentmondás, hiszen $k$ a legnagyobb olyan szám, 
    amelyre a bizonyítás elején szereplő 1. és 2. feltételek egyszerre állnak fenn.
\end{proof}
\begin{corollary}
    Egy vektortérben bármely két véges egyszerre lineárisan független és egyszerre generátorrendszer elmeszáma azonos.
    Konkrétabban, ha
    \[
        \left\{ x_1,\dots,x_m \right\} \text{ és } \left\{ y_1,\dots,y_n \right\}
    \]
    lineárisan független generátorrendszerek, akkor $n=m$.
    \label{co:baziselemszam}
\end{corollary}
\begin{definition}
    Egy vektorteret \emph{végesen generáltnak} nevezünk,
    ha létezik véges elemszámú generátorrendszere.
\end{definition}[végesen generált vektortér]\index{végesen generált vektortér}
\begin{proposition}
    Minden végesen generált vektortérnek van olyan vektorrendszere, 
    amely egyszerre lineárisan független és generátorrendszer.
    \label{pr:bazisletezik}
\end{proposition}
\begin{proof}
    Tekintsünk egy véges generátorrendszert.
    Ha minden elem kívül esik a többi elem lineáris burkában, akkor a rendszer linárisan független, és készen is vagyunk.
    Ha van olyan elem, amely a többi lineáris burkában van, akkor dobjuk el ezt az elemet, és tekintsük, a most már
    eggyel kevesebb elemből álló vektorrendszert. 
    Világos, hogy ez is generátorrendszer marad.

    Folytassuk az eljárást.
    Mivel véges sok vektor van az eredeti generátorrendszerben az algoritmus előbb-utóbb megáll,
    ami azt jelenti, hogy olyan generátorrendszert kapunk, 
    ahol már minden elem a többi lineáris burkán kívül van,
    ergo lineárisan független.
\end{proof}
A lineárisan független generátorrendszerek olyan sűrűn fordulnak elő a tárgyalásban,
hogy rövidebb külön nevet adni nekik.
\begin{definition}[bázis]\index{bázis}
    Egy vektorrendszert \emph{bázisnak} nevezünk, ha ez egyszerre lineárisan független és generátorrendszer.
\end{definition}
\Aref{pr:bazisletezik}. állítást tehát úgy fogalmazhatjuk, hogy végesen generált vektortérnek van bázisa,
és hasonlóan \aref{co:baziselemszam}. következmény pedig azt jelenti, 
hogy egy vektortérben bármely két bázis azonos elemeszámú.
Ezutóbbi tény ad értelmet a következő definíciónak:
\begin{proposition}
    Egy végesen generált vektortérről azt mondjuk, hogy $n$ dimenziós, vagy $n$ a dimenzió száma,
    ha a vektortérben van $n$ elemű bázis.
\end{proposition}
Fontos látni, hogy éppen azt gondoltuk meg, hogy \emph{minden végesen generált vektortérben van bázis}, 
\footnote{Ez nem végesen generált vektorterekre is igaz, de itt nem igazoljuk}
és \emph{bármely két bázis pontosan annyi vektorból áll mint a tér dimenziója.}
A végesen generált vektortereket sokszor szinonimaként \emph{véges dimenziósnak} is mondjuk.\index{véges dimenziós}

A Steinitz-lemma kulcs szerepet játszott dimenzió fogalmának megértésében,
hiszen a bázis elemszáma nem lehetne a tér dimenziója, anélkül hogy tudnánk a tényt: 
bármely két bázis azonos elemszámú! 
Márpedig egy vektortérben nagyon sok bázis van. 
A Steinitz-lemma 2. pontja segít ennek megértéséhez.

\begin{proposition}
    Egy végesen generált vektortér bármely lineárisan független rendszere kiegészíthető bázissá.
    \label{pr:lfgtenbazissa}
\end{proposition}
\begin{proof}
    Tegyük fel, hogy a tér $m$ dimenziós, ami azt jelenti, hogy van 
    \[
        \left\{ x_1,\dots,x_m \right\}
    \]
    $m$ elemű lineárisan független generátorrendszere.
    Legyen $\left\{ y_1,\dots,y_n \right\}$ egy lineárisan független.
    A Steinitz-lemma szerint ez a rendszer kiterjesztehő egy 
    \[
        \left\{ y_1,\dots,y_n,x_{n+1},\dots,x_m \right\}\tag{\dag}
    \]
    generátorrendszerré.
    Ha ez nem lenne lineárisan független,
    akkor lenne olyan eleme, amelynek elhagyásával $m-1$ elemű generátorrendszert kapnánk.
    No de ez képtelenség, hiszen a bázis $m$ elemű lineárisan független rendszer,
    és a Steinitz-lemma éppen azt állítja, hogy ilyen helyzet nem lehetséges.
    Tehát a fenti (\dag) rendszer egy bázis.
    Ezt kellett belátni. 
\end{proof}

Meggondoltuk tehát, hogy bármely véges generátorrendszerből elhagyható néhány elem úgy, 
hogy a rendszer lineárisan független generátorrendszerré váljon,
és hasonlóan bármely lineárisan független rendszerhez, hozzátehető néhány elem úgy, hogy a
rendszer lineárisan független generátorrendszerré váljon.





\begin{thebibliography}{MMMMMMMMM}
    \bibitem[Dancs István, Puskás Csaba (2006)]{PCSDI}
        Dancs István, Puskás Csaba: \textit{Vektorterek}, Aula kiadó 2001, Budapest, ISBN:963 9345 53 9, BCE Catalogue: bcek.379187.
\end{thebibliography}
\printindex
\end{document}
Created: Sat 20 Jul 2019 05:02:12 AM CEST
Last Modified : Mon 22 Jul 2019 01:18:30 PM CEST
