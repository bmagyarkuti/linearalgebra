%%la.tex
\documentclass[9pt,showtrims]{memoir}
\let\Aref\relax
\usepackage[x11names]{xcolor}
%%%%%%% pdflatex %%%%%%%%%%
%\usepackage[T1]{fontenc}
%\usepackage[utf8]{inputenc}
%\usepackage[hungarian]{babel}[2015/11/24]
%%%%%%% pdflatex %%%%%%%%%%

%%%%%% lualatex %%%%%%%%%%%
%\usepackage{polyglossia}\setdefaultlanguage{magyar}
\usepackage[hungarian]{babel}[2015/11/24]
\usepackage{fontspec}
\defaultfontfeatures{Ligatures=TeX}
\setmainfont{TeX Gyre Pagella}
\setsansfont{Kurier}[Scale=MatchLowercase]
\setmonofont{inconsolata}[Scale=MatchLowercase]
%%%%%% lualatex %%%%%%%%%%%

\frenchspacing
\usepackage{amsmath,amsthm,amsfonts,fixme}
\usepackage[numbers]{natbib}
\fxsetup{status=draft, theme=color, layout={inline}}
\renewcommand{\fixmelogo}{\textcolor{black}{\colorbox{Firebrick1}{\textsf{\textbf{FIX}}}}}

\usepackage[unicode]{hyperref}\hypersetup{final}\usepackage{memhfixc}


\usepackage[a4paper]{geometry}
\usepackage[missing={(Oh my!)},dirty={Oh no!},mark]{gitinfo2}

\setsecheadstyle{\Large\sffamily\bfseries\raggedright}
\setsubsecheadstyle{\large\sffamily\bfseries\raggedright}
\setsubsubsecheadstyle{\sffamily\bfseries\raggedright}


\nouppercaseheads
%\makeoddhead{myheadings}{\footnotesize\selectfont\sffamily\leftmark}{}{\footnotesize\selectfont\sffamily\thepage}
%\makeevenhead{myheadings}{\footnotesize\selectfont\sffamily\thepage}{}{\footnotesize\selectfont\sffamily\rightmark}
\makeoddhead{myheadings}{\footnotesize\leftmark}{}{\footnotesize\thepage}
\makeevenhead{myheadings}{\footnotesize\thepage}{\footnotesize\myBotmark}{\footnotesize\rightmark}
\makepsmarks{myheadings}{%
    \renewcommand\chaptermark[1]{%
    \markboth{%
%      \ifnum \value{secnumdepth} > 1
%      \if@mainmatter %
            \thechapter.~\chaptername:~%
%      \fi
%      \fi
        ##1}{\thepart.~\partname}}%
    }

    \makeatletter
    \newcommand\arraybslash{\let\\\@arraycr}
    \patchcmd{\@makechapterhead}
    {\printchaptername \chapternamenum \printchapternum}
    {\printchapternum.\@\chapternamenum \printchaptername}
    {}{}
    \renewenvironment{proof}[1][\proofname]
    {\par\pushQED{\qed}%
    \normalfont \topsep6\p@\@plus6\p@\relax
    \trivlist
\item[\hskip\labelsep
        \itshape
    #1\@addpunct{:}]\ignorespaces}
    {\popQED\endtrivlist\@endpefalse}
    \makeatother

    \newcommand{\addQEDstyle}[2]{\AtBeginEnvironment{#1}{\pushQED{\qed}\renewcommand{\qedsymbol}{#2}}\AtEndEnvironment{#1}{\popQED}}
%% Ezt kellett belátni. trükkök:
%% https://tex.stackexchange.com/questions/16453/denoting-the-end-of-example-remark
%\swapnumbers %% a magyar.ldf megfordítja. A polyglossia nem. De a magyar ldf pontot is tesz a számcimke után
    \theoremstyle{plain}

    \newtheorem{proposition}{állítás}[chapter]
    \newtheorem{lemma}[proposition]{lemma}
    \newtheorem*{SL}{Steinitz-lemma}
%
    \theoremstyle{remark}
    \newtheorem{note}[proposition]{megjegyzés}

    \theoremstyle{definition}
    \newtheorem{definition}[proposition]{definíció}
    \newtheorem{corollary}[proposition]{következmény}
    \newtheorem{defprop}[proposition]{definíció-állítás}
    \addQEDstyle{definition}{$\Diamond$}\addQEDstyle{proposition}{$\Diamond$}\addQEDstyle{lemma}{$\Diamond$}\addQEDstyle{note}{$\Diamond$}\addQEDstyle{corollary}{$\Diamond$}\addQEDstyle{SL}{$\Diamond$}\addQEDstyle{defprop}{$\Diamond$}


%%% https://tex.stackexchange.com/questions/319474/put-current-theorem-like-items-name-number-in-header
% \myBotmark feltöltése a lapon lévő utolsó tétel környezettel
    \makeatletter
    \@ifdefinable\@my@claim@mark{\newmarks\@my@claim@mark}
    \newcommand*\myMark[1]{\marks\@my@claim@mark{#1}}
    \newcommand*\myBotmark{\botmarks\@my@claim@mark}
    \patchcmd{\@begintheorem}{% search for:
        \thm@swap\swappedhead\thmhead % more specific than before
    }{% replace with:
        \myMark{#2.\@ifnotempty{#1}{\ #1}\@ifnotempty{#3}{\ (#3)}}%
        \thm@swap\swappedhead\thmhead
    }{
        \typeout{>>> Made patch specific for amsthm.}
    }{
        \typeout{>>> Patch specific for amsthm FAILED!}
    }
    \makeatother

    \renewcommand{\mathbf}{\mathbb}
    \DeclareMathOperator{\lin}{lin}
    \DeclareMathOperator{\crank}{crank}
    \DeclareMathOperator{\rrank}{rrank}
    \DeclareMathOperator{\srank}{srank}
    \DeclareMathOperator{\rank}{rank}
%\DeclareMathOperator{\dim}{dim}
    \def\scwords #1 #2 #3 {\textsc{#1} \textsc{#2} \textsc{#3} }
    \citeindextrue
    \makeindex
    \synctex=1
\begin{document}
\frontmatter*
\chapter*{Előszó}
\scwords Hogy nekem mennyi bajom van.
A legfontosabb forrás \citep{DancsPuskas2001}.

\noindent MGy


\noindent Budapest, \today
\tableofcontents*
\pagestyle{myheadings}
\mainmatter*
\part{F-dúr hegedűverseny No. 3, Op. 8, RV 293, ,,L'autunno''}
\chapter{Gauss-Jordan elimináció}
\scwords Lineáris egyenletrendszerek megoldását automatizáljuk.\index{lineáris egyenletrenszer}

\chapter{Algebrai struktúrák}
\chapter{Vektortér fogalma}
\chapter{Altér}
\chapter{Lineárisan független rendszerek és generátorrendszerek}
\begin{definition}[lineárisan összefüggő vektorrendszer]\index{lineárisan összefüggő rendszer}
    Egy véges $\left\{ y_1,\dots,y_n \right\}$ vektorrendszert \emph{lineárisan összefüggőnek}
    mondunk, ha egyik vektora kifejezhető a többi vektor lineáris kombinációjaként.
\end{definition}
Úgy is fogalmazhatnánk, hogy az $\left\{ y_1,\dots,y_n \right\}$ rendszer pontosan akkor
lineárisan összefüggő, ha létezik $1\leq k\leq n$ index, amelyre
\[
    y_k\in\lin\left\{ y_1,\dots,y_{k-1},y_{k+1},\dots,y_n \right\}.
\]
\begin{proposition}
    Legyen $\left\{ y_1,\dots,y_n \right\}$ vektorrendszer rögzítve a $V$ vektortérben.
    A vektorrendszerre tett alábbi feltevések egymással ekvivalensek.
    \begin{enumerate}
        \item Lineárisan összefüggő;
        \item Van olyan elem a vektortérben, amely nem csak egyféleképpen áll elő mint az $y_1,\dots,y_n$
            vektorok lineáriskombinációja,\\
            azaz formálisabban:
            létezik z\in V, amelyre $z=\sum_{j=1}^n\xi_jy_j$ és $z=\sum_{j=1}^n\eta_jy_j$
            és létezik $1\leq k\leq n$, amelyre $\xi_k\neq\eta_k$.
        \item Vannak olyan nem mind zérus $\alpha_1,\dots,\alpha_n$ skalárok, amellyekkel
            \[
                \sum_{j=1}^n\alpha_jy_j=0.\qedhere
            \]
    \end{enumerate}
\end{proposition}
\begin{proof}
    Körben bizonyítunk.
    \begin{description}
        \item[$1.\Rightarrow 2.$] 
            Tegyük fel, hogy $y_k=\sum_{j=1}^{k-1}\eta_jy_j+\sum_{j=k+1}^n\eta_jy_j$.
            Ekkor az alábbi együtthatórendszerek
            \[
                \left( \alpha_1,\dots,\alpha_{k-1},0,\alpha_{k+1},\dots,\alpha_n \right)
                \qquad
                \left( 0,\dots,0,1,0,\dots,0 \right)
            \]
            a $k$-adik helyen biztosan különböznek, 
            hiszen $0\neq 1$, 
            és mind a két együtthatórendszerrel képzett lineáris kombináció ugyanazt az $y_k$ vektort eredményezi.
        \item[$2.\Rightarrow 3.$]
            Világos, hogy 
            \[
                0=z-z=
                \sum_{j=1}^n\left( \xi_j-\eta_j \right)y_j
            \]
            és a $k$-adik skalár nem zérus.
        \item[$3.\Rightarrow 1.$]
            Tegyük fel most, hogy 
            \(
            \sum_{j=1}^n\alpha_jy_j=0,
            \)
            és, hogy $\alpha_k\neq 0.$
            Ekkor 
            \[
                y_k=\sum_{j=1,j\neq k}^n-\frac{1}{\alpha_k}\alpha_jy_j
            \]
            azaz a $k$-adik vektor tekinthető mint a többi vektor valamely lineáris kombinációja.
    \end{description}
    Ezt kellett belátni. 
\end{proof}
Fontos észrevétel a következő.
\begin{proposition}
    Minden, 
    valamely lineárisan összefüggő vektorrendszert tartalmazó vektorrendszer maga is lineárisan összefüggő.
\end{proposition}
\begin{definition}
    Egy nem feltétetlen véges vektorrendszert lineárisan összefüggőnek nevezünk,
    ha van véges részrendszere, amely lineárisan összefüggő.
\end{definition}
\begin{definition}[lineárisan független vektorrendszer]\index{lineárisan független rendszer}
    Egy vektorrendszer \emph{lineárisan független}, ha nem lineárisan összefüggő.
\end{definition}
Így egy nem véges vektorrendszer akkor lineárisan független, ha minden véges részrendszere is az.
Egy véges vektorrendszer lineárisan függetlenségét, pedig a következő egymással ekvivalens állítások karakterizálják.
\begin{proposition}
    Legyen $\left\{ y_1,\dots,y_n \right\}$ vektorrendszer rögzítve a $V$ vektortérben.
    A vektorrendszerre tett alábbi feltevések egymással ekvivalensek.
    \begin{enumerate}
        \item Lineárisan független;
        \item A vektorrendszer lineáris burkában minden elem egyetlen egy féle képpen áll elő,
            mint az $y_1,\dots,y_n$ vektorok lineáris kombinációja.
        \item Az $y_1,\dots,y_n$ vektoroknak csak a triviális lináris kombinációja zérus,
            azaz
            \[
                \sum_{j=1}^n\alpha_jy_j=0\text{ esetén }\alpha_1=\alpha_2=\dots=\alpha_n=0.\qedhere
            \]
    \end{enumerate}
\end{proposition}
\begin{proof}
    Nyilvánvaló a lineáris összefüggés karakterizációjából.
\end{proof}
\begin{definition}[maximális lineárisan független-- és minimális generátorrendszer]\index{maximális lineárisan független rendszer}\index{minimális generátorrendszer}
    Egy lineárisan független rendszert \emph{maximális lineárisan független rendszernek} nevezünk,
    ha nem lehet bővíteni úgy, hogy lineárisan független maradjon.

    Egy generátorrendszert \emph{minimális generátorrendszernek} mondunk, ha nem lehet szűkíteni úgy,
    hogy generátorrendszer maradjon.
\end{definition}
\begin{proposition}
    Legyen $\left\{ x_1,\dots,x_m \right\}$ egy vektorrendszere a $V$ vektortérnek.
    Az alábbi feltevések ekvivalensek.
    \begin{enumerate}
        \item A vektorrendszer maximális lineárisan független rendszer.
        \item A vektorrendszer egyszerre lineárisan független és generátorrendszer.
        \item A vektorrendszer minimális generátorrendszer.\qedhere
    \end{enumerate}
\end{proposition}
\begin{proof}
    Az alábbi lépéseket követjük.
    \begin{itemize}
        \item[1.\Rightarrow 2.]
            Ha a vektorrendszer nem lenne generátorrendszer is,
            akkor a lineáris burkán kívül lenne egy vektor.
            Ezt a vektorrendszerhez illesztve, a vektorrendszer egy valódi lineárisan független bővítését kapjuk,
            ami ellentmond a maximalitás feltételének.
        \item[3.\Rightarrow 2.]
            Ha a vektorrendszer egyik eleme a többi elem lineáris kombinációja,
            akkor azt az elemet elhagyva is generátorrendszert kapunk,
            ami ellentmond a minimalitás feltételének.
        \item[2.\Rightarrow 1.]
            Ha nem lenne maximális a lineárisan független tulajdonságra nézve,
            akkor létezne egy vektor a lineáris burkán kívül is,
            ami ellentmond a generátorrendszer tulajdonságnak.
        \item[2.\Rightarrow 3.]
            Mivel a vektorrendszer egyik eleme, sincs a többi lineáris burkában,
            ezért egyetlen elemet sem hagyhatunk el a generátorrendszer tulajdonság megtartásával,
            ami azt jelenti, hogy ez egy minimális generátorrendszer.\qedhere
    \end{itemize}
\end{proof}
Egy a zéró vektort tartalmazó vektorrendszer persze lineárisan összefüggő, 
és egy nem zérus vektorból álló egyelemű vektorrendszer lineárisan független.
A következő állítás sokszor teszi kényelmessé a gondolatmenetünket. 
\begin{proposition}
    Legyen $\left\{ y_1,\dots,y_n \right\}$ egy olyan legalább két elemű vektorrendszer,
    amelynek első eleme nem a zérus vektor, tehát $y_1\neq 0$.
    A vektorrendszer pontosan akkor lineárisan összefüggő,
    ha létezik olyan eleme, 
    amely pusztán az előző elemek lineárisan kombinációja.

    Formálisabban: akkor és csak akkor, 
    ha 
    $\exists k\quad 2\leq k\leq n : y_k\in\lin\left\{ y_1,\dots,y_{k-1} \right\}$
\end{proposition}
\begin{proof}
    Tegyük fel, hogy a vektorrendszer lineárisan összefüggő.
    Ekkor van olyan a zérus vektort eredményező lineáris kombinációja
    \(
    \alpha_1y_1+\dots+\alpha_ny_n=0,
    \)
    ahol nem az összes együttható nulla.
    Legyen $k$ a lineáris kombinációban a legnagyobb nem nulla együttható indexe.
    Világos, hogy $k\neq 1$, 
    hiszen $y_1\neq 0$.
    Persze a $k$ feletti együtthatók mind nullák,
    emiatt
    \[
        \alpha_1y_1+\dots+\alpha_ky_k=0.
    \]
    Itt már $\alpha_k\neq 0$, tehát $y_k$ kifejezhető az előző vektorok segítségével:
    \[
        y_k=
        \frac{-1}{\alpha_k}\alpha_1y_1+\dots+\frac{-1}{\alpha_{k-1}}\alpha_{k-1}y_{k-1}.\qedhere
    \]
\end{proof}
\chapter{Vektortér bázisa}
\scwords A Steinitz-lemma fontosságát kell kiemelni.
A Steinitz-lemma legegyszerűbb megfogalmazásában azt állítja, 
hogy \emph{egy lineárisan független rendszer elemszáma, soha nem nagyobb mint egy generátorrendszer elemszáma.}
Ez a tény vezet a bázis és a dimenzió fogalmához.

A generátorrendszer cseréről szóló legelső lemmának is fontos szerepe van az itt választott felépítésben.
Egyrészt használjuk a Steinitz-lemma igazolásában, másrészt ennek segítségével tisztázzuk majd azt a kérdést,
hogy hogyan alakulnak egy vektor koordinátái, ha a bázist, tehát a vonatkoztatási rendszert változtatjuk.

\begin{lemma}[generátorrendszer csere]
    Legyen $\left\{ x_1,\dots,x_m \right\}$ egy generátorrendszere valamely vektortérnek,
    és tegyük fel, hogy valamely $y$ vektorra
    \[
        y=\sum_{j=1}^m\eta_jx_j,
    \]
    ahol $\eta_k\neq 0$ valamely $1\leq k\leq m$ mellett. 
    Ekkor $y$ becserélhető a $k$-adik helyen a generátorrendszerbe, 
    úgy hogy az generátorrendszer maradjon, azaz a
    \[
        \left\{ x_1,\dots,x_{k-1},y,x_{k+1},\dots,x_m \right\}
    \]
    vektorrendszer is generátorrendszer.
    \label{le:gencsere}
\end{lemma}
\begin{proof}
    Fejezzük ki $x_k$-t az $y$-ra feírt formulából:
    \(
    x_k=\frac{1}{\eta_k}y+\sum_{j=1,j\neq k}^m\frac{-1}{\eta_k}\eta_jx_j.
    \)
    Ha $z$ eredetileg 
    \[
        z=\sum_{j=1}^m\zeta_jx_j
    \]
    alakú, akkor $x_k$ helyére betéve, a fent kifejezett formulát és bevezetve a 
    $\delta=\frac{\zeta_k}{\eta_k}$ jelölést, azt kapjuk hogy:
    \begin{multline*}
        z=\zeta_kx_k+\sum_{j=1,j\neq k}^m\zeta_jx_j=
        \\
        =
        \zeta_k
        \left( 
        \frac{1}{\eta_k}y+\sum_{j=1,j\neq k}^m\frac{-1}{\eta_k}\eta_jx_j
        \right)
        +\sum_{j=1,j\neq k}^m\zeta_jx_j
        =
        \frac{\zeta_k}{\eta_k}y+
        \sum_{j=1,j\neq k}^m\left( \zeta_j-\frac{\zeta_k}{\eta_k}\eta_j \right)x_j=
        \\
        =\delta y+
        \sum_{j=1,j\neq k}^m\left( \zeta_j-\delta\eta_j \right)x_j.
    \end{multline*}
    Azt kaptuk tehát, hogy ha egy vektor kifejezhető az eredeti vektorrendszerből az 
    \[
        \left( \zeta_1,\dots,\zeta_m \right) 
    \]
    együtthatókkal, akkor ugyanez a vektor a módosított vektorrendszerből is kifejezhető,
    méghozzá az 
    \[
        \left( 
        \underbrace{\zeta_1-\delta\eta_1}_{1.},
        \underbrace{\zeta_2-\delta\eta_2}_{2.},
        \dots,
        \underbrace{\zeta_{k-1}-\delta\eta_{k-1}}_{k-1.},
        \underbrace{\delta}_{k.},
        \underbrace{\zeta_{k+1}-\delta\eta_{k+1}}_{k+1.},\dots,
        \underbrace{\zeta_m-\delta\eta_m}_{m.}
        \right)
    \]
    együtthatókkal.
\end{proof}
\begin{SL}\index{Steinitz}
    Tegyük fel, hogy az $\left\{ y_1,\dots,y_n \right\}$ egy lineárisan független rendszer és az
    $\left\{ x_1,\dots,x_m \right\}$ vektorrendszer egy generátor rendszer.
    Ekkor
    \begin{enumerate}
        \item $n\leq m$;
        \item Az $x_1,\dots,x_m$ vektorok alkalmas átindexelésével az
            \[
                \left\{ y_1,\dots,y_n,x_{n+1},\dots,x_m \right\}
            \]
            vektorrendszer is generátorrendszer.%
            \footnote{Úgy kell a jelöléseket érteni, hogy az $n=0$, de az $n=m$ eset is lehetséges. 
                Az $n=0$ esetben az $y$-okkal jelölt vektorok egyike sem,
                míg az $n=m$ esetben az $x$-el jelölt vektorok egyike sem szerepel az 
                \(
                \left\{ y_1,\dots,y_n,x_{n+1},\dots,x_m \right\}
                \)
            vektorrendszer elemei közt.}%
            \qedhere
    \end{enumerate}
    \label{le:Stienitz}
\end{SL}
\begin{proof}
    Legyen $k$ a legnagyobb egész a $\left\{ 0,\dots,n \right\}$ egészek közül, amelyre
    \begin{enumerate}
        \item $k\leq m$, és
        \item az $x_1,\dots,x_m$ vektorok alkalmas átindexelésével az
            \[
                \left\{ y_1,\dots,y_k,x_{k+1},\dots,x_m \right\}\tag{\dag}
            \]
            vektorrendszer is generátorrendszer.
    \end{enumerate}
    Ilyen $k$ biztosan van van, hiszen $k=0$ triviálisan jó.
    Összesen azt kell meggondolnunk, hogy $k=n$.
    Ha $k<n$ lenne, 
    \begin{itemize}
        \item 
            akkor létezne $y_{k+1}$ vektor.
            No de, ez az $y_{k+1}$ nem szerepel az $\left\{ y_1,\dots,y_k \right\}$ lineáris burkában,
            ami (\dag) generátorrendszer volta miatt csak úgy lehetséges, hogy $k<m$, 
            ergo $k+1\leq m$.
        \item
            \Aref{le:gencsere}. lemma szerint a (\dag) vektorrendszerben az $y_{k+1}$ vektor 
            avval az $x$-el
            --- a generátorrendszer tulajdonság megtartásával is --- 
            kicserélhető, 
            amely $x$ szerepel az $y_{k+1}$ vektornak a (\dag)-beli
            vektorokkal képzett lineáris kombinációjában. 
    \end{itemize}
    Ez ellentmondás, hiszen $k$ a legnagyobb olyan szám, 
    amelyre a bizonyítás elején szereplő 1. és 2. feltételek egyszerre állnak fenn.
\end{proof}
\begin{corollary}
    Egy vektortérben bármely két véges egyszerre lineárisan független és egyszerre generátorrendszer elemszáma azonos.
    Konkrétabban, ha
    \[
        \left\{ x_1,\dots,x_m \right\} \text{ és } \left\{ y_1,\dots,y_n \right\}
    \]
    lineárisan független generátorrendszerek, akkor $n=m$.
    \label{co:baziselemszam}
\end{corollary}
\begin{definition}[végesen generált vektortér]\index{végesen generált vektortér}
    Egy vektorteret \emph{végesen generáltnak} nevezünk,
    ha létezik véges elemszámú generátorrendszere.
\end{definition}
Teljesen világos, hogy ha van egy vektortérben véges generátorrendszer,
akkor van minimális generátorrendszer is, azaz van a térben lineárisan független generátorrendszer.
Ezt rögzítjük a következőekben.
\begin{proposition}
    Minden végesen generált vektortérnek van olyan vektorrendszere, 
    amely egyszerre lineárisan független és generátorrendszer.
    \label{pr:bazisletezik}
\end{proposition}
\begin{proof}
    Tekintsünk egy véges generátorrendszert.
    Ha minden elem kívül esik a többi elem lineáris burkában, akkor a rendszer lineárisan független, és készen is vagyunk.
    Ha van olyan elem, amely a többi elem lineáris burkában van, akkor dobjuk el ezt az elemet, és tekintsük, a most már
    eggyel kevesebb elemből álló vektorrendszert. 
    Világos, hogy ez is generátorrendszer marad.

    Folytassuk az eljárást.
    Mivel véges sok vektor van az eredeti generátorrendszerben az algoritmus előbb-utóbb megáll,
    ami azt jelenti, hogy olyan generátorrendszert kapunk, 
    ahol már minden elem a többi lineáris burkán kívül van,
    ergo lineárisan független.
\end{proof}
A lineárisan független generátorrendszerek olyan sűrűn fordulnak elő a tárgyalásban,
hogy rövidebb külön nevet adni nekik.
\begin{definition}[bázis]\index{bázis}
    Egy vektorrendszert \emph{bázisnak} nevezünk, ha ez egyszerre lineárisan független és generátorrendszer.
\end{definition}
\Aref{pr:bazisletezik}. állítást tehát úgy fogalmazhatjuk, hogy végesen generált vektortérnek van bázisa,
és hasonlóan \aref{co:baziselemszam}. következmény pedig azt jelenti, 
hogy egy vektortérben bármely két bázis azonos elemeszámú.
Ezutóbbi tény ad értelmet a következő definíciónak:
\begin{proposition}
    Egy végesen generált vektortérről azt mondjuk, hogy $n$ dimenziós, vagy $n$ a dimenzió száma,
    ha a vektortérben van $n$ elemű bázis.
\end{proposition}
Fontos látni, hogy éppen azt gondoltuk meg, hogy \emph{minden végesen generált vektortérben van bázis}, 
\footnote{Ez nem végesen generált vektorterekre is igaz, de itt nem igazoljuk}
és \emph{bármely két bázis pontosan annyi vektorból áll mint a tér dimenziója.}
A végesen generált vektortereket sokszor szinonimaként \emph{véges dimenziósnak} is mondjuk.\index{véges dimenziós}

Az eddigiek összefoglalásaként is tekinthető a következő állítás.
\begin{proposition}
    Tekintsünk egy $m$-dimenziós vektorteret, és abban egy $m$-elemű
    $\left\{ x_1,\dots,x_m \right\}$
    vektorrendszert.
    E vektorrendszerre tett alábbi feltevések ekvivalensek.
    \begin{enumerate}
        \item Lineárisan független;
        \item Maximális lineárisan független rendszer;
        \item Generátorrendszer;
        \item Minimális generátorrendszer;
        \item Bázis.\qedhere
    \end{enumerate}
\end{proposition}
\begin{proof}
    Az első négy feltétel ekvivalenciájával kezdünk.
    \begin{itemize}
        \item[1.\Rightarrow 2.]
            Mivel a tér $m$-dimenziós, ezért van $m$-elemű generátorrendszere,
            így a Steinitz-lemma szerint nincs $m$-nél több elemet tartalmazó lineárisan független
            rendszere, ergo bármely $m$ elemet tartalmazó lineárisan független rendszer maximális is.
        \item[2.\Rightarrow 3.]
            Láttuk korábban.
        \item[3.\Rightarrow 4.]
            Mivel a tér $m$-dimenziós, ezért van $m$-elemű lineárisan független rendszere,
            így a Steinitz-lemma szerint nincs $m$-nél kevesebb elemet tartalmazó generátorrendszere,
            ergo bármely $m$ elemet tartalmazó generátorrendszer rendszer minimális is.
        \item[4.\Rightarrow 1.]
            Láttuk korábban.
    \end{itemize}

    Az első négy feltétel tehát ugyanazt jelenti. 
    Így ha 1.-et feltesszük, akkor 3. is fennáll, ami azt jelenti, hogy 1. feltétel és 5. feltétel is ekvivalensek.
\end{proof}
A Steinitz-lemma kulcs szerepet játszott dimenzió fogalmának megértésében,
hiszen a bázis elemszáma nem lehetne a tér dimenziója, anélkül hogy tudnánk a tényt: 
bármely két bázis azonos elemszámú! 
Márpedig egy vektortérben nagyon sok bázis van. 
A Steinitz-lemma 2. pontja segít ennek megértéséhez.

\begin{proposition}
    Egy végesen generált vektortér bármely lineárisan független rendszere kiegészíthető bázissá.
    \label{pr:lfgtenbazissa}
\end{proposition}
\begin{proof}
    Tegyük fel, hogy a tér $m$ dimenziós, ami azt jelenti, hogy van 
    \[
        \left\{ x_1,\dots,x_m \right\}
    \]
    $m$ elemű lineárisan független generátorrendszere.
    Legyen $\left\{ y_1,\dots,y_n \right\}$ egy lineárisan független.
    A Steinitz-lemma szerint ez a rendszer kiterjesztehő egy 
    \[
        \left\{ y_1,\dots,y_n,x_{n+1},\dots,x_m \right\}
    \]
    generátorrendszerré, ami persze bázis is.
    Ezt kellett belátni. 
\end{proof}

Meggondoltuk tehát, hogy bármely véges generátorrendszerből elhagyható néhány elem úgy, 
hogy a rendszer lineárisan független generátorrendszerré váljon,
és hasonlóan bármely lineárisan független rendszerhez, hozzátehető néhány elem úgy, hogy a
rendszer lineárisan független generátorrendszerré váljon.





\chapter{Koordinátázás}
\section{Rang-tétel}

\begin{definition}[rang, oszloprang, sorrang, feszítőrang]\index{vektorrendszer rangja}\index{oszloprang}\index{sorrang}\index{feszítőrang}
    Egy véges vektorrendszer \emph{rangján} a vektorrendszer generálta altér dimenzióját értjük.
    Egy mátrix \emph{oszloprangján} a mátrix oszlopai mint vektorrendszer rangját értjük.
    Egy mátrix \emph{sorrangján} a mátrix sorai mint vektorrendszer rangját értjük.
    Ha $A\in\mathbb{F}^{n\times m}$ nemzérus mátrix,
    akkor legkisebb olyan $r$ számot, amelyre
    létezik $B\in\mathbb{F}^{n\times r}$ és $C\in\mathbb{F}^{r\times m}$ mátrix úgy, hogy 
    $A=BC$ szorzatfelbontás teljesül,
    az $A$ mátrix \emph{feszítőrangjának} nevezzük.

    Jelölések: Ha $\left\{ x_1,\dots,x_m \right\}$ a szóbanforgó vektorrendszer, akkor
    \[
        \rank\left\{ x_1,\dots,x_m \right\}=\dim\lin\left\{ x_1,\dots,x_m \right\}
    \]
    Ha $A\in\mathbb{F}^{n\times m}$ egy mátrix,
    akkor 
    \[
        \crank{A}=\rank\left\{ [A]^{j}:j=1,\dots,m \right\},
        \quad
        \rrank{A}=\rank\left\{ [A]_k:k=1,\dots,n\right\},
    \]
    továbbá $\srank{A}$ jelöli a feszítőrangját $A$-nak.
\end{definition}
\begin{proposition}(Rang-tétel)\label{pr:rang}
    Tetszőleges test feletti tetszőleges mátrix mellett, a fent bevezett három rang-koncepció azonos.

    Formálisabban: Minden $A\in\mathbb{F}^{n\times m}$ mellett
    \[
        \crank{A}=\srank{A}=\rrank{A}\qedhere
    \]
\end{proposition}
\begin{proof}
    Induljunk ki a feszítőrang fogalmából.
    Legyen $r=\srank{A}$, és 
    \[
        A=BC,\tag{\dag}
    \]
    ahol $B\in\mathbf{F}^{n\times r},C\in\mathbf{F}^{r\times m}$.
    Azt mutatjuk meg, hogy ekkor $B$ oszloprendszere minimális generátorrendszere $A$ oszlopvektorterének
    és $C$ sorrendszere minimális generátorrendszere $A$ sorvektorterének.

    Vegyük észre, hogy $\srank{A}\leq \crank{A}$.
    Ugyanis ha az $A$ mátrix oszlopvektorterének veszünk egy tetszőleges választott
    $b_1,\dots,b_k$ generátorrendszerét, 
    akkor van $B\in\mathbf{F}^{n\times k}$ és $C\in\mathbf{F}^{k\times m}$ mátrix, hogy $A=BC$.
    Az $r=\srank{A}$ szám az ilyen $k$ számok legkisebbike, tehát valóban $r\leq\crank{A}$.
    \\
    Most tekintsük a (\dag)-ben rögzített szorzatot.
    Jelölje $W$ a $B$ mátrix és $V$ az $A$ mátrix oszlopvektorterét.
    Mivel $BC$ oszlopai $B$ oszlopainak lineáris kombinációja, 
    ezért $A$ minden oszlopa beleesik $W$-be, 
    így az $A$ oszlopainak lineáris burka is részhalmaza $W$-nek,
    azaz 
    \[
        V\subseteq W.
    \]
    A $B$ mátrixnak $r$ darab oszlopa van, 
    tehát $\dim W\leq r$.
    Látjuk tehát, hogy 
    \[\dim W\leq r\leq\crank{A}=\dim V,
    \]
    ami csak úgy lehetséges, 
    hogy $V=W$.
    A $B$ oszlopai tehát $V$-nek is generátorrendszerét alkotját,
    és $r$ minimalitása szerint egy elem sem elhagyható a generátorrendszer tulajdonság
    elvesztése nélkül.

    A sorokra vonatkozó indoklás analóg.
    Először is $\srank{A}\leq \rrank{A}$.
    Ugyanis ha az $A$ mátrix sorvektorterének veszünk egy tetszőleges választott
    $c_1,\dots,c_k$ generátorrendszerét, 
    akkor van $B\in\mathbf{F}^{n\times k}$ és $C\in\mathbf{F}^{k\times m}$ mátrix, hogy $A=BC$.
    Az $r=\srank{A}$ szám az ilyen $k$ számok legkisebbike, tehát valóban $r\leq\rrank{A}$.
    \\
    Most tekintsük a (\dag)-ben rögzített szorzatot.
    Jelölje $W$ a $C$ mátrix és $V$ az $A$ mátrix sorvektorterét.
    Mivel $BC$ sorai $C$ sorainak lineáris kombinációja, 
    ezért $A$ minden sora $W$-be esik,
    így az $A$ sorainak lineáris burka is részhalmaza $W$-nek,
    azaz 
    \[
        V\subseteq W.
    \]
    A $C$ mátrixnak $r$ sora van, 
    tehát $\dim W\leq r$.
    Látjuk tehát, hogy 
    \[\dim W\leq r\leq\crank{A}=\dim V,
    \]
    ami csak úgy lehetséges, 
    hogy $V=W$.
    A $C$ oszlopai tehát $V$-nek is generátorrendszerét alkotját,
    és $r$ minimalitása szerint egy elem sem elhagyható a generátorrendszer tulajdonság
    elvesztése nélkül.

    Ezt kellett belátni. 
\end{proof}
\begin{definition}[mátrix rangja]\index{mátrix rangja}
    Mivel a sorrang, az oszloprang, a feszítőrang minden mátrix mellett azonos,
    ezért a továbbiakban a közös értékre a \emph{rang} szót is használjuk.\footnote{Lásd: \citep{Wardlaw2005}}
\end{definition}
\begin{note}
    Érdemes a rang-tétel következő összegzését megjegyezni.
    Legyen $A\in\mathbb{F}^{n\times m}$ mátrix, amelynek $r$ a rangja.
    Ekkor létezik $A=BC$ felbontása, ahol $B\in\mathbb{F}^{n\times r},C\in\mathbb{F}^{r\times m}$.
    Ez a felbontás persze nem egyértelmű, hiszen $A$ oszlopvektorterének nagyon sok bázisa van.
    Viszont minden ilyen felbontásban $B$ oszloprendszere az $A$ oszlopvektorterének, 
    míg $C$ sorrendszere az $A$ sorvektorterének minimális generátorrendszerét, ergo bázisát alkotja.
\end{note}
Következményképpen érdemes meggondolni a mátrix és inverzének felcserélhetőségére vezető állítást.
\begin{proposition}
    Legyenek $A,B\in\mathbb{F}^{n\times n}$ négyzetes mátrixok, amelyekre AB=I.
    Ekkor BA=I is teljesül.
\end{proposition}
\begin{proof}
    Az identitás mátrix rangja nyilván $n$.
    E mátrix feszítőrangjának definíciójára gondolva, 
    az előző megjegyzés szerint $A$ oszlopai $\mathbb{F}^n$ lineárisan független rendszerét alkotják.
    Vegyük észre, hogy a mátrix szorzás asszociativitását is kihasználva
    \[
        A\left( BA-I \right)=A\left( BA \right)-AI=\left( AB \right)A-AI=IA-AI=A-A=0.
    \]
    Namost, 
    ha $BA\neq I$ lenne, 
    akkor a $BA-I$ mátrixnak lenne egy olyan nem zérus oszlopa, 
    melynek elemeivel mint együtthatókkal képzett lineáris kombinációja az $A$ oszlopainak 
    a zéro vektort eredményezi.
    Ez persze ellentmond az $A$ oszloprendszer lineáris függetlenségének,
    tehát $BA=I$ valóban fennáll.%
    \footnote{%
        A feszítőrang fogalmának ismerete nélküli -- talán még elemibb -- bizonyítás: \citep{doi:10.4169/college.math.j.48.5.366}%
    }%
\end{proof}
\begin{defprop}[invertálható mátrix]
    Legyen $A\in\mathbb{F}^{n\times n}$ egy négyzetes mátrix.
    Az alábbi feltételek egymással ekvivalensek.
    \begin{enumerate}
        \item Van egyetlen olyan $B\in\mathbb{F}^{n\times n}$ mátrix,
            amelyre $AB=I$,
        \item Van olyan $B\in\mathbb{F}^{n\times n}$ mátrix,
            amelyre $AB=I$,
        \item Van egyetlen olyan $B\in\mathbb{F}^{n\times n}$ mátrix,
            amelyre $BA=I$,
        \item Van olyan $B\in\mathbb{F}^{n\times n}$ mátrix,
            amelyre $BA=I$,
        \item $\rank A=n$,
        \item $A$ oszlopai lineárisan független rendszer alkotnak,
        \item $A$ sorai lineárisan független rendszert alkotnak.
    \end{enumerate}
    Ha a fenti feltételek egyike (ergo mindegyike) fennáll, 
    akkor azt mondjuk, hogy $A$ mátrix \emph{invertálható}\index{invertálható mátrix}.
    Szinonímaként használjuk még a \emph{nemszinguláris}\index{nemszinguláris mátrix}, 
    vagy az \emph{reguláris}\index{reguláris mátrix} szavakat.
    Ha egy mátrix nem invertálható, akkor \emph{szingulárisnak}\index{szinguláris mátrix} nevezzük.

    Egy invertálható négyzetes mátrix esetén azt az egyetlen $B$ mátrixot, amelyre
    \(
        AB=I
    \)
    fennáll az $A$ inverzének mondjuk, és $A^{-1}=B$, vel jelöljük.
    Világos, hogy
\[
    AA^{-1}=I=A^{-1}A,\quad (A^{-1})^{-1}=A.\qedhere
\]
\end{defprop}
\begin{proof}
    A 2., 4., 5., 6., 7. állítások ekvivalenciája nyilvánvaló az előzőek szerint.
    Ha $AB=I=AC$, akkor $A\left( B-C \right)=0$ így $A$ oszloprendszere lineáris függetlensége 
    miatt $B=C$. 
    Ezzel $2.\Rightarrow 1.$ implikációt is beláttuk.
    Az 1. és 3. feltevések ekvivalenciája az előző állítás miatt teljesül.
\end{proof}

\part{E-dúr hegedűverseny No. 1, Op. 8, RV 269, ,,La primavera''}
\backmatter
\pagestyle{empty}
\bibliographystyle{plainurl}
\bibliography{la.bib}
\printindex

\end{document}

\begin{thebibliography}{MMMMMMMMM}
    \bibitem[Dancs István, Puskás Csaba (2006)]{PCSDI}
        Dancs István, Puskás Csaba: \textit{Vektorterek}, Aula kiadó 2001, Budapest, ISBN:963 9345 53 9, BCE Catalogue: bcek.379187.
\end{thebibliography}

Created: Sat 20 Jul 2019 05:02:12 AM CEST
Last Modified : Thu 01 Aug 2019 05:50:20 AM CEST
